\documentclass[10pt, openany, draft]{article}

\usepackage{fancyhdr}
\usepackage{multind}
\usepackage{pstricks}
\usepackage{graphicx}
\usepackage[yyyymmdd]{datetime}
\renewcommand{\dateseparator}{-}
\usepackage{geometry}
\geometry{letterpaper}
%
% Front Matter
%
\title{Arduino Sensor Network Documentation -- Overview}
\author{Brent Seidel \\ Phoenix, AZ}
\date{ \today }
%========================================================
%%% BEGIN DOCUMENT
\begin{document}
\maketitle
\section{Synopsis}
This document will contain descriptions of the various components of the Arduino Sensor Network.  First is a brief overview of the various components and then a discussion of the communication protocol used between them.  An example network is shown in Figure \ref{fig:Example}.

\begin{figure}
  \centering
  %LaTeX with PSTricks extensions
%%Creator: inkscape 0.91
%%Please note this file requires PSTricks extensions
\psset{xunit=.5pt,yunit=.5pt,runit=.5pt}
\begin{pspicture}(649.10016398,303.04604778)
{
\newrgbcolor{curcolor}{0 0 0}
\pscustom[linewidth=5,linecolor=curcolor]
{
\newpath
\moveto(9.550069,280.33396778)
\lineto(639.550069,280.33396778)
}
}
{
\newrgbcolor{curcolor}{0 0 0}
\pscustom[linestyle=none,fillstyle=solid,fillcolor=curcolor]
{
\newpath
\moveto(29.550069,280.33396778)
\lineto(39.550069,290.33396778)
\lineto(4.550069,280.33396778)
\lineto(39.550069,270.33396778)
\lineto(29.550069,280.33396778)
\closepath
}
}
{
\newrgbcolor{curcolor}{0 0 0}
\pscustom[linewidth=2.5,linecolor=curcolor]
{
\newpath
\moveto(29.550069,280.33396778)
\lineto(39.550069,290.33396778)
\lineto(4.550069,280.33396778)
\lineto(39.550069,270.33396778)
\lineto(29.550069,280.33396778)
\closepath
}
}
{
\newrgbcolor{curcolor}{0 0 0}
\pscustom[linestyle=none,fillstyle=solid,fillcolor=curcolor]
{
\newpath
\moveto(619.550069,280.33396778)
\lineto(609.550069,270.33396778)
\lineto(644.550069,280.33396778)
\lineto(609.550069,290.33396778)
\lineto(619.550069,280.33396778)
\closepath
}
}
{
\newrgbcolor{curcolor}{0 0 0}
\pscustom[linewidth=2.5,linecolor=curcolor]
{
\newpath
\moveto(619.550069,280.33396778)
\lineto(609.550069,270.33396778)
\lineto(644.550069,280.33396778)
\lineto(609.550069,290.33396778)
\lineto(619.550069,280.33396778)
\closepath
}
}
{
\newrgbcolor{curcolor}{0 0 0}
\pscustom[linestyle=none,fillstyle=solid,fillcolor=curcolor]
{
\newpath
\moveto(56.43422427,296.77945025)
\curveto(57.25453677,296.77945025)(57.90199771,296.94351275)(58.37660709,297.27163775)
\curveto(58.85707584,297.59976275)(59.09731021,298.19155963)(59.09731021,299.04702838)
\curveto(59.09731021,299.96695025)(58.76332584,300.59390338)(58.09535709,300.92788775)
\curveto(57.73793521,301.103669)(57.26039615,301.19155963)(56.6627399,301.19155963)
\lineto(52.39125552,301.19155963)
\lineto(52.39125552,296.77945025)
\lineto(56.43422427,296.77945025)
\closepath
\moveto(50.64223209,302.69448931)
\lineto(56.61879459,302.69448931)
\curveto(57.60316959,302.69448931)(58.41469302,302.55093463)(59.0533649,302.26382525)
\curveto(60.26625552,301.713044)(60.87270084,300.69644244)(60.87270084,299.21402056)
\curveto(60.87270084,298.44058306)(60.71156802,297.80777056)(60.3893024,297.31558306)
\curveto(60.07289615,296.82339556)(59.62758365,296.42788775)(59.0533649,296.12905963)
\curveto(59.55727115,295.9239815)(59.93520084,295.65445025)(60.18715396,295.32046588)
\curveto(60.44496646,294.9864815)(60.58852115,294.44448931)(60.61781802,293.69448931)
\lineto(60.67934146,291.963044)
\curveto(60.69691959,291.4708565)(60.73793521,291.10464556)(60.80238834,290.86441119)
\curveto(60.90785709,290.45425494)(61.09535709,290.19058306)(61.36488834,290.07339556)
\lineto(61.36488834,289.7833565)
\lineto(59.22035709,289.7833565)
\curveto(59.16176334,289.89468463)(59.11488834,290.03823931)(59.07973209,290.21402056)
\curveto(59.04457584,290.38980181)(59.01527896,290.72964556)(58.99184146,291.23355181)
\lineto(58.88637271,293.38687213)
\curveto(58.84535709,294.23062213)(58.53188052,294.79605181)(57.94594302,295.08316119)
\curveto(57.61195865,295.24136431)(57.08754459,295.32046588)(56.37270084,295.32046588)
\lineto(52.39125552,295.32046588)
\lineto(52.39125552,289.7833565)
\lineto(50.64223209,289.7833565)
\lineto(50.64223209,302.69448931)
\closepath
}
}
{
\newrgbcolor{curcolor}{0 0 0}
\pscustom[linestyle=none,fillstyle=solid,fillcolor=curcolor]
{
\newpath
\moveto(64.58168521,293.94937213)
\curveto(64.62270084,293.21695025)(64.7955524,292.62222369)(65.1002399,292.16519244)
\curveto(65.68031802,291.30972369)(66.70277896,290.88198931)(68.16762271,290.88198931)
\curveto(68.82387271,290.88198931)(69.42152896,290.97573931)(69.96059146,291.16323931)
\curveto(71.00356021,291.52652056)(71.52504459,292.17691119)(71.52504459,293.11441119)
\curveto(71.52504459,293.81753619)(71.30531802,294.31851275)(70.8658649,294.61734088)
\curveto(70.4205524,294.91030963)(69.72328677,295.16519244)(68.77406802,295.38198931)
\lineto(67.02504459,295.77749713)
\curveto(65.88246646,296.03530963)(65.07387271,296.31948931)(64.59926334,296.63003619)
\curveto(63.77895084,297.16909869)(63.36879459,297.97476275)(63.36879459,299.04702838)
\curveto(63.36879459,300.20718463)(63.77016177,301.15933306)(64.57289615,301.90347369)
\curveto(65.37563052,302.64761431)(66.51234927,303.01968463)(67.9830524,303.01968463)
\curveto(69.33656802,303.01968463)(70.48500552,302.69155963)(71.4283649,302.03530963)
\curveto(72.37758365,301.384919)(72.85219302,300.34195025)(72.85219302,298.90640338)
\lineto(71.20863834,298.90640338)
\curveto(71.12074771,299.59780963)(70.93324771,300.12808306)(70.64613834,300.49722369)
\curveto(70.11293521,301.17105181)(69.20766177,301.50796588)(67.93031802,301.50796588)
\curveto(66.89906802,301.50796588)(66.15785709,301.291169)(65.70668521,300.85757525)
\curveto(65.25551334,300.4239815)(65.0299274,299.92007525)(65.0299274,299.3458565)
\curveto(65.0299274,298.713044)(65.29359927,298.25015338)(65.82094302,297.95718463)
\curveto(66.16664615,297.76968463)(66.94887271,297.53530963)(68.16762271,297.25405963)
\lineto(69.97816959,296.84097369)
\curveto(70.85121646,296.64175494)(71.52504459,296.369294)(71.99965396,296.02359088)
\curveto(72.81996646,295.42007525)(73.23012271,294.54409869)(73.23012271,293.39566119)
\curveto(73.23012271,291.96597369)(72.70863834,290.94351275)(71.66566959,290.32827838)
\curveto(70.62856021,289.713044)(69.42152896,289.40542681)(68.04457584,289.40542681)
\curveto(66.43910709,289.40542681)(65.18227115,289.81558306)(64.27406802,290.63589556)
\curveto(63.3658649,291.45034869)(62.9205524,292.55484088)(62.93813052,293.94937213)
\lineto(64.58168521,293.94937213)
\closepath
\moveto(68.11488834,303.04605181)
\lineto(68.11488834,303.04605181)
\closepath
}
}
{
\newrgbcolor{curcolor}{0 0 0}
\pscustom[linestyle=none,fillstyle=solid,fillcolor=curcolor]
{
\newpath
\moveto(74.8033649,295.61050494)
\lineto(79.21547427,295.61050494)
\lineto(79.21547427,293.98452838)
\lineto(74.8033649,293.98452838)
\lineto(74.8033649,295.61050494)
\closepath
}
}
{
\newrgbcolor{curcolor}{0 0 0}
\pscustom[linestyle=none,fillstyle=solid,fillcolor=curcolor]
{
\newpath
\moveto(86.01820865,294.23941119)
\lineto(86.01820865,299.94351275)
\lineto(81.98402896,294.23941119)
\lineto(86.01820865,294.23941119)
\closepath
\moveto(86.04457584,289.7833565)
\lineto(86.04457584,292.85952838)
\lineto(80.52504459,292.85952838)
\lineto(80.52504459,294.40640338)
\lineto(86.29066959,302.40445025)
\lineto(87.62660709,302.40445025)
\lineto(87.62660709,294.23941119)
\lineto(89.48109927,294.23941119)
\lineto(89.48109927,292.85952838)
\lineto(87.62660709,292.85952838)
\lineto(87.62660709,289.7833565)
\lineto(86.04457584,289.7833565)
\closepath
}
}
{
\newrgbcolor{curcolor}{0 0 0}
\pscustom[linestyle=none,fillstyle=solid,fillcolor=curcolor]
{
\newpath
\moveto(94.9830524,297.0958565)
\curveto(95.68031802,297.0958565)(96.2252399,297.28921588)(96.61781802,297.67593463)
\curveto(97.01039615,298.06851275)(97.20668521,298.53433306)(97.20668521,299.07339556)
\curveto(97.20668521,299.54214556)(97.01918521,299.97280963)(96.64418521,300.36538775)
\curveto(96.26918521,300.75796588)(95.69789615,300.95425494)(94.93031802,300.95425494)
\curveto(94.16859927,300.95425494)(93.61781802,300.75796588)(93.27797427,300.36538775)
\curveto(92.93813052,299.97280963)(92.76820865,299.51284869)(92.76820865,298.98550494)
\curveto(92.76820865,298.39370806)(92.98793521,297.93081744)(93.42738834,297.59683306)
\curveto(93.86684146,297.26284869)(94.38539615,297.0958565)(94.9830524,297.0958565)
\closepath
\moveto(95.07973209,290.86441119)
\curveto(95.81215396,290.86441119)(96.41859927,291.06070025)(96.89906802,291.45327838)
\curveto(97.38539615,291.85171588)(97.62856021,292.44351275)(97.62856021,293.228669)
\curveto(97.62856021,294.04312213)(97.37953677,294.66128619)(96.8814899,295.08316119)
\curveto(96.38344302,295.50503619)(95.74477115,295.71597369)(94.96547427,295.71597369)
\curveto(94.2096149,295.71597369)(93.59145084,295.49917681)(93.11098209,295.06558306)
\curveto(92.63637271,294.63784869)(92.39906802,294.04312213)(92.39906802,293.28140338)
\curveto(92.39906802,292.62515338)(92.6158649,292.056794)(93.04945865,291.57632525)
\curveto(93.48891177,291.10171588)(94.16566959,290.86441119)(95.07973209,290.86441119)
\closepath
\moveto(92.82973209,296.48941119)
\curveto(92.39027896,296.67691119)(92.04750552,296.89663775)(91.80141177,297.14859088)
\curveto(91.33852115,297.61734088)(91.10707584,298.22671588)(91.10707584,298.97671588)
\curveto(91.10707584,299.91421588)(91.44691959,300.71987994)(92.12660709,301.39370806)
\curveto(92.80629459,302.06753619)(93.77016177,302.40445025)(95.01820865,302.40445025)
\curveto(96.2252399,302.40445025)(97.17152896,302.08511431)(97.85707584,301.44644244)
\curveto(98.54262271,300.81362994)(98.88539615,300.072419)(98.88539615,299.22280963)
\curveto(98.88539615,298.43765338)(98.6861774,297.80191119)(98.2877399,297.31558306)
\curveto(98.06508365,297.04019244)(97.71938052,296.77066119)(97.25063052,296.50698931)
\curveto(97.7721149,296.26675494)(98.18227115,295.99136431)(98.48109927,295.68081744)
\curveto(99.0377399,295.09487994)(99.31606021,294.33316119)(99.31606021,293.39566119)
\curveto(99.31606021,292.28823931)(98.9439899,291.34780963)(98.19984927,290.57437213)
\curveto(97.45570865,289.806794)(96.40395084,289.42300494)(95.04457584,289.42300494)
\curveto(93.81996646,289.42300494)(92.78285709,289.75405963)(91.93324771,290.416169)
\curveto(91.08949771,291.08413775)(90.66762271,292.05093463)(90.66762271,293.31655963)
\curveto(90.66762271,294.06070025)(90.84926334,294.70230181)(91.21254459,295.24136431)
\curveto(91.57582584,295.78628619)(92.11488834,296.20230181)(92.82973209,296.48941119)
\closepath
}
}
{
\newrgbcolor{curcolor}{0 0 0}
\pscustom[linestyle=none,fillstyle=solid,fillcolor=curcolor]
{
\newpath
\moveto(102.33070865,292.99136431)
\curveto(102.4361774,292.08902056)(102.85512271,291.46499713)(103.58754459,291.119294)
\curveto(103.96254459,290.94351275)(104.39613834,290.85562213)(104.88832584,290.85562213)
\curveto(105.82582584,290.85562213)(106.52016177,291.15445025)(106.97133365,291.7521065)
\curveto(107.42250552,292.34976275)(107.64809146,293.01187213)(107.64809146,293.73843463)
\curveto(107.64809146,294.61734088)(107.37856021,295.29702838)(106.83949771,295.77749713)
\curveto(106.30629459,296.25796588)(105.66469302,296.49820025)(104.91469302,296.49820025)
\curveto(104.36977115,296.49820025)(103.90102115,296.3927315)(103.50844302,296.181794)
\curveto(103.12172427,295.9708565)(102.79066959,295.67788775)(102.51527896,295.30288775)
\lineto(101.14418521,295.38198931)
\lineto(102.10219302,302.1583565)
\lineto(108.64125552,302.1583565)
\lineto(108.64125552,300.62905963)
\lineto(103.28871646,300.62905963)
\lineto(102.75258365,297.13101275)
\curveto(103.0455524,297.353669)(103.32387271,297.52066119)(103.58754459,297.63198931)
\curveto(104.05629459,297.82534869)(104.59828677,297.92202838)(105.21352115,297.92202838)
\curveto(106.36781802,297.92202838)(107.34633365,297.54995806)(108.14906802,296.80581744)
\curveto(108.9518024,296.06167681)(109.35316959,295.11831744)(109.35316959,293.97573931)
\curveto(109.35316959,292.78628619)(108.98402896,291.73745806)(108.24574771,290.82925494)
\curveto(107.51332584,289.92105181)(106.34145084,289.46695025)(104.73012271,289.46695025)
\curveto(103.70473209,289.46695025)(102.79652896,289.75405963)(102.00551334,290.32827838)
\curveto(101.22035709,290.9083565)(100.78090396,291.79605181)(100.68715396,292.99136431)
\lineto(102.33070865,292.99136431)
\closepath
}
}
{
\newrgbcolor{curcolor}{0 0 0}
\pscustom[linestyle=none,fillstyle=solid,fillcolor=curcolor]
{
\newpath
\moveto(121.34145084,297.2364815)
\curveto(122.07973209,297.2364815)(122.65395084,297.33902056)(123.06410709,297.54409869)
\curveto(123.70863834,297.86636431)(124.03090396,298.44644244)(124.03090396,299.28433306)
\curveto(124.03090396,300.12808306)(123.68813052,300.69644244)(123.00258365,300.98941119)
\curveto(122.6158649,301.15347369)(122.04164615,301.23550494)(121.2799274,301.23550494)
\lineto(118.15981021,301.23550494)
\lineto(118.15981021,297.2364815)
\lineto(121.34145084,297.2364815)
\closepath
\moveto(121.93031802,291.27749713)
\curveto(123.00258365,291.27749713)(123.76723209,291.588044)(124.22426334,292.20913775)
\curveto(124.51137271,292.60171588)(124.6549274,293.07632525)(124.6549274,293.63296588)
\curveto(124.6549274,294.57046588)(124.23598209,295.20913775)(123.39809146,295.5489815)
\curveto(122.95277896,295.73062213)(122.36391177,295.82144244)(121.6314899,295.82144244)
\lineto(118.15981021,295.82144244)
\lineto(118.15981021,291.27749713)
\lineto(121.93031802,291.27749713)
\closepath
\moveto(116.44594302,302.69448931)
\lineto(121.99184146,302.69448931)
\curveto(123.50356021,302.69448931)(124.57875552,302.24331744)(125.2174274,301.34097369)
\curveto(125.5924274,300.80777056)(125.7799274,300.19253619)(125.7799274,299.49527056)
\curveto(125.7799274,298.68081744)(125.54848209,298.01284869)(125.08559146,297.49136431)
\curveto(124.84535709,297.21597369)(124.49965396,296.96402056)(124.04848209,296.73550494)
\curveto(124.71059146,296.48355181)(125.20570865,296.19937213)(125.53383365,295.88296588)
\curveto(126.11391177,295.32046588)(126.40395084,294.54409869)(126.40395084,293.55386431)
\curveto(126.40395084,292.72183306)(126.14320865,291.96890338)(125.62172427,291.29507525)
\curveto(124.8424274,290.28726275)(123.60316959,289.7833565)(121.90395084,289.7833565)
\lineto(116.44594302,289.7833565)
\lineto(116.44594302,302.69448931)
\closepath
}
}
{
\newrgbcolor{curcolor}{0 0 0}
\pscustom[linestyle=none,fillstyle=solid,fillcolor=curcolor]
{
\newpath
\moveto(129.88441959,299.19644244)
\lineto(129.88441959,292.947419)
\curveto(129.88441959,292.46695025)(129.96059146,292.07437213)(130.11293521,291.76968463)
\curveto(130.39418521,291.20718463)(130.91859927,290.92593463)(131.6861774,290.92593463)
\curveto(132.7877399,290.92593463)(133.5377399,291.41812213)(133.9361774,292.40249713)
\curveto(134.15297427,292.92984088)(134.26137271,293.65347369)(134.26137271,294.57339556)
\lineto(134.26137271,299.19644244)
\lineto(135.84340396,299.19644244)
\lineto(135.84340396,289.7833565)
\lineto(134.34926334,289.7833565)
\lineto(134.36684146,291.17202838)
\curveto(134.16176334,290.8146065)(133.90688052,290.51284869)(133.60219302,290.26675494)
\curveto(132.9986774,289.77456744)(132.26625552,289.52847369)(131.4049274,289.52847369)
\curveto(130.06313052,289.52847369)(129.14906802,289.97671588)(128.6627399,290.87320025)
\curveto(128.39906802,291.353669)(128.26723209,291.99527056)(128.26723209,292.79800494)
\lineto(128.26723209,299.19644244)
\lineto(129.88441959,299.19644244)
\closepath
\moveto(132.05531802,299.42495806)
\lineto(132.05531802,299.42495806)
\closepath
}
}
{
\newrgbcolor{curcolor}{0 0 0}
\pscustom[linestyle=none,fillstyle=solid,fillcolor=curcolor]
{
\newpath
\moveto(139.26234927,292.7364815)
\curveto(139.30922427,292.20913775)(139.44106021,291.80484088)(139.65785709,291.52359088)
\curveto(140.05629459,291.01382525)(140.74770084,290.75894244)(141.73207584,290.75894244)
\curveto(142.31801334,290.75894244)(142.83363834,290.884919)(143.27895084,291.13687213)
\curveto(143.72426334,291.39468463)(143.94691959,291.79019244)(143.94691959,292.32339556)
\curveto(143.94691959,292.72769244)(143.76820865,293.03530963)(143.41078677,293.24624713)
\curveto(143.18227115,293.37515338)(142.73109927,293.52456744)(142.05727115,293.69448931)
\lineto(140.80043521,294.01089556)
\curveto(139.99770084,294.21011431)(139.40590396,294.43277056)(139.02504459,294.67886431)
\curveto(138.34535709,295.10659869)(138.00551334,295.69839556)(138.00551334,296.45425494)
\curveto(138.00551334,297.34487994)(138.32484927,298.06558306)(138.96352115,298.61636431)
\curveto(139.6080524,299.16714556)(140.47231021,299.44253619)(141.55629459,299.44253619)
\curveto(142.97426334,299.44253619)(143.99672427,299.02652056)(144.6236774,298.19448931)
\curveto(145.01625552,297.66714556)(145.20668521,297.09878619)(145.19496646,296.48941119)
\lineto(143.70082584,296.48941119)
\curveto(143.67152896,296.84683306)(143.5455524,297.17202838)(143.32289615,297.46499713)
\curveto(142.9596149,297.88101275)(142.32973209,298.08902056)(141.43324771,298.08902056)
\curveto(140.83559146,298.08902056)(140.3814899,297.97476275)(140.07094302,297.74624713)
\curveto(139.76625552,297.5177315)(139.61391177,297.21597369)(139.61391177,296.84097369)
\curveto(139.61391177,296.43081744)(139.81606021,296.10269244)(140.22035709,295.85659869)
\curveto(140.45473209,295.71011431)(140.80043521,295.58120806)(141.25746646,295.46987994)
\lineto(142.3033649,295.21499713)
\curveto(143.44008365,294.9396065)(144.2018024,294.67300494)(144.58852115,294.41519244)
\curveto(145.20375552,294.01089556)(145.51137271,293.37515338)(145.51137271,292.50796588)
\curveto(145.51137271,291.67007525)(145.19203677,290.94644244)(144.5533649,290.33706744)
\curveto(143.9205524,289.72769244)(142.95375552,289.42300494)(141.65297427,289.42300494)
\curveto(140.25258365,289.42300494)(139.25941959,289.73941119)(138.67348209,290.37222369)
\curveto(138.09340396,291.01089556)(137.78285709,291.7989815)(137.74184146,292.7364815)
\lineto(139.26234927,292.7364815)
\closepath
\moveto(141.6002399,299.42495806)
\lineto(141.6002399,299.42495806)
\closepath
}
}
{
\newrgbcolor{curcolor}{0 0 0}
\pscustom[linewidth=1.06423664,linecolor=curcolor]
{
\newpath
\moveto(29.58218875,220.30185015)
\lineto(108.93697848,220.30185015)
\lineto(108.93697848,150.36608966)
\lineto(29.58218875,150.36608966)
\closepath
}
}
{
\newrgbcolor{curcolor}{0 0 0}
\pscustom[linewidth=1,linecolor=curcolor]
{
\newpath
\moveto(69.550069,280.33396778)
\lineto(69.550069,220.33396778)
}
}
{
\newrgbcolor{curcolor}{0 0 0}
\pscustom[linestyle=none,fillstyle=solid,fillcolor=curcolor]
{
\newpath
\moveto(69.550069,224.33396778)
\lineto(67.550069,226.33396778)
\lineto(69.550069,219.33396778)
\lineto(71.550069,226.33396778)
\lineto(69.550069,224.33396778)
\closepath
}
}
{
\newrgbcolor{curcolor}{0 0 0}
\pscustom[linewidth=0.5,linecolor=curcolor]
{
\newpath
\moveto(69.550069,224.33396778)
\lineto(67.550069,226.33396778)
\lineto(69.550069,219.33396778)
\lineto(71.550069,226.33396778)
\lineto(69.550069,224.33396778)
\closepath
}
}
{
\newrgbcolor{curcolor}{0 0 0}
\pscustom[linestyle=none,fillstyle=solid,fillcolor=curcolor]
{
\newpath
\moveto(40.50221744,189.02472003)
\lineto(41.71681705,189.02472003)
\lineto(41.71681705,181.12677081)
\lineto(46.25783267,181.12677081)
\lineto(46.25783267,180.05865557)
\lineto(40.50221744,180.05865557)
\lineto(40.50221744,189.02472003)
\closepath
}
}
{
\newrgbcolor{curcolor}{0 0 0}
\pscustom[linestyle=none,fillstyle=solid,fillcolor=curcolor]
{
\newpath
\moveto(47.31374088,186.56500323)
\lineto(48.43068423,186.56500323)
\lineto(48.43068423,180.05865557)
\lineto(47.31374088,180.05865557)
\lineto(47.31374088,186.56500323)
\closepath
\moveto(47.31374088,189.02472003)
\lineto(48.43068423,189.02472003)
\lineto(48.43068423,187.77960284)
\lineto(47.31374088,187.77960284)
\lineto(47.31374088,189.02472003)
\closepath
}
}
{
\newrgbcolor{curcolor}{0 0 0}
\pscustom[linestyle=none,fillstyle=solid,fillcolor=curcolor]
{
\newpath
\moveto(50.75002017,182.10943682)
\curveto(50.78257226,181.74322589)(50.87412499,181.46246417)(51.02467838,181.26715167)
\curveto(51.30137108,180.91314776)(51.78151431,180.73614581)(52.46510806,180.73614581)
\curveto(52.8720091,180.73614581)(53.23008202,180.82362953)(53.53932681,180.99859698)
\curveto(53.8485716,181.17763344)(54.003194,181.45229164)(54.003194,181.82257159)
\curveto(54.003194,182.10333331)(53.87908918,182.31695636)(53.63087955,182.46344073)
\curveto(53.47218814,182.55295896)(53.15887434,182.65671873)(52.69093814,182.77472003)
\lineto(51.81813541,182.99444659)
\curveto(51.26068098,183.13279294)(50.84971093,183.28741534)(50.58522525,183.45831378)
\curveto(50.11322004,183.75535154)(49.87721744,184.16632159)(49.87721744,184.69122393)
\curveto(49.87721744,185.30971352)(50.09897851,185.8102018)(50.54250064,186.19268878)
\curveto(50.99009179,186.57517576)(51.59027082,186.76641925)(52.34303775,186.76641925)
\curveto(53.32773827,186.76641925)(54.03778059,186.47751951)(54.4731647,185.89972003)
\curveto(54.7457884,185.53350909)(54.87803124,185.13881508)(54.86989322,184.715638)
\lineto(53.83229556,184.715638)
\curveto(53.81195051,184.96384763)(53.72446679,185.18967771)(53.56984439,185.39312823)
\curveto(53.31756574,185.68202797)(52.88014713,185.82647784)(52.25758853,185.82647784)
\curveto(51.84254947,185.82647784)(51.52720116,185.74713214)(51.31154361,185.58844073)
\curveto(51.09995507,185.42974932)(50.9941608,185.22019529)(50.9941608,184.95977862)
\curveto(50.9941608,184.67494789)(51.13454166,184.44708331)(51.41530338,184.27618487)
\curveto(51.57806379,184.17445961)(51.81813541,184.08494138)(52.13551822,184.00763018)
\lineto(52.86183658,183.83062823)
\curveto(53.6512246,183.63938474)(54.18019595,183.45424477)(54.44875064,183.27520831)
\curveto(54.87599673,182.99444659)(55.08961978,182.55295896)(55.08961978,181.95074542)
\curveto(55.08961978,181.36887693)(54.86785871,180.86635414)(54.42433658,180.44317706)
\curveto(53.98488345,180.01999998)(53.31349673,179.80841143)(52.41017642,179.80841143)
\curveto(51.43768293,179.80841143)(50.74798567,180.028138)(50.34108463,180.46759112)
\curveto(49.93825259,180.91111326)(49.72259504,181.45839516)(49.69411197,182.10943682)
\lineto(50.75002017,182.10943682)
\closepath
\moveto(52.37355533,186.75421221)
\lineto(52.37355533,186.75421221)
\closepath
}
}
{
\newrgbcolor{curcolor}{0 0 0}
\pscustom[linestyle=none,fillstyle=solid,fillcolor=curcolor]
{
\newpath
\moveto(56.59108463,188.42047198)
\lineto(57.70192447,188.42047198)
\lineto(57.70192447,186.59552081)
\lineto(58.74562564,186.59552081)
\lineto(58.74562564,185.69830401)
\lineto(57.70192447,185.69830401)
\lineto(57.70192447,181.43194659)
\curveto(57.70192447,181.20408201)(57.77923567,181.05149412)(57.93385806,180.97418292)
\curveto(58.01930728,180.9294238)(58.16172265,180.90704425)(58.36110416,180.90704425)
\lineto(58.53200259,180.90704425)
\curveto(58.59303775,180.91111326)(58.66424543,180.91721677)(58.74562564,180.92535479)
\lineto(58.74562564,180.05865557)
\curveto(58.61948632,180.02203448)(58.48724348,179.99558591)(58.34889713,179.97930987)
\curveto(58.21461978,179.96303383)(58.06813541,179.95489581)(57.909444,179.95489581)
\curveto(57.39674869,179.95489581)(57.0488483,180.08510414)(56.86574283,180.34552081)
\curveto(56.68263736,180.61000649)(56.59108463,180.95180336)(56.59108463,181.37091143)
\lineto(56.59108463,185.69830401)
\lineto(55.70607486,185.69830401)
\lineto(55.70607486,186.59552081)
\lineto(56.59108463,186.59552081)
\lineto(56.59108463,188.42047198)
\closepath
}
}
{
\newrgbcolor{curcolor}{0 0 0}
\pscustom[linestyle=none,fillstyle=solid,fillcolor=curcolor]
{
\newpath
\moveto(62.58473697,186.74200518)
\curveto(63.04860416,186.74200518)(63.49822981,186.6321419)(63.93361392,186.41241534)
\curveto(64.36899804,186.19675779)(64.70062239,185.91599607)(64.92848697,185.57013018)
\curveto(65.14821353,185.24054034)(65.29469791,184.85601886)(65.36794009,184.41656573)
\curveto(65.43304426,184.11545896)(65.46559634,183.63531573)(65.46559634,182.97613604)
\lineto(60.67433658,182.97613604)
\curveto(60.69468163,182.31288735)(60.85133853,181.77984698)(61.14430728,181.37701495)
\curveto(61.43727603,180.97825193)(61.89097069,180.77887042)(62.50539127,180.77887042)
\curveto(63.07912173,180.77887042)(63.53688541,180.9680794)(63.87868228,181.34649737)
\curveto(64.07399478,181.56622393)(64.21234114,181.82053708)(64.29372134,182.10943682)
\lineto(65.37404361,182.10943682)
\curveto(65.34556054,181.86936521)(65.24993879,181.60081052)(65.08717838,181.30377276)
\curveto(64.92848697,181.01080401)(64.74945051,180.7707324)(64.550069,180.58355792)
\curveto(64.21641015,180.25803708)(63.80340559,180.03831052)(63.31105533,179.92437823)
\curveto(63.04656965,179.85927406)(62.74749739,179.82672198)(62.41383853,179.82672198)
\curveto(61.60003645,179.82672198)(60.91033918,180.12172524)(60.34474673,180.71173175)
\curveto(59.77915429,181.30580727)(59.49635806,182.13588539)(59.49635806,183.20196612)
\curveto(59.49635806,184.25177081)(59.78118879,185.10422849)(60.35085025,185.75933917)
\curveto(60.92051171,186.41444985)(61.66514061,186.74200518)(62.58473697,186.74200518)
\closepath
\moveto(64.33644595,183.84893878)
\curveto(64.29168684,184.325013)(64.18792707,184.70546547)(64.02516666,184.9902962)
\curveto(63.72405989,185.51926755)(63.2215371,185.78375323)(62.5175983,185.78375323)
\curveto(62.01304101,185.78375323)(61.58986392,185.60064776)(61.24806705,185.23443682)
\curveto(60.90627017,184.8722949)(60.72519921,184.41046221)(60.70485416,183.84893878)
\lineto(64.33644595,183.84893878)
\closepath
\moveto(62.4809772,186.75421221)
\lineto(62.4809772,186.75421221)
\closepath
}
}
{
\newrgbcolor{curcolor}{0 0 0}
\pscustom[linestyle=none,fillstyle=solid,fillcolor=curcolor]
{
\newpath
\moveto(66.82057681,186.59552081)
\lineto(67.86427798,186.59552081)
\lineto(67.86427798,185.66778643)
\curveto(68.17352278,186.05027341)(68.50107811,186.32493162)(68.846944,186.49176104)
\curveto(69.19280989,186.65859047)(69.57733137,186.74200518)(70.00050845,186.74200518)
\curveto(70.92824283,186.74200518)(71.55487043,186.41851886)(71.88039127,185.7715462)
\curveto(72.05942772,185.41754229)(72.14894595,184.9109505)(72.14894595,184.25177081)
\lineto(72.14894595,180.05865557)
\lineto(71.03200259,180.05865557)
\lineto(71.03200259,184.17852862)
\curveto(71.03200259,184.57729164)(70.97300194,184.89874346)(70.85500064,185.14288409)
\curveto(70.65968814,185.54978513)(70.30568423,185.75323565)(69.79298892,185.75323565)
\curveto(69.53257226,185.75323565)(69.31894921,185.72678708)(69.15211978,185.67388995)
\curveto(68.85101301,185.58437172)(68.58652733,185.40533526)(68.35866275,185.13678057)
\curveto(68.17555728,184.92112302)(68.05552147,184.69732745)(67.99855533,184.46539386)
\curveto(67.94565819,184.23752927)(67.91920963,183.90997393)(67.91920963,183.48272784)
\lineto(67.91920963,180.05865557)
\lineto(66.82057681,180.05865557)
\lineto(66.82057681,186.59552081)
\closepath
\moveto(69.40236392,186.75421221)
\lineto(69.40236392,186.75421221)
\closepath
}
}
{
\newrgbcolor{curcolor}{0 0 0}
\pscustom[linestyle=none,fillstyle=solid,fillcolor=curcolor]
{
\newpath
\moveto(76.50075259,186.74200518)
\curveto(76.96461978,186.74200518)(77.41424543,186.6321419)(77.84962955,186.41241534)
\curveto(78.28501366,186.19675779)(78.61663801,185.91599607)(78.84450259,185.57013018)
\curveto(79.06422916,185.24054034)(79.21071353,184.85601886)(79.28395572,184.41656573)
\curveto(79.34905989,184.11545896)(79.38161197,183.63531573)(79.38161197,182.97613604)
\lineto(74.5903522,182.97613604)
\curveto(74.61069726,182.31288735)(74.76735416,181.77984698)(75.06032291,181.37701495)
\curveto(75.35329166,180.97825193)(75.80698632,180.77887042)(76.42140689,180.77887042)
\curveto(76.99513736,180.77887042)(77.45290103,180.9680794)(77.79469791,181.34649737)
\curveto(77.99001041,181.56622393)(78.12835676,181.82053708)(78.20973697,182.10943682)
\lineto(79.29005923,182.10943682)
\curveto(79.26157616,181.86936521)(79.16595442,181.60081052)(79.003194,181.30377276)
\curveto(78.84450259,181.01080401)(78.66546614,180.7707324)(78.46608463,180.58355792)
\curveto(78.13242577,180.25803708)(77.71942121,180.03831052)(77.22707095,179.92437823)
\curveto(76.96258528,179.85927406)(76.66351301,179.82672198)(76.32985416,179.82672198)
\curveto(75.51605207,179.82672198)(74.82635481,180.12172524)(74.26076236,180.71173175)
\curveto(73.69516991,181.30580727)(73.41237369,182.13588539)(73.41237369,183.20196612)
\curveto(73.41237369,184.25177081)(73.69720442,185.10422849)(74.26686588,185.75933917)
\curveto(74.83652733,186.41444985)(75.58115624,186.74200518)(76.50075259,186.74200518)
\closepath
\moveto(78.25246158,183.84893878)
\curveto(78.20770246,184.325013)(78.1039427,184.70546547)(77.94118228,184.9902962)
\curveto(77.64007551,185.51926755)(77.13755272,185.78375323)(76.43361392,185.78375323)
\curveto(75.92905663,185.78375323)(75.50587955,185.60064776)(75.16408267,185.23443682)
\curveto(74.8222858,184.8722949)(74.64121483,184.41046221)(74.62086978,183.84893878)
\lineto(78.25246158,183.84893878)
\closepath
\moveto(76.39699283,186.75421221)
\lineto(76.39699283,186.75421221)
\closepath
}
}
{
\newrgbcolor{curcolor}{0 0 0}
\pscustom[linestyle=none,fillstyle=solid,fillcolor=curcolor]
{
\newpath
\moveto(80.76711002,186.59552081)
\lineto(81.81081119,186.59552081)
\lineto(81.81081119,185.46637042)
\curveto(81.89626041,185.68609698)(82.10581444,185.95261716)(82.4394733,186.26593096)
\curveto(82.77313215,186.58331378)(83.15765364,186.74200518)(83.59303775,186.74200518)
\curveto(83.6133828,186.74200518)(83.64796939,186.73997068)(83.69679752,186.73590167)
\curveto(83.74562564,186.73183266)(83.82904035,186.72369464)(83.94704166,186.71148761)
\lineto(83.94704166,185.55181964)
\curveto(83.88193749,185.56402667)(83.82090233,185.57216469)(83.76393619,185.5762337)
\curveto(83.71103905,185.58030271)(83.6520384,185.58233721)(83.58693423,185.58233721)
\curveto(83.03354882,185.58233721)(82.60833723,185.40330076)(82.31129947,185.04522784)
\curveto(82.01426171,184.69122393)(81.86574283,184.28228839)(81.86574283,183.8184212)
\lineto(81.86574283,180.05865557)
\lineto(80.76711002,180.05865557)
\lineto(80.76711002,186.59552081)
\closepath
}
}
{
\newrgbcolor{curcolor}{0 0 0}
\pscustom[linewidth=1.06423664,linecolor=curcolor]
{
\newpath
\moveto(140.04169192,218.58929522)
\lineto(219.39648166,218.58929522)
\lineto(219.39648166,148.65353472)
\lineto(140.04169192,148.65353472)
\closepath
}
}
{
\newrgbcolor{curcolor}{0 0 0}
\pscustom[linewidth=1,linecolor=curcolor]
{
\newpath
\moveto(179.550069,280.33396778)
\lineto(179.550069,220.33396778)
}
}
{
\newrgbcolor{curcolor}{0 0 0}
\pscustom[linestyle=none,fillstyle=solid,fillcolor=curcolor]
{
\newpath
\moveto(179.550069,276.33396778)
\lineto(181.550069,274.33396778)
\lineto(179.550069,281.33396778)
\lineto(177.550069,274.33396778)
\lineto(179.550069,276.33396778)
\closepath
}
}
{
\newrgbcolor{curcolor}{0 0 0}
\pscustom[linewidth=0.5,linecolor=curcolor]
{
\newpath
\moveto(179.550069,276.33396778)
\lineto(181.550069,274.33396778)
\lineto(179.550069,281.33396778)
\lineto(177.550069,274.33396778)
\lineto(179.550069,276.33396778)
\closepath
}
}
{
\newrgbcolor{curcolor}{0 0 0}
\pscustom[linestyle=none,fillstyle=solid,fillcolor=curcolor]
{
\newpath
\moveto(179.550069,224.33396778)
\lineto(177.550069,226.33396778)
\lineto(179.550069,219.33396778)
\lineto(181.550069,226.33396778)
\lineto(179.550069,224.33396778)
\closepath
}
}
{
\newrgbcolor{curcolor}{0 0 0}
\pscustom[linewidth=0.5,linecolor=curcolor]
{
\newpath
\moveto(179.550069,224.33396778)
\lineto(177.550069,226.33396778)
\lineto(179.550069,219.33396778)
\lineto(181.550069,226.33396778)
\lineto(179.550069,224.33396778)
\closepath
}
}
{
\newrgbcolor{curcolor}{0 0 0}
\pscustom[linestyle=none,fillstyle=solid,fillcolor=curcolor]
{
\newpath
\moveto(154.67091861,185.19235308)
\curveto(155.24058007,185.19235308)(155.69020572,185.30628538)(156.01979556,185.53414996)
\curveto(156.35345442,185.76201454)(156.52028384,186.17298459)(156.52028384,186.76706012)
\curveto(156.52028384,187.40589475)(156.28835025,187.84127887)(155.82448306,188.07321246)
\curveto(155.57627343,188.19528277)(155.24464908,188.25631793)(154.82961002,188.25631793)
\lineto(151.86330142,188.25631793)
\lineto(151.86330142,185.19235308)
\lineto(154.67091861,185.19235308)
\closepath
\moveto(150.64870181,189.3000191)
\lineto(154.79909244,189.3000191)
\curveto(155.48268619,189.3000191)(156.04624413,189.20032834)(156.48976627,189.00094683)
\curveto(157.33205142,188.61845985)(157.753194,187.91248655)(157.753194,186.88302691)
\curveto(157.753194,186.34591754)(157.64129621,185.90646441)(157.41750064,185.56466754)
\curveto(157.19777408,185.22287066)(156.88852929,184.94821246)(156.48976627,184.74069293)
\curveto(156.83970116,184.59827756)(157.10215233,184.41110308)(157.27711978,184.17916949)
\curveto(157.45615624,183.9472359)(157.55584699,183.57085243)(157.57619205,183.0500191)
\lineto(157.61891666,181.84762652)
\curveto(157.63112369,181.50582965)(157.65960676,181.2515165)(157.70436588,181.08468707)
\curveto(157.77760806,180.79985634)(157.9078164,180.61675087)(158.09499088,180.53537066)
\lineto(158.09499088,180.33395465)
\lineto(156.60573306,180.33395465)
\curveto(156.56504296,180.41126584)(156.53249088,180.5109566)(156.50807681,180.63302691)
\curveto(156.48366275,180.75509722)(156.4633177,180.99109983)(156.44704166,181.34103472)
\lineto(156.37379947,182.83639605)
\curveto(156.3453164,183.42233355)(156.12762434,183.81499306)(155.7207233,184.01437457)
\curveto(155.4887897,184.12423785)(155.12461327,184.17916949)(154.628194,184.17916949)
\lineto(151.86330142,184.17916949)
\lineto(151.86330142,180.33395465)
\lineto(150.64870181,180.33395465)
\lineto(150.64870181,189.3000191)
\closepath
}
}
{
\newrgbcolor{curcolor}{0 0 0}
\pscustom[linestyle=none,fillstyle=solid,fillcolor=curcolor]
{
\newpath
\moveto(162.11110416,187.01730426)
\curveto(162.57497134,187.01730426)(163.02459699,186.90744097)(163.45998111,186.68771441)
\curveto(163.89536522,186.47205686)(164.22698957,186.19129514)(164.45485416,185.84542926)
\curveto(164.67458072,185.51583941)(164.82106509,185.13131793)(164.89430728,184.6918648)
\curveto(164.95941145,184.39075803)(164.99196353,183.9106148)(164.99196353,183.25143512)
\lineto(160.20070377,183.25143512)
\curveto(160.22104882,182.58818642)(160.37770572,182.05514605)(160.67067447,181.65231402)
\curveto(160.96364322,181.253551)(161.41733788,181.05416949)(162.03175845,181.05416949)
\curveto(162.60548892,181.05416949)(163.06325259,181.24337847)(163.40504947,181.62179644)
\curveto(163.60036197,181.84152301)(163.73870832,182.09583616)(163.82008853,182.3847359)
\lineto(164.9004108,182.3847359)
\curveto(164.87192772,182.14466428)(164.77630598,181.87610959)(164.61354556,181.57907183)
\curveto(164.45485416,181.28610308)(164.2758177,181.04603147)(164.07643619,180.85885699)
\curveto(163.74277733,180.53333616)(163.32977278,180.31360959)(162.83742252,180.1996773)
\curveto(162.57293684,180.13457314)(162.27386457,180.10202105)(161.94020572,180.10202105)
\curveto(161.12640364,180.10202105)(160.43670637,180.39702431)(159.87111392,180.98703082)
\curveto(159.30552147,181.58110634)(159.02272525,182.41118446)(159.02272525,183.47726519)
\curveto(159.02272525,184.52706988)(159.30755598,185.37952756)(159.87721744,186.03463824)
\curveto(160.4468789,186.68974892)(161.1915078,187.01730426)(162.11110416,187.01730426)
\closepath
\moveto(163.86281314,184.12423785)
\curveto(163.81805403,184.60031207)(163.71429426,184.98076454)(163.55153384,185.26559527)
\curveto(163.25042707,185.79456663)(162.74790429,186.0590523)(162.04396548,186.0590523)
\curveto(161.53940819,186.0590523)(161.11623111,185.87594683)(160.77443423,185.5097359)
\curveto(160.43263736,185.14759397)(160.2515664,184.68576129)(160.23122134,184.12423785)
\lineto(163.86281314,184.12423785)
\closepath
\moveto(162.00734439,187.02951129)
\lineto(162.00734439,187.02951129)
\closepath
}
}
{
\newrgbcolor{curcolor}{0 0 0}
\pscustom[linestyle=none,fillstyle=solid,fillcolor=curcolor]
{
\newpath
\moveto(167.00002017,182.3847359)
\curveto(167.03257226,182.01852496)(167.12412499,181.73776324)(167.27467838,181.54245074)
\curveto(167.55137108,181.18844683)(168.03151431,181.01144488)(168.71510806,181.01144488)
\curveto(169.1220091,181.01144488)(169.48008202,181.0989286)(169.78932681,181.27389605)
\curveto(170.0985716,181.45293251)(170.253194,181.72759071)(170.253194,182.09787066)
\curveto(170.253194,182.37863238)(170.12908918,182.59225543)(169.88087955,182.7387398)
\curveto(169.72218814,182.82825803)(169.40887434,182.9320178)(168.94093814,183.0500191)
\lineto(168.06813541,183.26974566)
\curveto(167.51068098,183.40809202)(167.09971093,183.56271441)(166.83522525,183.73361285)
\curveto(166.36322004,184.03065061)(166.12721744,184.44162066)(166.12721744,184.96652301)
\curveto(166.12721744,185.58501259)(166.34897851,186.08550087)(166.79250064,186.46798785)
\curveto(167.24009179,186.85047483)(167.84027082,187.04171832)(168.59303775,187.04171832)
\curveto(169.57773827,187.04171832)(170.28778059,186.75281858)(170.7231647,186.1750191)
\curveto(170.9957884,185.80880816)(171.12803124,185.41411415)(171.11989322,184.99093707)
\lineto(170.08229556,184.99093707)
\curveto(170.06195051,185.2391467)(169.97446679,185.46497678)(169.81984439,185.6684273)
\curveto(169.56756574,185.95732704)(169.13014713,186.10177691)(168.50758853,186.10177691)
\curveto(168.09254947,186.10177691)(167.77720116,186.02243121)(167.56154361,185.8637398)
\curveto(167.34995507,185.7050484)(167.2441608,185.49549436)(167.2441608,185.23507769)
\curveto(167.2441608,184.95024696)(167.38454166,184.72238238)(167.66530338,184.55148394)
\curveto(167.82806379,184.44975868)(168.06813541,184.36024045)(168.38551822,184.28292926)
\lineto(169.11183658,184.1059273)
\curveto(169.9012246,183.91468381)(170.43019595,183.72954384)(170.69875064,183.55050738)
\curveto(171.12599673,183.26974566)(171.33961978,182.82825803)(171.33961978,182.22604449)
\curveto(171.33961978,181.644176)(171.11785871,181.14165321)(170.67433658,180.71847613)
\curveto(170.23488345,180.29529905)(169.56349673,180.08371051)(168.66017642,180.08371051)
\curveto(167.68768293,180.08371051)(166.99798567,180.30343707)(166.59108463,180.74289019)
\curveto(166.18825259,181.18641233)(165.97259504,181.73369423)(165.94411197,182.3847359)
\lineto(167.00002017,182.3847359)
\closepath
\moveto(168.62355533,187.02951129)
\lineto(168.62355533,187.02951129)
\closepath
}
}
{
\newrgbcolor{curcolor}{0 0 0}
\pscustom[linestyle=none,fillstyle=solid,fillcolor=curcolor]
{
\newpath
\moveto(175.38014713,181.07248004)
\curveto(175.89284244,181.07248004)(176.31805403,181.28610308)(176.65578189,181.71334918)
\curveto(176.99757877,182.14466428)(177.1684772,182.78756793)(177.1684772,183.64206012)
\curveto(177.1684772,184.16289345)(177.09320051,184.61048459)(176.94264713,184.98483355)
\curveto(176.6578164,185.7050484)(176.13698306,186.06515582)(175.38014713,186.06515582)
\curveto(174.61924218,186.06515582)(174.09840884,185.68470334)(173.81764713,184.9237984)
\curveto(173.66709374,184.51689735)(173.59181705,184.00013303)(173.59181705,183.37350543)
\curveto(173.59181705,182.86894814)(173.66709374,182.43966754)(173.81764713,182.08566363)
\curveto(174.10247785,181.4102079)(174.62331119,181.07248004)(175.38014713,181.07248004)
\closepath
\moveto(172.53590884,186.8403023)
\lineto(173.60402408,186.8403023)
\lineto(173.60402408,185.97360308)
\curveto(173.82375064,186.27064084)(174.06382226,186.50053993)(174.32423892,186.66330035)
\curveto(174.69451887,186.90744097)(175.12990298,187.02951129)(175.63039127,187.02951129)
\curveto(176.37095116,187.02951129)(176.99961327,186.74468056)(177.51637759,186.1750191)
\curveto(178.03314192,185.60942665)(178.29152408,184.79969358)(178.29152408,183.74581988)
\curveto(178.29152408,182.32166624)(177.91920963,181.30441363)(177.17458072,180.69406207)
\curveto(176.70257551,180.30750608)(176.1532591,180.11422808)(175.5266315,180.11422808)
\curveto(175.03428124,180.11422808)(174.62127668,180.22205686)(174.28761783,180.43771441)
\curveto(174.09230533,180.55978472)(173.87461327,180.76933876)(173.63454166,181.06637652)
\lineto(173.63454166,177.72775347)
\lineto(172.53590884,177.72775347)
\lineto(172.53590884,186.8403023)
\closepath
}
}
{
\newrgbcolor{curcolor}{0 0 0}
\pscustom[linestyle=none,fillstyle=solid,fillcolor=curcolor]
{
\newpath
\moveto(182.17336002,181.04196246)
\curveto(182.90171288,181.04196246)(183.40016666,181.31662066)(183.66872134,181.86593707)
\curveto(183.94134504,182.41932249)(184.07765689,183.03374306)(184.07765689,183.70919879)
\curveto(184.07765689,184.31955035)(183.98000064,184.81596962)(183.78468814,185.1984566)
\curveto(183.47544335,185.80067014)(182.94240298,186.10177691)(182.18556705,186.10177691)
\curveto(181.51418033,186.10177691)(181.02589908,185.84542926)(180.7207233,185.33273394)
\curveto(180.41554752,184.82003863)(180.26295963,184.20154905)(180.26295963,183.47726519)
\curveto(180.26295963,182.78146441)(180.41554752,182.20163043)(180.7207233,181.73776324)
\curveto(181.02589908,181.27389605)(181.51011132,181.04196246)(182.17336002,181.04196246)
\closepath
\moveto(182.21608463,187.06002887)
\curveto(183.05836978,187.06002887)(183.7704466,186.77926715)(184.35231509,186.21774371)
\curveto(184.93418358,185.65622027)(185.22511783,184.83021116)(185.22511783,183.73971637)
\curveto(185.22511783,182.68584267)(184.96877017,181.81507444)(184.45607486,181.12741168)
\curveto(183.94337955,180.43974892)(183.14788801,180.09591754)(182.06960025,180.09591754)
\curveto(181.17034895,180.09591754)(180.45623762,180.39905881)(179.92726627,181.00534137)
\curveto(179.39829491,181.61569293)(179.13380923,182.43356402)(179.13380923,183.45895465)
\curveto(179.13380923,184.55758746)(179.41253645,185.4324247)(179.96999088,186.08346637)
\curveto(180.5274453,186.73450803)(181.27614322,187.06002887)(182.21608463,187.06002887)
\closepath
\moveto(182.17946353,187.02951129)
\lineto(182.17946353,187.02951129)
\closepath
}
}
{
\newrgbcolor{curcolor}{0 0 0}
\pscustom[linestyle=none,fillstyle=solid,fillcolor=curcolor]
{
\newpath
\moveto(186.53737369,186.87081988)
\lineto(187.58107486,186.87081988)
\lineto(187.58107486,185.94308551)
\curveto(187.89031965,186.32557249)(188.21787499,186.60023069)(188.56374088,186.76706012)
\curveto(188.90960676,186.93388954)(189.29412824,187.01730426)(189.71730533,187.01730426)
\curveto(190.6450397,187.01730426)(191.27166731,186.69381793)(191.59718814,186.04684527)
\curveto(191.7762246,185.69284137)(191.86574283,185.18624957)(191.86574283,184.52706988)
\lineto(191.86574283,180.33395465)
\lineto(190.74879947,180.33395465)
\lineto(190.74879947,184.45382769)
\curveto(190.74879947,184.85259071)(190.68979882,185.17404254)(190.57179752,185.41818316)
\curveto(190.37648502,185.8250842)(190.02248111,186.02853472)(189.5097858,186.02853472)
\curveto(189.24936913,186.02853472)(189.03574608,186.00208616)(188.86891666,185.94918902)
\curveto(188.56780989,185.85967079)(188.30332421,185.68063433)(188.07545963,185.41207965)
\curveto(187.89235416,185.19642209)(187.77231835,184.97262652)(187.7153522,184.74069293)
\curveto(187.66245507,184.51282834)(187.6360065,184.18527301)(187.6360065,183.75802691)
\lineto(187.6360065,180.33395465)
\lineto(186.53737369,180.33395465)
\lineto(186.53737369,186.87081988)
\closepath
\moveto(189.1191608,187.02951129)
\lineto(189.1191608,187.02951129)
\closepath
}
}
{
\newrgbcolor{curcolor}{0 0 0}
\pscustom[linestyle=none,fillstyle=solid,fillcolor=curcolor]
{
\newpath
\moveto(194.19118228,183.52609332)
\curveto(194.19118228,182.82622353)(194.33970116,182.24028603)(194.63673892,181.76828082)
\curveto(194.93377668,181.29627561)(195.4098509,181.06027301)(196.06496158,181.06027301)
\curveto(196.57358788,181.06027301)(196.99066145,181.27796506)(197.31618228,181.71334918)
\curveto(197.64577213,182.1528023)(197.81056705,182.78146441)(197.81056705,183.59933551)
\curveto(197.81056705,184.42534462)(197.64170311,185.03569618)(197.30397525,185.43039019)
\curveto(196.96624739,185.82915321)(196.54917382,186.02853472)(196.05275455,186.02853472)
\curveto(195.49936913,186.02853472)(195.04974348,185.81694618)(194.70387759,185.3937691)
\curveto(194.36208072,184.97059202)(194.19118228,184.34803342)(194.19118228,183.52609332)
\closepath
\moveto(195.84523502,186.98678668)
\curveto(196.3457233,186.98678668)(196.76483137,186.88099241)(197.10255923,186.66940387)
\curveto(197.29787173,186.54733355)(197.5196328,186.33371051)(197.76784244,186.02853472)
\lineto(197.76784244,189.33053668)
\lineto(198.82375064,189.33053668)
\lineto(198.82375064,180.33395465)
\lineto(197.83498111,180.33395465)
\lineto(197.83498111,181.24337847)
\curveto(197.57863345,180.84054644)(197.27549218,180.5496122)(196.92555728,180.37057574)
\curveto(196.57562239,180.19153928)(196.17482486,180.10202105)(195.7231647,180.10202105)
\curveto(194.99481184,180.10202105)(194.36411522,180.40719683)(193.83107486,181.0175484)
\curveto(193.29803449,181.63196897)(193.03151431,182.44780556)(193.03151431,183.46505816)
\curveto(193.03151431,184.4172066)(193.27362043,185.24118121)(193.75783267,185.93698199)
\curveto(194.24611392,186.63685178)(194.9419147,186.98678668)(195.84523502,186.98678668)
\closepath
}
}
{
\newrgbcolor{curcolor}{0 0 0}
\pscustom[linestyle=none,fillstyle=solid,fillcolor=curcolor]
{
\newpath
\moveto(203.17555728,187.01730426)
\curveto(203.63942447,187.01730426)(204.08905012,186.90744097)(204.52443423,186.68771441)
\curveto(204.95981835,186.47205686)(205.2914427,186.19129514)(205.51930728,185.84542926)
\curveto(205.73903384,185.51583941)(205.88551822,185.13131793)(205.95876041,184.6918648)
\curveto(206.02386457,184.39075803)(206.05641666,183.9106148)(206.05641666,183.25143512)
\lineto(201.26515689,183.25143512)
\curveto(201.28550194,182.58818642)(201.44215884,182.05514605)(201.73512759,181.65231402)
\curveto(202.02809634,181.253551)(202.48179101,181.05416949)(203.09621158,181.05416949)
\curveto(203.66994205,181.05416949)(204.12770572,181.24337847)(204.46950259,181.62179644)
\curveto(204.66481509,181.84152301)(204.80316145,182.09583616)(204.88454166,182.3847359)
\lineto(205.96486392,182.3847359)
\curveto(205.93638085,182.14466428)(205.8407591,181.87610959)(205.67799869,181.57907183)
\curveto(205.51930728,181.28610308)(205.34027082,181.04603147)(205.14088931,180.85885699)
\curveto(204.80723046,180.53333616)(204.3942259,180.31360959)(203.90187564,180.1996773)
\curveto(203.63738996,180.13457314)(203.3383177,180.10202105)(203.00465884,180.10202105)
\curveto(202.19085676,180.10202105)(201.50115949,180.39702431)(200.93556705,180.98703082)
\curveto(200.3699746,181.58110634)(200.08717838,182.41118446)(200.08717838,183.47726519)
\curveto(200.08717838,184.52706988)(200.3720091,185.37952756)(200.94167056,186.03463824)
\curveto(201.51133202,186.68974892)(202.25596093,187.01730426)(203.17555728,187.01730426)
\closepath
\moveto(204.92726627,184.12423785)
\curveto(204.88250715,184.60031207)(204.77874739,184.98076454)(204.61598697,185.26559527)
\curveto(204.3148802,185.79456663)(203.81235741,186.0590523)(203.10841861,186.0590523)
\curveto(202.60386132,186.0590523)(202.18068423,185.87594683)(201.83888736,185.5097359)
\curveto(201.49709048,185.14759397)(201.31601952,184.68576129)(201.29567447,184.12423785)
\lineto(204.92726627,184.12423785)
\closepath
\moveto(203.07179752,187.02951129)
\lineto(203.07179752,187.02951129)
\closepath
}
}
{
\newrgbcolor{curcolor}{0 0 0}
\pscustom[linestyle=none,fillstyle=solid,fillcolor=curcolor]
{
\newpath
\moveto(207.4419147,186.87081988)
\lineto(208.48561588,186.87081988)
\lineto(208.48561588,185.74166949)
\curveto(208.57106509,185.96139605)(208.78061913,186.22791624)(209.11427798,186.54123004)
\curveto(209.44793684,186.85861285)(209.83245832,187.01730426)(210.26784244,187.01730426)
\curveto(210.28818749,187.01730426)(210.32277408,187.01526975)(210.3716022,187.01120074)
\curveto(210.42043033,187.00713173)(210.50384504,186.99899371)(210.62184634,186.98678668)
\lineto(210.62184634,185.82711871)
\curveto(210.55674218,185.83932574)(210.49570702,185.84746376)(210.43874088,185.85153277)
\curveto(210.38584374,185.85560178)(210.32684309,185.85763629)(210.26173892,185.85763629)
\curveto(209.70835351,185.85763629)(209.28314192,185.67859983)(208.98610416,185.32052691)
\curveto(208.6890664,184.96652301)(208.54054752,184.55758746)(208.54054752,184.09372027)
\lineto(208.54054752,180.33395465)
\lineto(207.4419147,180.33395465)
\lineto(207.4419147,186.87081988)
\closepath
}
}
{
\newrgbcolor{curcolor}{0 0 0}
\pscustom[linewidth=1.06423664,linecolor=curcolor]
{
\newpath
\moveto(250.50117984,219.44555743)
\lineto(329.85596957,219.44555743)
\lineto(329.85596957,149.50979693)
\lineto(250.50117984,149.50979693)
\closepath
}
}
{
\newrgbcolor{curcolor}{0 0 0}
\pscustom[linewidth=1.06423664,linecolor=curcolor]
{
\newpath
\moveto(360.10439029,219.44555743)
\lineto(439.45918002,219.44555743)
\lineto(439.45918002,149.50979693)
\lineto(360.10439029,149.50979693)
\closepath
}
}
{
\newrgbcolor{curcolor}{0 0 0}
\pscustom[linewidth=1.06423664,linecolor=curcolor]
{
\newpath
\moveto(470.56389346,219.44555743)
\lineto(549.9186832,219.44555743)
\lineto(549.9186832,149.50979693)
\lineto(470.56389346,149.50979693)
\closepath
}
}
{
\newrgbcolor{curcolor}{0 0 0}
\pscustom[linewidth=1,linecolor=curcolor]
{
\newpath
\moveto(289.153289,280.33396778)
\lineto(289.153289,220.33396778)
}
}
{
\newrgbcolor{curcolor}{0 0 0}
\pscustom[linestyle=none,fillstyle=solid,fillcolor=curcolor]
{
\newpath
\moveto(289.153289,276.33396778)
\lineto(291.153289,274.33396778)
\lineto(289.153289,281.33396778)
\lineto(287.153289,274.33396778)
\lineto(289.153289,276.33396778)
\closepath
}
}
{
\newrgbcolor{curcolor}{0 0 0}
\pscustom[linewidth=0.5,linecolor=curcolor]
{
\newpath
\moveto(289.153289,276.33396778)
\lineto(291.153289,274.33396778)
\lineto(289.153289,281.33396778)
\lineto(287.153289,274.33396778)
\lineto(289.153289,276.33396778)
\closepath
}
}
{
\newrgbcolor{curcolor}{0 0 0}
\pscustom[linestyle=none,fillstyle=solid,fillcolor=curcolor]
{
\newpath
\moveto(289.153289,224.33396778)
\lineto(287.153289,226.33396778)
\lineto(289.153289,219.33396778)
\lineto(291.153289,226.33396778)
\lineto(289.153289,224.33396778)
\closepath
}
}
{
\newrgbcolor{curcolor}{0 0 0}
\pscustom[linewidth=0.5,linecolor=curcolor]
{
\newpath
\moveto(289.153289,224.33396778)
\lineto(287.153289,226.33396778)
\lineto(289.153289,219.33396778)
\lineto(291.153289,226.33396778)
\lineto(289.153289,224.33396778)
\closepath
}
}
{
\newrgbcolor{curcolor}{0 0 0}
\pscustom[linewidth=1,linecolor=curcolor]
{
\newpath
\moveto(510.928559,281.19024778)
\lineto(510.928559,221.19024778)
}
}
{
\newrgbcolor{curcolor}{0 0 0}
\pscustom[linestyle=none,fillstyle=solid,fillcolor=curcolor]
{
\newpath
\moveto(510.928559,277.19024778)
\lineto(512.928559,275.19024778)
\lineto(510.928559,282.19024778)
\lineto(508.928559,275.19024778)
\lineto(510.928559,277.19024778)
\closepath
}
}
{
\newrgbcolor{curcolor}{0 0 0}
\pscustom[linewidth=0.5,linecolor=curcolor]
{
\newpath
\moveto(510.928559,277.19024778)
\lineto(512.928559,275.19024778)
\lineto(510.928559,282.19024778)
\lineto(508.928559,275.19024778)
\lineto(510.928559,277.19024778)
\closepath
}
}
{
\newrgbcolor{curcolor}{0 0 0}
\pscustom[linestyle=none,fillstyle=solid,fillcolor=curcolor]
{
\newpath
\moveto(510.928559,225.19024778)
\lineto(508.928559,227.19024778)
\lineto(510.928559,220.19024778)
\lineto(512.928559,227.19024778)
\lineto(510.928559,225.19024778)
\closepath
}
}
{
\newrgbcolor{curcolor}{0 0 0}
\pscustom[linewidth=0.5,linecolor=curcolor]
{
\newpath
\moveto(510.928559,225.19024778)
\lineto(508.928559,227.19024778)
\lineto(510.928559,220.19024778)
\lineto(512.928559,227.19024778)
\lineto(510.928559,225.19024778)
\closepath
}
}
{
\newrgbcolor{curcolor}{0 0 0}
\pscustom[linewidth=1,linecolor=curcolor]
{
\newpath
\moveto(400.072289,281.47769778)
\lineto(400.072289,221.47769778)
}
}
{
\newrgbcolor{curcolor}{0 0 0}
\pscustom[linestyle=none,fillstyle=solid,fillcolor=curcolor]
{
\newpath
\moveto(400.072289,225.47769778)
\lineto(398.072289,227.47769778)
\lineto(400.072289,220.47769778)
\lineto(402.072289,227.47769778)
\lineto(400.072289,225.47769778)
\closepath
}
}
{
\newrgbcolor{curcolor}{0 0 0}
\pscustom[linewidth=0.5,linecolor=curcolor]
{
\newpath
\moveto(400.072289,225.47769778)
\lineto(398.072289,227.47769778)
\lineto(400.072289,220.47769778)
\lineto(402.072289,227.47769778)
\lineto(400.072289,225.47769778)
\closepath
}
}
{
\newrgbcolor{curcolor}{0 0 0}
\pscustom[linestyle=none,fillstyle=solid,fillcolor=curcolor]
{
\newpath
\moveto(264.28029361,189.54415972)
\curveto(265.41554752,189.54415972)(266.29648827,189.24508746)(266.92311588,188.64694293)
\curveto(267.54974348,188.0487984)(267.89764387,187.36927366)(267.96681705,186.60836871)
\lineto(266.78273502,186.60836871)
\curveto(266.64845767,187.18616819)(266.37990298,187.64393186)(265.97707095,187.98165972)
\curveto(265.57830793,188.31938759)(265.01678449,188.48825152)(264.29250064,188.48825152)
\curveto(263.40952538,188.48825152)(262.69541405,188.17697222)(262.15016666,187.55441363)
\curveto(261.60898827,186.93592405)(261.33839908,185.98581012)(261.33839908,184.70407183)
\curveto(261.33839908,183.65426715)(261.5825397,182.80180946)(262.07082095,182.14669879)
\curveto(262.56317121,181.49565712)(263.29559309,181.17013629)(264.26808658,181.17013629)
\curveto(265.16326887,181.17013629)(265.84482811,181.51396767)(266.31276431,182.20163043)
\curveto(266.56097395,182.56377235)(266.74611392,183.03984657)(266.86818423,183.62985308)
\lineto(268.05226627,183.62985308)
\curveto(267.94647199,182.68584267)(267.5965371,181.89442014)(267.00246158,181.25558551)
\curveto(266.29038476,180.48654254)(265.3300983,180.10202105)(264.1216022,180.10202105)
\curveto(263.07993554,180.10202105)(262.2050983,180.41736936)(261.49709048,181.04806597)
\curveto(260.5652871,181.88221311)(260.09938541,183.17005491)(260.09938541,184.91159137)
\curveto(260.09938541,186.23401975)(260.4493203,187.31841103)(261.14919009,188.16476519)
\curveto(261.90602603,189.08436155)(262.9497272,189.54415972)(264.28029361,189.54415972)
\closepath
\moveto(264.03615298,189.54415972)
\lineto(264.03615298,189.54415972)
\closepath
}
}
{
\newrgbcolor{curcolor}{0 0 0}
\pscustom[linestyle=none,fillstyle=solid,fillcolor=curcolor]
{
\newpath
\moveto(271.98293033,181.04196246)
\curveto(272.71128319,181.04196246)(273.20973697,181.31662066)(273.47829166,181.86593707)
\curveto(273.75091535,182.41932249)(273.8872272,183.03374306)(273.8872272,183.70919879)
\curveto(273.8872272,184.31955035)(273.78957095,184.81596962)(273.59425845,185.1984566)
\curveto(273.28501366,185.80067014)(272.7519733,186.10177691)(271.99513736,186.10177691)
\curveto(271.32375064,186.10177691)(270.83546939,185.84542926)(270.53029361,185.33273394)
\curveto(270.22511783,184.82003863)(270.07252994,184.20154905)(270.07252994,183.47726519)
\curveto(270.07252994,182.78146441)(270.22511783,182.20163043)(270.53029361,181.73776324)
\curveto(270.83546939,181.27389605)(271.31968163,181.04196246)(271.98293033,181.04196246)
\closepath
\moveto(272.02565494,187.06002887)
\curveto(272.86794009,187.06002887)(273.58001692,186.77926715)(274.16188541,186.21774371)
\curveto(274.7437539,185.65622027)(275.03468814,184.83021116)(275.03468814,183.73971637)
\curveto(275.03468814,182.68584267)(274.77834048,181.81507444)(274.26564517,181.12741168)
\curveto(273.75294986,180.43974892)(272.95745832,180.09591754)(271.87917056,180.09591754)
\curveto(270.97991926,180.09591754)(270.26580793,180.39905881)(269.73683658,181.00534137)
\curveto(269.20786522,181.61569293)(268.94337955,182.43356402)(268.94337955,183.45895465)
\curveto(268.94337955,184.55758746)(269.22210676,185.4324247)(269.77956119,186.08346637)
\curveto(270.33701561,186.73450803)(271.08571353,187.06002887)(272.02565494,187.06002887)
\closepath
\moveto(271.98903384,187.02951129)
\lineto(271.98903384,187.02951129)
\closepath
}
}
{
\newrgbcolor{curcolor}{0 0 0}
\pscustom[linestyle=none,fillstyle=solid,fillcolor=curcolor]
{
\newpath
\moveto(276.346944,186.87081988)
\lineto(277.39064517,186.87081988)
\lineto(277.39064517,185.94308551)
\curveto(277.69988996,186.32557249)(278.0274453,186.60023069)(278.37331119,186.76706012)
\curveto(278.71917707,186.93388954)(279.10369856,187.01730426)(279.52687564,187.01730426)
\curveto(280.45461002,187.01730426)(281.08123762,186.69381793)(281.40675845,186.04684527)
\curveto(281.58579491,185.69284137)(281.67531314,185.18624957)(281.67531314,184.52706988)
\lineto(281.67531314,180.33395465)
\lineto(280.55836978,180.33395465)
\lineto(280.55836978,184.45382769)
\curveto(280.55836978,184.85259071)(280.49936913,185.17404254)(280.38136783,185.41818316)
\curveto(280.18605533,185.8250842)(279.83205142,186.02853472)(279.31935611,186.02853472)
\curveto(279.05893944,186.02853472)(278.8453164,186.00208616)(278.67848697,185.94918902)
\curveto(278.3773802,185.85967079)(278.11289452,185.68063433)(277.88502994,185.41207965)
\curveto(277.70192447,185.19642209)(277.58188866,184.97262652)(277.52492252,184.74069293)
\curveto(277.47202538,184.51282834)(277.44557681,184.18527301)(277.44557681,183.75802691)
\lineto(277.44557681,180.33395465)
\lineto(276.346944,180.33395465)
\lineto(276.346944,186.87081988)
\closepath
\moveto(278.92873111,187.02951129)
\lineto(278.92873111,187.02951129)
\closepath
}
}
{
\newrgbcolor{curcolor}{0 0 0}
\pscustom[linestyle=none,fillstyle=solid,fillcolor=curcolor]
{
\newpath
\moveto(283.52467838,188.69577105)
\lineto(284.63551822,188.69577105)
\lineto(284.63551822,186.87081988)
\lineto(285.67921939,186.87081988)
\lineto(285.67921939,185.97360308)
\lineto(284.63551822,185.97360308)
\lineto(284.63551822,181.70724566)
\curveto(284.63551822,181.47938108)(284.71282942,181.32679319)(284.86745181,181.24948199)
\curveto(284.95290103,181.20472288)(285.0953164,181.18234332)(285.29469791,181.18234332)
\lineto(285.46559634,181.18234332)
\curveto(285.5266315,181.18641233)(285.59783918,181.19251584)(285.67921939,181.20065387)
\lineto(285.67921939,180.33395465)
\curveto(285.55308007,180.29733355)(285.42083723,180.27088499)(285.28249088,180.25460894)
\curveto(285.14821353,180.2383329)(285.00172916,180.23019488)(284.84303775,180.23019488)
\curveto(284.33034244,180.23019488)(283.98244205,180.36040321)(283.79933658,180.62081988)
\curveto(283.61623111,180.88530556)(283.52467838,181.22710243)(283.52467838,181.64621051)
\lineto(283.52467838,185.97360308)
\lineto(282.63966861,185.97360308)
\lineto(282.63966861,186.87081988)
\lineto(283.52467838,186.87081988)
\lineto(283.52467838,188.69577105)
\closepath
}
}
{
\newrgbcolor{curcolor}{0 0 0}
\pscustom[linestyle=none,fillstyle=solid,fillcolor=curcolor]
{
\newpath
\moveto(286.82668033,186.87081988)
\lineto(287.8703815,186.87081988)
\lineto(287.8703815,185.74166949)
\curveto(287.95583072,185.96139605)(288.16538476,186.22791624)(288.49904361,186.54123004)
\curveto(288.83270246,186.85861285)(289.21722395,187.01730426)(289.65260806,187.01730426)
\curveto(289.67295311,187.01730426)(289.7075397,187.01526975)(289.75636783,187.01120074)
\curveto(289.80519595,187.00713173)(289.88861067,186.99899371)(290.00661197,186.98678668)
\lineto(290.00661197,185.82711871)
\curveto(289.9415078,185.83932574)(289.88047265,185.84746376)(289.8235065,185.85153277)
\curveto(289.77060936,185.85560178)(289.71160871,185.85763629)(289.64650455,185.85763629)
\curveto(289.09311913,185.85763629)(288.66790754,185.67859983)(288.37086978,185.32052691)
\curveto(288.07383202,184.96652301)(287.92531314,184.55758746)(287.92531314,184.09372027)
\lineto(287.92531314,180.33395465)
\lineto(286.82668033,180.33395465)
\lineto(286.82668033,186.87081988)
\closepath
}
}
{
\newrgbcolor{curcolor}{0 0 0}
\pscustom[linestyle=none,fillstyle=solid,fillcolor=curcolor]
{
\newpath
\moveto(293.56496158,181.04196246)
\curveto(294.29331444,181.04196246)(294.79176822,181.31662066)(295.06032291,181.86593707)
\curveto(295.3329466,182.41932249)(295.46925845,183.03374306)(295.46925845,183.70919879)
\curveto(295.46925845,184.31955035)(295.3716022,184.81596962)(295.1762897,185.1984566)
\curveto(294.86704491,185.80067014)(294.33400455,186.10177691)(293.57716861,186.10177691)
\curveto(292.90578189,186.10177691)(292.41750064,185.84542926)(292.11232486,185.33273394)
\curveto(291.80714908,184.82003863)(291.65456119,184.20154905)(291.65456119,183.47726519)
\curveto(291.65456119,182.78146441)(291.80714908,182.20163043)(292.11232486,181.73776324)
\curveto(292.41750064,181.27389605)(292.90171288,181.04196246)(293.56496158,181.04196246)
\closepath
\moveto(293.60768619,187.06002887)
\curveto(294.44997134,187.06002887)(295.16204817,186.77926715)(295.74391666,186.21774371)
\curveto(296.32578515,185.65622027)(296.61671939,184.83021116)(296.61671939,183.73971637)
\curveto(296.61671939,182.68584267)(296.36037173,181.81507444)(295.84767642,181.12741168)
\curveto(295.33498111,180.43974892)(294.53948957,180.09591754)(293.46120181,180.09591754)
\curveto(292.56195051,180.09591754)(291.84783918,180.39905881)(291.31886783,181.00534137)
\curveto(290.78989647,181.61569293)(290.5254108,182.43356402)(290.5254108,183.45895465)
\curveto(290.5254108,184.55758746)(290.80413801,185.4324247)(291.36159244,186.08346637)
\curveto(291.91904686,186.73450803)(292.66774478,187.06002887)(293.60768619,187.06002887)
\closepath
\moveto(293.57106509,187.02951129)
\lineto(293.57106509,187.02951129)
\closepath
}
}
{
\newrgbcolor{curcolor}{0 0 0}
\pscustom[linestyle=none,fillstyle=solid,fillcolor=curcolor]
{
\newpath
\moveto(297.95949283,189.3000191)
\lineto(299.05812564,189.3000191)
\lineto(299.05812564,180.33395465)
\lineto(297.95949283,180.33395465)
\lineto(297.95949283,189.3000191)
\closepath
}
}
{
\newrgbcolor{curcolor}{0 0 0}
\pscustom[linestyle=none,fillstyle=solid,fillcolor=curcolor]
{
\newpath
\moveto(300.74269595,189.3000191)
\lineto(301.84132877,189.3000191)
\lineto(301.84132877,180.33395465)
\lineto(300.74269595,180.33395465)
\lineto(300.74269595,189.3000191)
\closepath
}
}
{
\newrgbcolor{curcolor}{0 0 0}
\pscustom[linestyle=none,fillstyle=solid,fillcolor=curcolor]
{
\newpath
\moveto(306.21754947,187.01730426)
\curveto(306.68141666,187.01730426)(307.13104231,186.90744097)(307.56642642,186.68771441)
\curveto(308.00181054,186.47205686)(308.33343489,186.19129514)(308.56129947,185.84542926)
\curveto(308.78102603,185.51583941)(308.92751041,185.13131793)(309.00075259,184.6918648)
\curveto(309.06585676,184.39075803)(309.09840884,183.9106148)(309.09840884,183.25143512)
\lineto(304.30714908,183.25143512)
\curveto(304.32749413,182.58818642)(304.48415103,182.05514605)(304.77711978,181.65231402)
\curveto(305.07008853,181.253551)(305.52378319,181.05416949)(306.13820377,181.05416949)
\curveto(306.71193423,181.05416949)(307.16969791,181.24337847)(307.51149478,181.62179644)
\curveto(307.70680728,181.84152301)(307.84515364,182.09583616)(307.92653384,182.3847359)
\lineto(309.00685611,182.3847359)
\curveto(308.97837304,182.14466428)(308.88275129,181.87610959)(308.71999088,181.57907183)
\curveto(308.56129947,181.28610308)(308.38226301,181.04603147)(308.1828815,180.85885699)
\curveto(307.84922265,180.53333616)(307.43621809,180.31360959)(306.94386783,180.1996773)
\curveto(306.67938215,180.13457314)(306.38030989,180.10202105)(306.04665103,180.10202105)
\curveto(305.23284895,180.10202105)(304.54315168,180.39702431)(303.97755923,180.98703082)
\curveto(303.41196679,181.58110634)(303.12917056,182.41118446)(303.12917056,183.47726519)
\curveto(303.12917056,184.52706988)(303.41400129,185.37952756)(303.98366275,186.03463824)
\curveto(304.55332421,186.68974892)(305.29795311,187.01730426)(306.21754947,187.01730426)
\closepath
\moveto(307.96925845,184.12423785)
\curveto(307.92449934,184.60031207)(307.82073957,184.98076454)(307.65797916,185.26559527)
\curveto(307.35687239,185.79456663)(306.8543496,186.0590523)(306.1504108,186.0590523)
\curveto(305.64585351,186.0590523)(305.22267642,185.87594683)(304.88087955,185.5097359)
\curveto(304.53908267,185.14759397)(304.35801171,184.68576129)(304.33766666,184.12423785)
\lineto(307.96925845,184.12423785)
\closepath
\moveto(306.1137897,187.02951129)
\lineto(306.1137897,187.02951129)
\closepath
}
}
{
\newrgbcolor{curcolor}{0 0 0}
\pscustom[linestyle=none,fillstyle=solid,fillcolor=curcolor]
{
\newpath
\moveto(310.48390689,186.87081988)
\lineto(311.52760806,186.87081988)
\lineto(311.52760806,185.74166949)
\curveto(311.61305728,185.96139605)(311.82261132,186.22791624)(312.15627017,186.54123004)
\curveto(312.48992903,186.85861285)(312.87445051,187.01730426)(313.30983463,187.01730426)
\curveto(313.33017968,187.01730426)(313.36476627,187.01526975)(313.41359439,187.01120074)
\curveto(313.46242252,187.00713173)(313.54583723,186.99899371)(313.66383853,186.98678668)
\lineto(313.66383853,185.82711871)
\curveto(313.59873436,185.83932574)(313.53769921,185.84746376)(313.48073306,185.85153277)
\curveto(313.42783593,185.85560178)(313.36883528,185.85763629)(313.30373111,185.85763629)
\curveto(312.75034569,185.85763629)(312.3251341,185.67859983)(312.02809634,185.32052691)
\curveto(311.73105858,184.96652301)(311.5825397,184.55758746)(311.5825397,184.09372027)
\lineto(311.5825397,180.33395465)
\lineto(310.48390689,180.33395465)
\lineto(310.48390689,186.87081988)
\closepath
}
}
{
\newrgbcolor{curcolor}{0 0 0}
\pscustom[linestyle=none,fillstyle=solid,fillcolor=curcolor]
{
\newpath
\moveto(371.95052066,194.43765338)
\lineto(373.64119449,187.14395221)
\lineto(375.66756168,194.43765338)
\lineto(376.98592105,194.43765338)
\lineto(379.01228824,187.14395221)
\lineto(380.70296207,194.43765338)
\lineto(382.03352847,194.43765338)
\lineto(379.67757144,185.47158892)
\lineto(378.40193668,185.47158892)
\lineto(376.33284488,192.90567096)
\lineto(374.25154605,185.47158892)
\lineto(372.97591129,185.47158892)
\lineto(370.63216129,194.43765338)
\lineto(371.95052066,194.43765338)
\closepath
}
}
{
\newrgbcolor{curcolor}{0 0 0}
\pscustom[linestyle=none,fillstyle=solid,fillcolor=curcolor]
{
\newpath
\moveto(385.72615543,192.15493853)
\curveto(386.19002261,192.15493853)(386.63964826,192.04507525)(387.07503238,191.82534869)
\curveto(387.51041649,191.60969114)(387.84204084,191.32892942)(388.06990543,190.98306353)
\curveto(388.28963199,190.65347369)(388.43611636,190.26895221)(388.50935855,189.82949908)
\curveto(388.57446272,189.52839231)(388.6070148,189.04824908)(388.6070148,188.38906939)
\lineto(383.81575504,188.38906939)
\curveto(383.83610009,187.72582069)(383.99275699,187.19278033)(384.28572574,186.7899483)
\curveto(384.57869449,186.39118528)(385.03238915,186.19180377)(385.64680972,186.19180377)
\curveto(386.22054019,186.19180377)(386.67830386,186.38101275)(387.02010074,186.75943072)
\curveto(387.21541324,186.97915728)(387.35375959,187.23347043)(387.4351398,187.52237017)
\lineto(388.51546207,187.52237017)
\curveto(388.48697899,187.28229856)(388.39135725,187.01374387)(388.22859683,186.71670611)
\curveto(388.06990543,186.42373736)(387.89086897,186.18366575)(387.69148746,185.99649127)
\curveto(387.3578286,185.67097043)(386.94482405,185.45124387)(386.45247379,185.33731158)
\curveto(386.18798811,185.27220741)(385.88891584,185.23965533)(385.55525699,185.23965533)
\curveto(384.7414549,185.23965533)(384.05175764,185.53465859)(383.48616519,186.1246651)
\curveto(382.92057274,186.71874062)(382.63777652,187.54881874)(382.63777652,188.61489947)
\curveto(382.63777652,189.66470416)(382.92260725,190.51716184)(383.49226871,191.17227252)
\curveto(384.06193017,191.82738319)(384.80655907,192.15493853)(385.72615543,192.15493853)
\closepath
\moveto(387.47786441,189.26187213)
\curveto(387.4331053,189.73794635)(387.32934553,190.11839882)(387.16658511,190.40322955)
\curveto(386.86547834,190.9322009)(386.36295556,191.19668658)(385.65901675,191.19668658)
\curveto(385.15445946,191.19668658)(384.73128238,191.01358111)(384.3894855,190.64737017)
\curveto(384.04768863,190.28522825)(383.86661767,189.82339556)(383.84627261,189.26187213)
\lineto(387.47786441,189.26187213)
\closepath
\moveto(385.62239566,192.16714556)
\lineto(385.62239566,192.16714556)
\closepath
}
}
{
\newrgbcolor{curcolor}{0 0 0}
\pscustom[linestyle=none,fillstyle=solid,fillcolor=curcolor]
{
\newpath
\moveto(389.87654605,194.46817096)
\lineto(390.94466129,194.46817096)
\lineto(390.94466129,191.21499713)
\curveto(391.1847329,191.52831093)(391.47159813,191.76634804)(391.80525699,191.92910846)
\curveto(392.13891584,192.09593788)(392.50105777,192.1793526)(392.89168277,192.1793526)
\curveto(393.70548485,192.1793526)(394.36466454,191.89859088)(394.86922183,191.33706744)
\curveto(395.37784813,190.77961301)(395.63216129,189.9556384)(395.63216129,188.86514361)
\curveto(395.63216129,187.83161497)(395.38191714,186.97305377)(394.88142886,186.28946002)
\curveto(394.38094058,185.60586627)(393.68717431,185.26406939)(392.80013004,185.26406939)
\curveto(392.30371076,185.26406939)(391.88460269,185.3841052)(391.54280582,185.62417681)
\curveto(391.3393553,185.76659218)(391.12166324,185.99445676)(390.88972964,186.30777056)
\lineto(390.88972964,185.47158892)
\lineto(389.87654605,185.47158892)
\lineto(389.87654605,194.46817096)
\closepath
\moveto(392.73299136,186.23452838)
\curveto(393.32706688,186.23452838)(393.77058902,186.47053098)(394.06355777,186.94253619)
\curveto(394.36059553,187.4145414)(394.50911441,188.03709999)(394.50911441,188.81021197)
\curveto(394.50911441,189.49787473)(394.36059553,190.06753619)(394.06355777,190.51919635)
\curveto(393.77058902,190.9708565)(393.33723941,191.19668658)(392.76350894,191.19668658)
\curveto(392.26302066,191.19668658)(391.82356754,191.01154661)(391.44514957,190.64126666)
\curveto(391.07080061,190.27098671)(390.88362613,189.66063515)(390.88362613,188.81021197)
\curveto(390.88362613,188.1957914)(390.96093733,187.69733762)(391.11555972,187.31485064)
\curveto(391.40445946,186.5946358)(391.94360334,186.23452838)(392.73299136,186.23452838)
\closepath
}
}
{
\newrgbcolor{curcolor}{0 0 0}
\pscustom[linestyle=none,fillstyle=solid,fillcolor=curcolor]
{
\newpath
\moveto(375.23421207,182.16958697)
\curveto(376.07649722,182.16958697)(376.80485009,182.00682655)(377.41927066,181.68130572)
\curveto(378.31038394,181.21336952)(378.85563134,180.39346392)(379.05501285,179.22158892)
\lineto(377.85262027,179.22158892)
\curveto(377.70613589,179.8766996)(377.40299462,180.35277382)(376.94319644,180.64981158)
\curveto(376.48339826,180.95091835)(375.90356428,181.10147174)(375.20369449,181.10147174)
\curveto(374.37361636,181.10147174)(373.67374657,180.79019244)(373.10408511,180.16763385)
\curveto(372.53849267,179.54507525)(372.25569644,178.61734088)(372.25569644,177.38443072)
\curveto(372.25569644,176.31834999)(372.48966454,175.44961627)(372.95760074,174.77822955)
\curveto(373.42553694,174.11091184)(374.18847639,173.77725299)(375.2464191,173.77725299)
\curveto(376.05615217,173.77725299)(376.72550438,174.01122109)(377.25447574,174.47915728)
\curveto(377.7875161,174.95116249)(378.0601398,175.71206744)(378.07234683,176.76187213)
\lineto(375.26472964,176.76187213)
\lineto(375.26472964,177.76895221)
\lineto(379.20149722,177.76895221)
\lineto(379.20149722,172.97158892)
\lineto(378.42024722,172.97158892)
\lineto(378.12727847,174.12515338)
\curveto(377.71630842,173.67349322)(377.35213199,173.36017942)(377.03474918,173.18521197)
\curveto(376.50170881,172.8841052)(375.82421858,172.73355181)(375.00227847,172.73355181)
\curveto(373.94026675,172.73355181)(373.02677392,173.07738319)(372.26179996,173.76504596)
\curveto(371.42765282,174.62767616)(371.01057925,175.81175819)(371.01057925,177.31729205)
\curveto(371.01057925,178.81875689)(371.4174803,180.01301145)(372.23128238,180.90005572)
\curveto(373.00439436,181.74640989)(374.00537092,182.16958697)(375.23421207,182.16958697)
\closepath
\moveto(375.03279605,182.181794)
\lineto(375.03279605,182.181794)
\closepath
}
}
{
\newrgbcolor{curcolor}{0 0 0}
\pscustom[linestyle=none,fillstyle=solid,fillcolor=curcolor]
{
\newpath
\moveto(381.79549136,174.71109088)
\curveto(381.79549136,174.39370806)(381.91145816,174.14346392)(382.14339175,173.96035846)
\curveto(382.37532535,173.77725299)(382.64998355,173.68570025)(382.96736636,173.68570025)
\curveto(383.35392235,173.68570025)(383.72827131,173.77521848)(384.09041324,173.95425494)
\curveto(384.7007648,174.2512927)(385.00594058,174.73753944)(385.00594058,175.41299517)
\lineto(385.00594058,176.29800494)
\curveto(384.87166324,176.21255572)(384.6987303,176.14134804)(384.48714175,176.08438189)
\curveto(384.27555321,176.02741575)(384.06803368,175.98672564)(383.86458316,175.96231158)
\lineto(383.19929996,175.87686236)
\curveto(382.80053694,175.82396523)(382.50146467,175.74055051)(382.30208316,175.62661822)
\curveto(381.9643553,175.43537473)(381.79549136,175.13019895)(381.79549136,174.71109088)
\closepath
\moveto(384.45662418,176.93277056)
\curveto(384.70890282,176.96532265)(384.87776675,177.07111692)(384.96321597,177.25015338)
\curveto(385.0120441,177.34780963)(385.03645816,177.48819049)(385.03645816,177.67129596)
\curveto(385.03645816,178.04564491)(384.90218082,178.31623411)(384.63362613,178.48306353)
\curveto(384.36914045,178.65396197)(383.98868798,178.73941119)(383.49226871,178.73941119)
\curveto(382.91853824,178.73941119)(382.5116372,178.58478879)(382.27156558,178.275544)
\curveto(382.13728824,178.10464556)(382.04980451,177.85033241)(382.00911441,177.51260455)
\lineto(380.98372379,177.51260455)
\curveto(381.00406884,178.31826861)(381.2644855,178.87775754)(381.76497379,179.19107135)
\curveto(382.26953108,179.50845416)(382.85343407,179.66714556)(383.51668277,179.66714556)
\curveto(384.28572574,179.66714556)(384.91031884,179.52066119)(385.39046207,179.22769244)
\curveto(385.86653629,178.93472369)(386.10457339,178.47899452)(386.10457339,177.86050494)
\lineto(386.10457339,174.0946358)
\curveto(386.10457339,173.98070351)(386.12695295,173.88915077)(386.17171207,173.8199776)
\curveto(386.22054019,173.75080442)(386.32023095,173.71621783)(386.47078433,173.71621783)
\curveto(386.51961246,173.71621783)(386.5745441,173.71825234)(386.63557925,173.72232135)
\curveto(386.69661441,173.73045937)(386.76171858,173.74063189)(386.83089175,173.75283892)
\lineto(386.83089175,172.94107135)
\curveto(386.65999332,172.89224322)(386.52978498,172.86172564)(386.44026675,172.84951861)
\curveto(386.35074852,172.83731158)(386.22867821,172.83120806)(386.07405582,172.83120806)
\curveto(385.69563785,172.83120806)(385.42097964,172.96548541)(385.25008121,173.2340401)
\curveto(385.16056298,173.37645546)(385.09749332,173.57787148)(385.06087222,173.83828814)
\curveto(384.83707665,173.54531939)(384.51562483,173.29100624)(384.09651675,173.07534869)
\curveto(383.67740868,172.85969114)(383.215576,172.75186236)(382.71101871,172.75186236)
\curveto(382.10473615,172.75186236)(381.60831688,172.93496783)(381.22176089,173.30117877)
\curveto(380.83927392,173.67145872)(380.64803043,174.1332914)(380.64803043,174.68667681)
\curveto(380.64803043,175.29295937)(380.83723941,175.76293007)(381.21565738,176.09658892)
\curveto(381.59407535,176.43024778)(382.09049462,176.6357328)(382.70491519,176.713044)
\lineto(384.45662418,176.93277056)
\closepath
\moveto(383.54720035,179.66714556)
\lineto(383.54720035,179.66714556)
\closepath
}
}
{
\newrgbcolor{curcolor}{0 0 0}
\pscustom[linestyle=none,fillstyle=solid,fillcolor=curcolor]
{
\newpath
\moveto(388.13094058,181.33340533)
\lineto(389.24178043,181.33340533)
\lineto(389.24178043,179.50845416)
\lineto(390.2854816,179.50845416)
\lineto(390.2854816,178.61123736)
\lineto(389.24178043,178.61123736)
\lineto(389.24178043,174.34487994)
\curveto(389.24178043,174.11701536)(389.31909162,173.96442747)(389.47371402,173.88711627)
\curveto(389.55916324,173.84235715)(389.7015786,173.8199776)(389.90096011,173.8199776)
\lineto(390.07185855,173.8199776)
\curveto(390.13289371,173.82404661)(390.20410139,173.83015012)(390.2854816,173.83828814)
\lineto(390.2854816,172.97158892)
\curveto(390.15934227,172.93496783)(390.02709944,172.90851926)(389.88875308,172.89224322)
\curveto(389.75447574,172.87596718)(389.60799136,172.86782916)(389.44929996,172.86782916)
\curveto(388.93660464,172.86782916)(388.58870425,172.99803749)(388.40559879,173.25845416)
\curveto(388.22249332,173.52293984)(388.13094058,173.86473671)(388.13094058,174.28384478)
\lineto(388.13094058,178.61123736)
\lineto(387.24593082,178.61123736)
\lineto(387.24593082,179.50845416)
\lineto(388.13094058,179.50845416)
\lineto(388.13094058,181.33340533)
\closepath
}
}
{
\newrgbcolor{curcolor}{0 0 0}
\pscustom[linestyle=none,fillstyle=solid,fillcolor=curcolor]
{
\newpath
\moveto(394.12459293,179.65493853)
\curveto(394.58846011,179.65493853)(395.03808576,179.54507525)(395.47346988,179.32534869)
\curveto(395.90885399,179.10969114)(396.24047834,178.82892942)(396.46834293,178.48306353)
\curveto(396.68806949,178.15347369)(396.83455386,177.76895221)(396.90779605,177.32949908)
\curveto(396.97290022,177.02839231)(397.0054523,176.54824908)(397.0054523,175.88906939)
\lineto(392.21419254,175.88906939)
\curveto(392.23453759,175.22582069)(392.39119449,174.69278033)(392.68416324,174.2899483)
\curveto(392.97713199,173.89118528)(393.43082665,173.69180377)(394.04524722,173.69180377)
\curveto(394.61897769,173.69180377)(395.07674136,173.88101275)(395.41853824,174.25943072)
\curveto(395.61385074,174.47915728)(395.75219709,174.73347043)(395.8335773,175.02237017)
\lineto(396.91389957,175.02237017)
\curveto(396.88541649,174.78229856)(396.78979475,174.51374387)(396.62703433,174.21670611)
\curveto(396.46834293,173.92373736)(396.28930647,173.68366575)(396.08992496,173.49649127)
\curveto(395.7562661,173.17097043)(395.34326155,172.95124387)(394.85091129,172.83731158)
\curveto(394.58642561,172.77220741)(394.28735334,172.73965533)(393.95369449,172.73965533)
\curveto(393.1398924,172.73965533)(392.45019514,173.03465859)(391.88460269,173.6246651)
\curveto(391.31901024,174.21874062)(391.03621402,175.04881874)(391.03621402,176.11489947)
\curveto(391.03621402,177.16470416)(391.32104475,178.01716184)(391.89070621,178.67227252)
\curveto(392.46036767,179.32738319)(393.20499657,179.65493853)(394.12459293,179.65493853)
\closepath
\moveto(395.87630191,176.76187213)
\curveto(395.8315428,177.23794635)(395.72778303,177.61839882)(395.56502261,177.90322955)
\curveto(395.26391584,178.4322009)(394.76139306,178.69668658)(394.05745425,178.69668658)
\curveto(393.55289696,178.69668658)(393.12971988,178.51358111)(392.787923,178.14737017)
\curveto(392.44612613,177.78522825)(392.26505517,177.32339556)(392.24471011,176.76187213)
\lineto(395.87630191,176.76187213)
\closepath
\moveto(394.02083316,179.66714556)
\lineto(394.02083316,179.66714556)
\closepath
}
}
{
\newrgbcolor{curcolor}{0 0 0}
\pscustom[linestyle=none,fillstyle=solid,fillcolor=curcolor]
{
\newpath
\moveto(398.86702457,179.50845416)
\lineto(400.12434879,174.35708697)
\lineto(401.39998355,179.50845416)
\lineto(402.63289371,179.50845416)
\lineto(403.91463199,174.38760455)
\lineto(405.25130191,179.50845416)
\lineto(406.34993472,179.50845416)
\lineto(404.45174136,172.97158892)
\lineto(403.31038394,172.97158892)
\lineto(401.97981754,178.03140338)
\lineto(400.69197574,172.97158892)
\lineto(399.55061832,172.97158892)
\lineto(397.66463199,179.50845416)
\lineto(398.86702457,179.50845416)
\closepath
}
}
{
\newrgbcolor{curcolor}{0 0 0}
\pscustom[linestyle=none,fillstyle=solid,fillcolor=curcolor]
{
\newpath
\moveto(408.23592105,174.71109088)
\curveto(408.23592105,174.39370806)(408.35188785,174.14346392)(408.58382144,173.96035846)
\curveto(408.81575504,173.77725299)(409.09041324,173.68570025)(409.40779605,173.68570025)
\curveto(409.79435204,173.68570025)(410.168701,173.77521848)(410.53084293,173.95425494)
\curveto(411.14119449,174.2512927)(411.44637027,174.73753944)(411.44637027,175.41299517)
\lineto(411.44637027,176.29800494)
\curveto(411.31209293,176.21255572)(411.13915998,176.14134804)(410.92757144,176.08438189)
\curveto(410.7159829,176.02741575)(410.50846337,175.98672564)(410.30501285,175.96231158)
\lineto(409.63972964,175.87686236)
\curveto(409.24096662,175.82396523)(408.94189436,175.74055051)(408.74251285,175.62661822)
\curveto(408.40478498,175.43537473)(408.23592105,175.13019895)(408.23592105,174.71109088)
\closepath
\moveto(410.89705386,176.93277056)
\curveto(411.14933251,176.96532265)(411.31819644,177.07111692)(411.40364566,177.25015338)
\curveto(411.45247379,177.34780963)(411.47688785,177.48819049)(411.47688785,177.67129596)
\curveto(411.47688785,178.04564491)(411.3426105,178.31623411)(411.07405582,178.48306353)
\curveto(410.80957014,178.65396197)(410.42911767,178.73941119)(409.93269839,178.73941119)
\curveto(409.35896793,178.73941119)(408.95206688,178.58478879)(408.71199527,178.275544)
\curveto(408.57771793,178.10464556)(408.4902342,177.85033241)(408.4495441,177.51260455)
\lineto(407.42415347,177.51260455)
\curveto(407.44449852,178.31826861)(407.70491519,178.87775754)(408.20540347,179.19107135)
\curveto(408.70996076,179.50845416)(409.29386376,179.66714556)(409.95711246,179.66714556)
\curveto(410.72615543,179.66714556)(411.35074852,179.52066119)(411.83089175,179.22769244)
\curveto(412.30696597,178.93472369)(412.54500308,178.47899452)(412.54500308,177.86050494)
\lineto(412.54500308,174.0946358)
\curveto(412.54500308,173.98070351)(412.56738264,173.88915077)(412.61214175,173.8199776)
\curveto(412.66096988,173.75080442)(412.76066063,173.71621783)(412.91121402,173.71621783)
\curveto(412.96004214,173.71621783)(413.01497379,173.71825234)(413.07600894,173.72232135)
\curveto(413.1370441,173.73045937)(413.20214826,173.74063189)(413.27132144,173.75283892)
\lineto(413.27132144,172.94107135)
\curveto(413.100423,172.89224322)(412.97021467,172.86172564)(412.88069644,172.84951861)
\curveto(412.79117821,172.83731158)(412.6691079,172.83120806)(412.5144855,172.83120806)
\curveto(412.13606754,172.83120806)(411.86140933,172.96548541)(411.69051089,173.2340401)
\curveto(411.60099267,173.37645546)(411.537923,173.57787148)(411.50130191,173.83828814)
\curveto(411.27750634,173.54531939)(410.95605451,173.29100624)(410.53694644,173.07534869)
\curveto(410.11783837,172.85969114)(409.65600569,172.75186236)(409.15144839,172.75186236)
\curveto(408.54516584,172.75186236)(408.04874657,172.93496783)(407.66219058,173.30117877)
\curveto(407.2797036,173.67145872)(407.08846011,174.1332914)(407.08846011,174.68667681)
\curveto(407.08846011,175.29295937)(407.2776691,175.76293007)(407.65608707,176.09658892)
\curveto(408.03450504,176.43024778)(408.53092431,176.6357328)(409.14534488,176.713044)
\lineto(410.89705386,176.93277056)
\closepath
\moveto(409.98763004,179.66714556)
\lineto(409.98763004,179.66714556)
\closepath
}
}
{
\newrgbcolor{curcolor}{0 0 0}
\pscustom[linestyle=none,fillstyle=solid,fillcolor=curcolor]
{
\newpath
\moveto(418.43489566,179.50845416)
\lineto(419.64949527,179.50845416)
\curveto(419.49487287,179.08934609)(419.15104149,178.13312864)(418.61800113,176.63980181)
\curveto(418.21923811,175.51675494)(417.88557925,174.6012276)(417.61702457,173.89321978)
\curveto(416.98225894,172.22492551)(416.5346678,171.20767291)(416.27425113,170.84146197)
\curveto(416.01383446,170.47525103)(415.56624332,170.29214556)(414.93147769,170.29214556)
\curveto(414.7768553,170.29214556)(414.65681949,170.29824908)(414.57137027,170.31045611)
\curveto(414.48999006,170.32266314)(414.3882648,170.3450427)(414.26619449,170.37759478)
\lineto(414.26619449,171.37857135)
\curveto(414.45743798,171.32567421)(414.59578433,171.29312213)(414.68123355,171.2809151)
\curveto(414.76668277,171.26870806)(414.84195946,171.26260455)(414.90706363,171.26260455)
\curveto(415.11051415,171.26260455)(415.25903303,171.29719114)(415.35262027,171.36636431)
\curveto(415.45027652,171.43146848)(415.53165673,171.51284869)(415.59676089,171.61050494)
\curveto(415.61710595,171.64305702)(415.69034813,171.80988645)(415.81648746,172.11099322)
\curveto(415.94262678,172.41209999)(416.03417951,172.63589556)(416.09114566,172.78237994)
\lineto(413.67415347,179.50845416)
\lineto(414.91927066,179.50845416)
\lineto(416.67097964,174.18618853)
\lineto(418.43489566,179.50845416)
\closepath
\moveto(416.66487613,179.66714556)
\lineto(416.66487613,179.66714556)
\closepath
}
}
{
\newrgbcolor{curcolor}{0 0 0}
\pscustom[linestyle=none,fillstyle=solid,fillcolor=curcolor]
{
\newpath
\moveto(485.19313541,185.19235308)
\curveto(485.76279686,185.19235308)(486.21242252,185.30628538)(486.54201236,185.53414996)
\curveto(486.87567121,185.76201454)(487.04250064,186.17298459)(487.04250064,186.76706012)
\curveto(487.04250064,187.40589475)(486.81056705,187.84127887)(486.34669986,188.07321246)
\curveto(486.09849022,188.19528277)(485.76686588,188.25631793)(485.35182681,188.25631793)
\lineto(482.38551822,188.25631793)
\lineto(482.38551822,185.19235308)
\lineto(485.19313541,185.19235308)
\closepath
\moveto(481.17091861,189.3000191)
\lineto(485.32130923,189.3000191)
\curveto(486.00490298,189.3000191)(486.56846093,189.20032834)(487.01198306,189.00094683)
\curveto(487.85426822,188.61845985)(488.2754108,187.91248655)(488.2754108,186.88302691)
\curveto(488.2754108,186.34591754)(488.16351301,185.90646441)(487.93971744,185.56466754)
\curveto(487.71999088,185.22287066)(487.41074608,184.94821246)(487.01198306,184.74069293)
\curveto(487.36191796,184.59827756)(487.62436913,184.41110308)(487.79933658,184.17916949)
\curveto(487.97837304,183.9472359)(488.07806379,183.57085243)(488.09840884,183.0500191)
\lineto(488.14113345,181.84762652)
\curveto(488.15334048,181.50582965)(488.18182356,181.2515165)(488.22658267,181.08468707)
\curveto(488.29982486,180.79985634)(488.43003319,180.61675087)(488.61720767,180.53537066)
\lineto(488.61720767,180.33395465)
\lineto(487.12794986,180.33395465)
\curveto(487.08725976,180.41126584)(487.05470767,180.5109566)(487.03029361,180.63302691)
\curveto(487.00587955,180.75509722)(486.98553449,180.99109983)(486.96925845,181.34103472)
\lineto(486.89601627,182.83639605)
\curveto(486.86753319,183.42233355)(486.64984114,183.81499306)(486.24294009,184.01437457)
\curveto(486.0110065,184.12423785)(485.64683007,184.17916949)(485.1504108,184.17916949)
\lineto(482.38551822,184.17916949)
\lineto(482.38551822,180.33395465)
\lineto(481.17091861,180.33395465)
\lineto(481.17091861,189.3000191)
\closepath
}
}
{
\newrgbcolor{curcolor}{0 0 0}
\pscustom[linestyle=none,fillstyle=solid,fillcolor=curcolor]
{
\newpath
\moveto(492.63332095,187.01730426)
\curveto(493.09718814,187.01730426)(493.54681379,186.90744097)(493.98219791,186.68771441)
\curveto(494.41758202,186.47205686)(494.74920637,186.19129514)(494.97707095,185.84542926)
\curveto(495.19679752,185.51583941)(495.34328189,185.13131793)(495.41652408,184.6918648)
\curveto(495.48162824,184.39075803)(495.51418033,183.9106148)(495.51418033,183.25143512)
\lineto(490.72292056,183.25143512)
\curveto(490.74326561,182.58818642)(490.89992252,182.05514605)(491.19289127,181.65231402)
\curveto(491.48586002,181.253551)(491.93955468,181.05416949)(492.55397525,181.05416949)
\curveto(493.12770572,181.05416949)(493.58546939,181.24337847)(493.92726627,181.62179644)
\curveto(494.12257877,181.84152301)(494.26092512,182.09583616)(494.34230533,182.3847359)
\lineto(495.42262759,182.3847359)
\curveto(495.39414452,182.14466428)(495.29852278,181.87610959)(495.13576236,181.57907183)
\curveto(494.97707095,181.28610308)(494.79803449,181.04603147)(494.59865298,180.85885699)
\curveto(494.26499413,180.53333616)(493.85198957,180.31360959)(493.35963931,180.1996773)
\curveto(493.09515364,180.13457314)(492.79608137,180.10202105)(492.46242252,180.10202105)
\curveto(491.64862043,180.10202105)(490.95892317,180.39702431)(490.39333072,180.98703082)
\curveto(489.82773827,181.58110634)(489.54494205,182.41118446)(489.54494205,183.47726519)
\curveto(489.54494205,184.52706988)(489.82977278,185.37952756)(490.39943423,186.03463824)
\curveto(490.96909569,186.68974892)(491.7137246,187.01730426)(492.63332095,187.01730426)
\closepath
\moveto(494.38502994,184.12423785)
\curveto(494.34027082,184.60031207)(494.23651106,184.98076454)(494.07375064,185.26559527)
\curveto(493.77264387,185.79456663)(493.27012108,186.0590523)(492.56618228,186.0590523)
\curveto(492.06162499,186.0590523)(491.63844791,185.87594683)(491.29665103,185.5097359)
\curveto(490.95485416,185.14759397)(490.77378319,184.68576129)(490.75343814,184.12423785)
\lineto(494.38502994,184.12423785)
\closepath
\moveto(492.52956119,187.02951129)
\lineto(492.52956119,187.02951129)
\closepath
}
}
{
\newrgbcolor{curcolor}{0 0 0}
\pscustom[linestyle=none,fillstyle=solid,fillcolor=curcolor]
{
\newpath
\moveto(497.52223697,182.3847359)
\curveto(497.55478905,182.01852496)(497.64634179,181.73776324)(497.79689517,181.54245074)
\curveto(498.07358788,181.18844683)(498.55373111,181.01144488)(499.23732486,181.01144488)
\curveto(499.6442259,181.01144488)(500.00229882,181.0989286)(500.31154361,181.27389605)
\curveto(500.6207884,181.45293251)(500.7754108,181.72759071)(500.7754108,182.09787066)
\curveto(500.7754108,182.37863238)(500.65130598,182.59225543)(500.40309634,182.7387398)
\curveto(500.24440494,182.82825803)(499.93109114,182.9320178)(499.46315494,183.0500191)
\lineto(498.5903522,183.26974566)
\curveto(498.03289778,183.40809202)(497.62192772,183.56271441)(497.35744205,183.73361285)
\curveto(496.88543684,184.03065061)(496.64943423,184.44162066)(496.64943423,184.96652301)
\curveto(496.64943423,185.58501259)(496.8711953,186.08550087)(497.31471744,186.46798785)
\curveto(497.76230858,186.85047483)(498.36248762,187.04171832)(499.11525455,187.04171832)
\curveto(500.09995507,187.04171832)(500.80999739,186.75281858)(501.2453815,186.1750191)
\curveto(501.5180052,185.80880816)(501.65024804,185.41411415)(501.64211002,184.99093707)
\lineto(500.60451236,184.99093707)
\curveto(500.58416731,185.2391467)(500.49668358,185.46497678)(500.34206119,185.6684273)
\curveto(500.08978254,185.95732704)(499.65236392,186.10177691)(499.02980533,186.10177691)
\curveto(498.61476627,186.10177691)(498.29941796,186.02243121)(498.08376041,185.8637398)
\curveto(497.87217186,185.7050484)(497.76637759,185.49549436)(497.76637759,185.23507769)
\curveto(497.76637759,184.95024696)(497.90675845,184.72238238)(498.18752017,184.55148394)
\curveto(498.35028059,184.44975868)(498.5903522,184.36024045)(498.90773502,184.28292926)
\lineto(499.63405338,184.1059273)
\curveto(500.4234414,183.91468381)(500.95241275,183.72954384)(501.22096744,183.55050738)
\curveto(501.64821353,183.26974566)(501.86183658,182.82825803)(501.86183658,182.22604449)
\curveto(501.86183658,181.644176)(501.64007551,181.14165321)(501.19655338,180.71847613)
\curveto(500.75710025,180.29529905)(500.08571353,180.08371051)(499.18239322,180.08371051)
\curveto(498.20989973,180.08371051)(497.52020246,180.30343707)(497.11330142,180.74289019)
\curveto(496.71046939,181.18641233)(496.49481184,181.73369423)(496.46632877,182.3847359)
\lineto(497.52223697,182.3847359)
\closepath
\moveto(499.14577213,187.02951129)
\lineto(499.14577213,187.02951129)
\closepath
}
}
{
\newrgbcolor{curcolor}{0 0 0}
\pscustom[linestyle=none,fillstyle=solid,fillcolor=curcolor]
{
\newpath
\moveto(505.90236392,181.07248004)
\curveto(506.41505923,181.07248004)(506.84027082,181.28610308)(507.17799869,181.71334918)
\curveto(507.51979556,182.14466428)(507.690694,182.78756793)(507.690694,183.64206012)
\curveto(507.690694,184.16289345)(507.61541731,184.61048459)(507.46486392,184.98483355)
\curveto(507.18003319,185.7050484)(506.65919986,186.06515582)(505.90236392,186.06515582)
\curveto(505.14145897,186.06515582)(504.62062564,185.68470334)(504.33986392,184.9237984)
\curveto(504.18931054,184.51689735)(504.11403384,184.00013303)(504.11403384,183.37350543)
\curveto(504.11403384,182.86894814)(504.18931054,182.43966754)(504.33986392,182.08566363)
\curveto(504.62469465,181.4102079)(505.14552798,181.07248004)(505.90236392,181.07248004)
\closepath
\moveto(503.05812564,186.8403023)
\lineto(504.12624088,186.8403023)
\lineto(504.12624088,185.97360308)
\curveto(504.34596744,186.27064084)(504.58603905,186.50053993)(504.84645572,186.66330035)
\curveto(505.21673567,186.90744097)(505.65211978,187.02951129)(506.15260806,187.02951129)
\curveto(506.89316796,187.02951129)(507.52183007,186.74468056)(508.03859439,186.1750191)
\curveto(508.55535871,185.60942665)(508.81374088,184.79969358)(508.81374088,183.74581988)
\curveto(508.81374088,182.32166624)(508.44142642,181.30441363)(507.69679752,180.69406207)
\curveto(507.22479231,180.30750608)(506.6754759,180.11422808)(506.0488483,180.11422808)
\curveto(505.55649804,180.11422808)(505.14349348,180.22205686)(504.80983463,180.43771441)
\curveto(504.61452213,180.55978472)(504.39683007,180.76933876)(504.15675845,181.06637652)
\lineto(504.15675845,177.72775347)
\lineto(503.05812564,177.72775347)
\lineto(503.05812564,186.8403023)
\closepath
}
}
{
\newrgbcolor{curcolor}{0 0 0}
\pscustom[linestyle=none,fillstyle=solid,fillcolor=curcolor]
{
\newpath
\moveto(512.69557681,181.04196246)
\curveto(513.42392968,181.04196246)(513.92238345,181.31662066)(514.19093814,181.86593707)
\curveto(514.46356184,182.41932249)(514.59987369,183.03374306)(514.59987369,183.70919879)
\curveto(514.59987369,184.31955035)(514.50221744,184.81596962)(514.30690494,185.1984566)
\curveto(513.99766015,185.80067014)(513.46461978,186.10177691)(512.70778384,186.10177691)
\curveto(512.03639713,186.10177691)(511.54811588,185.84542926)(511.24294009,185.33273394)
\curveto(510.93776431,184.82003863)(510.78517642,184.20154905)(510.78517642,183.47726519)
\curveto(510.78517642,182.78146441)(510.93776431,182.20163043)(511.24294009,181.73776324)
\curveto(511.54811588,181.27389605)(512.03232811,181.04196246)(512.69557681,181.04196246)
\closepath
\moveto(512.73830142,187.06002887)
\curveto(513.58058658,187.06002887)(514.2926634,186.77926715)(514.87453189,186.21774371)
\curveto(515.45640038,185.65622027)(515.74733463,184.83021116)(515.74733463,183.73971637)
\curveto(515.74733463,182.68584267)(515.49098697,181.81507444)(514.97829166,181.12741168)
\curveto(514.46559634,180.43974892)(513.67010481,180.09591754)(512.59181705,180.09591754)
\curveto(511.69256574,180.09591754)(510.97845442,180.39905881)(510.44948306,181.00534137)
\curveto(509.92051171,181.61569293)(509.65602603,182.43356402)(509.65602603,183.45895465)
\curveto(509.65602603,184.55758746)(509.93475324,185.4324247)(510.49220767,186.08346637)
\curveto(511.0496621,186.73450803)(511.79836002,187.06002887)(512.73830142,187.06002887)
\closepath
\moveto(512.70168033,187.02951129)
\lineto(512.70168033,187.02951129)
\closepath
}
}
{
\newrgbcolor{curcolor}{0 0 0}
\pscustom[linestyle=none,fillstyle=solid,fillcolor=curcolor]
{
\newpath
\moveto(517.05959048,186.87081988)
\lineto(518.10329166,186.87081988)
\lineto(518.10329166,185.94308551)
\curveto(518.41253645,186.32557249)(518.74009179,186.60023069)(519.08595767,186.76706012)
\curveto(519.43182356,186.93388954)(519.81634504,187.01730426)(520.23952213,187.01730426)
\curveto(521.1672565,187.01730426)(521.7938841,186.69381793)(522.11940494,186.04684527)
\curveto(522.2984414,185.69284137)(522.38795963,185.18624957)(522.38795963,184.52706988)
\lineto(522.38795963,180.33395465)
\lineto(521.27101627,180.33395465)
\lineto(521.27101627,184.45382769)
\curveto(521.27101627,184.85259071)(521.21201561,185.17404254)(521.09401431,185.41818316)
\curveto(520.89870181,185.8250842)(520.54469791,186.02853472)(520.03200259,186.02853472)
\curveto(519.77158593,186.02853472)(519.55796288,186.00208616)(519.39113345,185.94918902)
\curveto(519.09002668,185.85967079)(518.82554101,185.68063433)(518.59767642,185.41207965)
\curveto(518.41457095,185.19642209)(518.29453515,184.97262652)(518.237569,184.74069293)
\curveto(518.18467186,184.51282834)(518.1582233,184.18527301)(518.1582233,183.75802691)
\lineto(518.1582233,180.33395465)
\lineto(517.05959048,180.33395465)
\lineto(517.05959048,186.87081988)
\closepath
\moveto(519.64137759,187.02951129)
\lineto(519.64137759,187.02951129)
\closepath
}
}
{
\newrgbcolor{curcolor}{0 0 0}
\pscustom[linestyle=none,fillstyle=solid,fillcolor=curcolor]
{
\newpath
\moveto(524.71339908,183.52609332)
\curveto(524.71339908,182.82622353)(524.86191796,182.24028603)(525.15895572,181.76828082)
\curveto(525.45599348,181.29627561)(525.9320677,181.06027301)(526.58717838,181.06027301)
\curveto(527.09580468,181.06027301)(527.51287824,181.27796506)(527.83839908,181.71334918)
\curveto(528.16798892,182.1528023)(528.33278384,182.78146441)(528.33278384,183.59933551)
\curveto(528.33278384,184.42534462)(528.16391991,185.03569618)(527.82619205,185.43039019)
\curveto(527.48846418,185.82915321)(527.07139061,186.02853472)(526.57497134,186.02853472)
\curveto(526.02158593,186.02853472)(525.57196028,185.81694618)(525.22609439,185.3937691)
\curveto(524.88429752,184.97059202)(524.71339908,184.34803342)(524.71339908,183.52609332)
\closepath
\moveto(526.36745181,186.98678668)
\curveto(526.86794009,186.98678668)(527.28704817,186.88099241)(527.62477603,186.66940387)
\curveto(527.82008853,186.54733355)(528.0418496,186.33371051)(528.29005923,186.02853472)
\lineto(528.29005923,189.33053668)
\lineto(529.34596744,189.33053668)
\lineto(529.34596744,180.33395465)
\lineto(528.35719791,180.33395465)
\lineto(528.35719791,181.24337847)
\curveto(528.10085025,180.84054644)(527.79770897,180.5496122)(527.44777408,180.37057574)
\curveto(527.09783918,180.19153928)(526.69704166,180.10202105)(526.2453815,180.10202105)
\curveto(525.51702864,180.10202105)(524.88633202,180.40719683)(524.35329166,181.0175484)
\curveto(523.82025129,181.63196897)(523.55373111,182.44780556)(523.55373111,183.46505816)
\curveto(523.55373111,184.4172066)(523.79583723,185.24118121)(524.28004947,185.93698199)
\curveto(524.76833072,186.63685178)(525.4641315,186.98678668)(526.36745181,186.98678668)
\closepath
}
}
{
\newrgbcolor{curcolor}{0 0 0}
\pscustom[linestyle=none,fillstyle=solid,fillcolor=curcolor]
{
\newpath
\moveto(533.69777408,187.01730426)
\curveto(534.16164127,187.01730426)(534.61126692,186.90744097)(535.04665103,186.68771441)
\curveto(535.48203515,186.47205686)(535.81365949,186.19129514)(536.04152408,185.84542926)
\curveto(536.26125064,185.51583941)(536.40773502,185.13131793)(536.4809772,184.6918648)
\curveto(536.54608137,184.39075803)(536.57863345,183.9106148)(536.57863345,183.25143512)
\lineto(531.78737369,183.25143512)
\curveto(531.80771874,182.58818642)(531.96437564,182.05514605)(532.25734439,181.65231402)
\curveto(532.55031314,181.253551)(533.0040078,181.05416949)(533.61842838,181.05416949)
\curveto(534.19215884,181.05416949)(534.64992252,181.24337847)(534.99171939,181.62179644)
\curveto(535.18703189,181.84152301)(535.32537824,182.09583616)(535.40675845,182.3847359)
\lineto(536.48708072,182.3847359)
\curveto(536.45859765,182.14466428)(536.3629759,181.87610959)(536.20021548,181.57907183)
\curveto(536.04152408,181.28610308)(535.86248762,181.04603147)(535.66310611,180.85885699)
\curveto(535.32944726,180.53333616)(534.9164427,180.31360959)(534.42409244,180.1996773)
\curveto(534.15960676,180.13457314)(533.86053449,180.10202105)(533.52687564,180.10202105)
\curveto(532.71307356,180.10202105)(532.02337629,180.39702431)(531.45778384,180.98703082)
\curveto(530.8921914,181.58110634)(530.60939517,182.41118446)(530.60939517,183.47726519)
\curveto(530.60939517,184.52706988)(530.8942259,185.37952756)(531.46388736,186.03463824)
\curveto(532.03354882,186.68974892)(532.77817772,187.01730426)(533.69777408,187.01730426)
\closepath
\moveto(535.44948306,184.12423785)
\curveto(535.40472395,184.60031207)(535.30096418,184.98076454)(535.13820377,185.26559527)
\curveto(534.83709699,185.79456663)(534.33457421,186.0590523)(533.63063541,186.0590523)
\curveto(533.12607811,186.0590523)(532.70290103,185.87594683)(532.36110416,185.5097359)
\curveto(532.01930728,185.14759397)(531.83823632,184.68576129)(531.81789127,184.12423785)
\lineto(535.44948306,184.12423785)
\closepath
\moveto(533.59401431,187.02951129)
\lineto(533.59401431,187.02951129)
\closepath
}
}
{
\newrgbcolor{curcolor}{0 0 0}
\pscustom[linestyle=none,fillstyle=solid,fillcolor=curcolor]
{
\newpath
\moveto(537.9641315,186.87081988)
\lineto(539.00783267,186.87081988)
\lineto(539.00783267,185.74166949)
\curveto(539.09328189,185.96139605)(539.30283593,186.22791624)(539.63649478,186.54123004)
\curveto(539.97015364,186.85861285)(540.35467512,187.01730426)(540.79005923,187.01730426)
\curveto(540.81040429,187.01730426)(540.84499088,187.01526975)(540.893819,187.01120074)
\curveto(540.94264713,187.00713173)(541.02606184,186.99899371)(541.14406314,186.98678668)
\lineto(541.14406314,185.82711871)
\curveto(541.07895897,185.83932574)(541.01792382,185.84746376)(540.96095767,185.85153277)
\curveto(540.90806054,185.85560178)(540.84905989,185.85763629)(540.78395572,185.85763629)
\curveto(540.2305703,185.85763629)(539.80535871,185.67859983)(539.50832095,185.32052691)
\curveto(539.21128319,184.96652301)(539.06276431,184.55758746)(539.06276431,184.09372027)
\lineto(539.06276431,180.33395465)
\lineto(537.9641315,180.33395465)
\lineto(537.9641315,186.87081988)
\closepath
}
}
{
\newrgbcolor{curcolor}{0 0 0}
\pscustom[linewidth=1,linecolor=curcolor]
{
\newpath
\moveto(139.550069,120.33398516)
\lineto(219.550069,120.33398516)
\lineto(219.550069,90.33398516)
\lineto(139.550069,90.33398516)
\closepath
}
}
{
\newrgbcolor{curcolor}{0 0 0}
\pscustom[linestyle=none,fillstyle=solid,fillcolor=curcolor]
{
\newpath
\moveto(151.29567447,103.22705157)
\curveto(151.32415754,102.71842527)(151.44419335,102.30542071)(151.65578189,101.9880379)
\curveto(152.05861392,101.39396238)(152.76865624,101.09692462)(153.78590884,101.09692462)
\curveto(154.24163801,101.09692462)(154.65667707,101.16202878)(155.03102603,101.29223712)
\curveto(155.75530989,101.54451576)(156.11745181,101.99617592)(156.11745181,102.64721759)
\curveto(156.11745181,103.13549884)(155.96486392,103.48339923)(155.65968814,103.69091876)
\curveto(155.35044335,103.89436928)(154.86623111,104.07137123)(154.20705142,104.22192462)
\lineto(152.99245181,104.49658282)
\curveto(152.19899478,104.67561928)(151.63747134,104.87296628)(151.3078815,105.08862384)
\curveto(150.73822004,105.46297279)(150.45338931,106.02246173)(150.45338931,106.76709063)
\curveto(150.45338931,107.5727547)(150.73211653,108.23396889)(151.28957095,108.75073321)
\curveto(151.84702538,109.26749753)(152.6364134,109.5258797)(153.65773502,109.5258797)
\curveto(154.59767642,109.5258797)(155.39520246,109.29801511)(156.05031314,108.84228595)
\curveto(156.70949283,108.39062579)(157.03908267,107.66634194)(157.03908267,106.66943438)
\lineto(155.89772525,106.66943438)
\curveto(155.83669009,107.14957761)(155.70648176,107.51782305)(155.50710025,107.77417071)
\curveto(155.1368203,108.24210691)(154.50815819,108.47607501)(153.62111392,108.47607501)
\curveto(152.90496809,108.47607501)(152.39023827,108.32552162)(152.07692447,108.02441485)
\curveto(151.76361067,107.72330808)(151.60695377,107.37337319)(151.60695377,106.97461016)
\curveto(151.60695377,106.53515704)(151.79005923,106.21370522)(152.15627017,106.0102547)
\curveto(152.39634179,105.88004636)(152.93955468,105.71728595)(153.78590884,105.52197345)
\lineto(155.04323306,105.23510821)
\curveto(155.64951561,105.09676186)(156.11745181,104.90755287)(156.44704166,104.66748126)
\curveto(157.01670311,104.24837319)(157.30153384,103.64005613)(157.30153384,102.84253009)
\curveto(157.30153384,101.84969154)(156.93939192,101.13964923)(156.21510806,100.71240313)
\curveto(155.49489322,100.28515704)(154.65667707,100.07153399)(153.70045963,100.07153399)
\curveto(152.58555077,100.07153399)(151.71274804,100.35636472)(151.08205142,100.92602618)
\curveto(150.45135481,101.49161863)(150.14211002,102.25862709)(150.15431705,103.22705157)
\lineto(151.29567447,103.22705157)
\closepath
\moveto(153.74928775,109.54419024)
\lineto(153.74928775,109.54419024)
\closepath
}
}
{
\newrgbcolor{curcolor}{0 0 0}
\pscustom[linestyle=none,fillstyle=solid,fillcolor=curcolor]
{
\newpath
\moveto(161.40309634,107.01733477)
\curveto(161.86696353,107.01733477)(162.31658918,106.90747149)(162.7519733,106.68774493)
\curveto(163.18735741,106.47208738)(163.51898176,106.19132566)(163.74684634,105.84545977)
\curveto(163.96657291,105.51586993)(164.11305728,105.13134845)(164.18629947,104.69189532)
\curveto(164.25140364,104.39078855)(164.28395572,103.91064532)(164.28395572,103.25146563)
\lineto(159.49269595,103.25146563)
\curveto(159.51304101,102.58821694)(159.66969791,102.05517657)(159.96266666,101.65234454)
\curveto(160.25563541,101.25358152)(160.70933007,101.05420001)(161.32375064,101.05420001)
\curveto(161.89748111,101.05420001)(162.35524478,101.24340899)(162.69704166,101.62182696)
\curveto(162.89235416,101.84155352)(163.03070051,102.09586667)(163.11208072,102.38476641)
\lineto(164.19240298,102.38476641)
\curveto(164.16391991,102.1446948)(164.06829817,101.87614011)(163.90553775,101.57910235)
\curveto(163.74684634,101.2861336)(163.56780989,101.04606199)(163.36842838,100.85888751)
\curveto(163.03476952,100.53336667)(162.62176496,100.31364011)(162.1294147,100.19970782)
\curveto(161.86492903,100.13460365)(161.56585676,100.10205157)(161.23219791,100.10205157)
\curveto(160.41839582,100.10205157)(159.72869856,100.39705483)(159.16310611,100.98706134)
\curveto(158.59751366,101.58113686)(158.31471744,102.41121498)(158.31471744,103.47729571)
\curveto(158.31471744,104.5271004)(158.59954817,105.37955808)(159.16920963,106.03466876)
\curveto(159.73887108,106.68977944)(160.48349999,107.01733477)(161.40309634,107.01733477)
\closepath
\moveto(163.15480533,104.12426837)
\curveto(163.11004621,104.60034259)(163.00628645,104.98079506)(162.84352603,105.26562579)
\curveto(162.54241926,105.79459714)(162.03989647,106.05908282)(161.33595767,106.05908282)
\curveto(160.83140038,106.05908282)(160.4082233,105.87597735)(160.06642642,105.50976641)
\curveto(159.72462955,105.14762449)(159.54355858,104.6857918)(159.52321353,104.12426837)
\lineto(163.15480533,104.12426837)
\closepath
\moveto(161.29933658,107.0295418)
\lineto(161.29933658,107.0295418)
\closepath
}
}
{
\newrgbcolor{curcolor}{0 0 0}
\pscustom[linestyle=none,fillstyle=solid,fillcolor=curcolor]
{
\newpath
\moveto(165.63893619,106.8708504)
\lineto(166.68263736,106.8708504)
\lineto(166.68263736,105.94311602)
\curveto(166.99188215,106.325603)(167.31943749,106.60026121)(167.66530338,106.76709063)
\curveto(168.01116926,106.93392006)(168.39569074,107.01733477)(168.81886783,107.01733477)
\curveto(169.7466022,107.01733477)(170.37322981,106.69384845)(170.69875064,106.04687579)
\curveto(170.8777871,105.69287188)(170.96730533,105.18628009)(170.96730533,104.5271004)
\lineto(170.96730533,100.33398516)
\lineto(169.85036197,100.33398516)
\lineto(169.85036197,104.45385821)
\curveto(169.85036197,104.85262123)(169.79136132,105.17407305)(169.67336002,105.41821368)
\curveto(169.47804752,105.82511472)(169.12404361,106.02856524)(168.6113483,106.02856524)
\curveto(168.35093163,106.02856524)(168.13730858,106.00211667)(167.97047916,105.94921954)
\curveto(167.66937239,105.85970131)(167.40488671,105.68066485)(167.17702213,105.41211016)
\curveto(166.99391666,105.19645261)(166.87388085,104.97265704)(166.8169147,104.74072345)
\curveto(166.76401757,104.51285886)(166.737569,104.18530352)(166.737569,103.75805743)
\lineto(166.737569,100.33398516)
\lineto(165.63893619,100.33398516)
\lineto(165.63893619,106.8708504)
\closepath
\moveto(168.2207233,107.0295418)
\lineto(168.2207233,107.0295418)
\closepath
}
}
{
\newrgbcolor{curcolor}{0 0 0}
\pscustom[linestyle=none,fillstyle=solid,fillcolor=curcolor]
{
\newpath
\moveto(173.25002017,102.38476641)
\curveto(173.28257226,102.01855548)(173.37412499,101.73779376)(173.52467838,101.54248126)
\curveto(173.80137108,101.18847735)(174.28151431,101.0114754)(174.96510806,101.0114754)
\curveto(175.3720091,101.0114754)(175.73008202,101.09895912)(176.03932681,101.27392657)
\curveto(176.3485716,101.45296303)(176.503194,101.72762123)(176.503194,102.09790118)
\curveto(176.503194,102.3786629)(176.37908918,102.59228595)(176.13087955,102.73877032)
\curveto(175.97218814,102.82828855)(175.65887434,102.93204832)(175.19093814,103.05004962)
\lineto(174.31813541,103.26977618)
\curveto(173.76068098,103.40812253)(173.34971093,103.56274493)(173.08522525,103.73364337)
\curveto(172.61322004,104.03068113)(172.37721744,104.44165118)(172.37721744,104.96655352)
\curveto(172.37721744,105.58504311)(172.59897851,106.08553139)(173.04250064,106.46801837)
\curveto(173.49009179,106.85050535)(174.09027082,107.04174884)(174.84303775,107.04174884)
\curveto(175.82773827,107.04174884)(176.53778059,106.7528491)(176.9731647,106.17504962)
\curveto(177.2457884,105.80883868)(177.37803124,105.41414467)(177.36989322,104.99096759)
\lineto(176.33229556,104.99096759)
\curveto(176.31195051,105.23917722)(176.22446679,105.4650073)(176.06984439,105.66845782)
\curveto(175.81756574,105.95735756)(175.38014713,106.10180743)(174.75758853,106.10180743)
\curveto(174.34254947,106.10180743)(174.02720116,106.02246173)(173.81154361,105.86377032)
\curveto(173.59995507,105.70507891)(173.4941608,105.49552488)(173.4941608,105.23510821)
\curveto(173.4941608,104.95027748)(173.63454166,104.7224129)(173.91530338,104.55151446)
\curveto(174.07806379,104.4497892)(174.31813541,104.36027097)(174.63551822,104.28295977)
\lineto(175.36183658,104.10595782)
\curveto(176.1512246,103.91471433)(176.68019595,103.72957436)(176.94875064,103.5505379)
\curveto(177.37599673,103.26977618)(177.58961978,102.82828855)(177.58961978,102.22607501)
\curveto(177.58961978,101.64420652)(177.36785871,101.14168373)(176.92433658,100.71850665)
\curveto(176.48488345,100.29532957)(175.81349673,100.08374102)(174.91017642,100.08374102)
\curveto(173.93768293,100.08374102)(173.24798567,100.30346759)(172.84108463,100.74292071)
\curveto(172.43825259,101.18644285)(172.22259504,101.73372475)(172.19411197,102.38476641)
\lineto(173.25002017,102.38476641)
\closepath
\moveto(174.87355533,107.0295418)
\lineto(174.87355533,107.0295418)
\closepath
}
}
{
\newrgbcolor{curcolor}{0 0 0}
\pscustom[linestyle=none,fillstyle=solid,fillcolor=curcolor]
{
\newpath
\moveto(181.4653522,101.04199298)
\curveto(182.19370507,101.04199298)(182.69215884,101.31665118)(182.96071353,101.86596759)
\curveto(183.23333723,102.419353)(183.36964908,103.03377358)(183.36964908,103.7092293)
\curveto(183.36964908,104.31958087)(183.27199283,104.81600014)(183.07668033,105.19848712)
\curveto(182.76743554,105.80070066)(182.23439517,106.10180743)(181.47755923,106.10180743)
\curveto(180.80617252,106.10180743)(180.31789127,105.84545977)(180.01271548,105.33276446)
\curveto(179.7075397,104.82006915)(179.55495181,104.20157957)(179.55495181,103.47729571)
\curveto(179.55495181,102.78149493)(179.7075397,102.20166095)(180.01271548,101.73779376)
\curveto(180.31789127,101.27392657)(180.80210351,101.04199298)(181.4653522,101.04199298)
\closepath
\moveto(181.50807681,107.06005938)
\curveto(182.35036197,107.06005938)(183.06243879,106.77929766)(183.64430728,106.21777423)
\curveto(184.22617577,105.65625079)(184.51711002,104.83024167)(184.51711002,103.73974688)
\curveto(184.51711002,102.68587319)(184.26076236,101.81510496)(183.74806705,101.1274422)
\curveto(183.23537173,100.43977944)(182.4398802,100.09594805)(181.36159244,100.09594805)
\curveto(180.46234114,100.09594805)(179.74822981,100.39908933)(179.21925845,101.00537188)
\curveto(178.6902871,101.61572345)(178.42580142,102.43359454)(178.42580142,103.45898516)
\curveto(178.42580142,104.55761798)(178.70452864,105.43245522)(179.26198306,106.08349688)
\curveto(179.81943749,106.73453855)(180.56813541,107.06005938)(181.50807681,107.06005938)
\closepath
\moveto(181.47145572,107.0295418)
\lineto(181.47145572,107.0295418)
\closepath
}
}
{
\newrgbcolor{curcolor}{0 0 0}
\pscustom[linestyle=none,fillstyle=solid,fillcolor=curcolor]
{
\newpath
\moveto(185.85988345,106.8708504)
\lineto(186.90358463,106.8708504)
\lineto(186.90358463,105.74170001)
\curveto(186.98903384,105.96142657)(187.19858788,106.22794675)(187.53224673,106.54126055)
\curveto(187.86590559,106.85864337)(188.25042707,107.01733477)(188.68581119,107.01733477)
\curveto(188.70615624,107.01733477)(188.74074283,107.01530027)(188.78957095,107.01123126)
\curveto(188.83839908,107.00716225)(188.92181379,106.99902423)(189.03981509,106.9868172)
\lineto(189.03981509,105.82714923)
\curveto(188.97471093,105.83935626)(188.91367577,105.84749428)(188.85670963,105.85156329)
\curveto(188.80381249,105.8556323)(188.74481184,105.8576668)(188.67970767,105.8576668)
\curveto(188.12632226,105.8576668)(187.70111067,105.67863035)(187.40407291,105.32055743)
\curveto(187.10703515,104.96655352)(186.95851627,104.55761798)(186.95851627,104.09375079)
\lineto(186.95851627,100.33398516)
\lineto(185.85988345,100.33398516)
\lineto(185.85988345,106.8708504)
\closepath
}
}
{
\newrgbcolor{curcolor}{0 0 0}
\pscustom[linestyle=none,fillstyle=solid,fillcolor=curcolor]
{
\newpath
\moveto(190.65724673,102.38476641)
\curveto(190.68979882,102.01855548)(190.78135155,101.73779376)(190.93190494,101.54248126)
\curveto(191.20859765,101.18847735)(191.68874088,101.0114754)(192.37233463,101.0114754)
\curveto(192.77923567,101.0114754)(193.13730858,101.09895912)(193.44655338,101.27392657)
\curveto(193.75579817,101.45296303)(193.91042056,101.72762123)(193.91042056,102.09790118)
\curveto(193.91042056,102.3786629)(193.78631574,102.59228595)(193.53810611,102.73877032)
\curveto(193.3794147,102.82828855)(193.0661009,102.93204832)(192.5981647,103.05004962)
\lineto(191.72536197,103.26977618)
\curveto(191.16790754,103.40812253)(190.75693749,103.56274493)(190.49245181,103.73364337)
\curveto(190.0204466,104.03068113)(189.784444,104.44165118)(189.784444,104.96655352)
\curveto(189.784444,105.58504311)(190.00620507,106.08553139)(190.4497272,106.46801837)
\curveto(190.89731835,106.85050535)(191.49749739,107.04174884)(192.25026431,107.04174884)
\curveto(193.23496483,107.04174884)(193.94500715,106.7528491)(194.38039127,106.17504962)
\curveto(194.65301496,105.80883868)(194.7852578,105.41414467)(194.77711978,104.99096759)
\lineto(193.73952213,104.99096759)
\curveto(193.71917707,105.23917722)(193.63169335,105.4650073)(193.47707095,105.66845782)
\curveto(193.22479231,105.95735756)(192.78737369,106.10180743)(192.16481509,106.10180743)
\curveto(191.74977603,106.10180743)(191.43442772,106.02246173)(191.21877017,105.86377032)
\curveto(191.00718163,105.70507891)(190.90138736,105.49552488)(190.90138736,105.23510821)
\curveto(190.90138736,104.95027748)(191.04176822,104.7224129)(191.32252994,104.55151446)
\curveto(191.48529035,104.4497892)(191.72536197,104.36027097)(192.04274478,104.28295977)
\lineto(192.76906314,104.10595782)
\curveto(193.55845116,103.91471433)(194.08742252,103.72957436)(194.3559772,103.5505379)
\curveto(194.7832233,103.26977618)(194.99684634,102.82828855)(194.99684634,102.22607501)
\curveto(194.99684634,101.64420652)(194.77508528,101.14168373)(194.33156314,100.71850665)
\curveto(193.89211002,100.29532957)(193.2207233,100.08374102)(192.31740298,100.08374102)
\curveto(191.34490949,100.08374102)(190.65521223,100.30346759)(190.24831119,100.74292071)
\curveto(189.84547916,101.18644285)(189.6298216,101.73372475)(189.60133853,102.38476641)
\lineto(190.65724673,102.38476641)
\closepath
\moveto(192.28078189,107.0295418)
\lineto(192.28078189,107.0295418)
\closepath
}
}
{
\newrgbcolor{curcolor}{0 0 0}
\pscustom[linewidth=1,linecolor=curcolor]
{
\newpath
\moveto(179.550069,150.33396778)
\lineto(179.550069,120.33396778)
}
}
{
\newrgbcolor{curcolor}{0 0 0}
\pscustom[linestyle=none,fillstyle=solid,fillcolor=curcolor]
{
\newpath
\moveto(179.550069,146.33396778)
\lineto(181.550069,144.33396778)
\lineto(179.550069,151.33396778)
\lineto(177.550069,144.33396778)
\lineto(179.550069,146.33396778)
\closepath
}
}
{
\newrgbcolor{curcolor}{0 0 0}
\pscustom[linewidth=0.5,linecolor=curcolor]
{
\newpath
\moveto(179.550069,146.33396778)
\lineto(181.550069,144.33396778)
\lineto(179.550069,151.33396778)
\lineto(177.550069,144.33396778)
\lineto(179.550069,146.33396778)
\closepath
}
}
{
\newrgbcolor{curcolor}{0 0 0}
\pscustom[linestyle=none,fillstyle=solid,fillcolor=curcolor]
{
\newpath
\moveto(179.550069,124.33396778)
\lineto(177.550069,126.33396778)
\lineto(179.550069,119.33396778)
\lineto(181.550069,126.33396778)
\lineto(179.550069,124.33396778)
\closepath
}
}
{
\newrgbcolor{curcolor}{0 0 0}
\pscustom[linewidth=0.5,linecolor=curcolor]
{
\newpath
\moveto(179.550069,124.33396778)
\lineto(177.550069,126.33396778)
\lineto(179.550069,119.33396778)
\lineto(181.550069,126.33396778)
\lineto(179.550069,124.33396778)
\closepath
}
}
{
\newrgbcolor{curcolor}{0 0 0}
\pscustom[linewidth=1,linecolor=curcolor]
{
\newpath
\moveto(469.21602359,119.76513751)
\lineto(549.21602359,119.76513751)
\lineto(549.21602359,89.76513751)
\lineto(469.21602359,89.76513751)
\closepath
}
}
{
\newrgbcolor{curcolor}{0 0 0}
\pscustom[linestyle=none,fillstyle=solid,fillcolor=curcolor]
{
\newpath
\moveto(480.96162906,102.65820391)
\curveto(480.99011213,102.14957761)(481.11014794,101.73657305)(481.32173648,101.41919024)
\curveto(481.72456851,100.82511472)(482.43461083,100.52807696)(483.45186343,100.52807696)
\curveto(483.9075926,100.52807696)(484.32263166,100.59318113)(484.69698062,100.72338946)
\curveto(485.42126448,100.97566811)(485.7834064,101.42732826)(485.7834064,102.07836993)
\curveto(485.7834064,102.56665118)(485.63081851,102.91455157)(485.32564273,103.1220711)
\curveto(485.01639794,103.32552162)(484.5321857,103.50252358)(483.87300601,103.65307696)
\lineto(482.6584064,103.92773516)
\curveto(481.86494937,104.10677162)(481.30342593,104.30411863)(480.97383609,104.51977618)
\curveto(480.40417463,104.89412514)(480.1193439,105.45361407)(480.1193439,106.19824298)
\curveto(480.1193439,107.00390704)(480.39807112,107.66512123)(480.95552554,108.18188555)
\curveto(481.51297997,108.69864988)(482.30236799,108.95703204)(483.32368961,108.95703204)
\curveto(484.26363101,108.95703204)(485.06115705,108.72916746)(485.71626773,108.27343829)
\curveto(486.37544742,107.82177813)(486.70503726,107.09749428)(486.70503726,106.10058673)
\lineto(485.56367984,106.10058673)
\curveto(485.50264468,106.58072996)(485.37243635,106.9489754)(485.17305484,107.20532305)
\curveto(484.80277489,107.67325925)(484.17411278,107.90722735)(483.28706851,107.90722735)
\curveto(482.57092268,107.90722735)(482.05619286,107.75667397)(481.74287906,107.4555672)
\curveto(481.42956526,107.15446042)(481.27290836,106.80452553)(481.27290836,106.40576251)
\curveto(481.27290836,105.96630938)(481.45601382,105.64485756)(481.82222476,105.44140704)
\curveto(482.06229638,105.31119871)(482.60550927,105.14843829)(483.45186343,104.95312579)
\lineto(484.70918765,104.66626055)
\curveto(485.3154702,104.5279142)(485.7834064,104.33870522)(486.11299625,104.0986336)
\curveto(486.6826577,103.67952553)(486.96748843,103.07120847)(486.96748843,102.27368243)
\curveto(486.96748843,101.28084389)(486.60534651,100.57080157)(485.88106265,100.14355548)
\curveto(485.16084781,99.71630938)(484.32263166,99.50268634)(483.36641421,99.50268634)
\curveto(482.25150536,99.50268634)(481.37870263,99.78751707)(480.74800601,100.35717852)
\curveto(480.1173094,100.92277097)(479.80806461,101.68977944)(479.82027164,102.65820391)
\lineto(480.96162906,102.65820391)
\closepath
\moveto(483.41524234,108.97534259)
\lineto(483.41524234,108.97534259)
\closepath
}
}
{
\newrgbcolor{curcolor}{0 0 0}
\pscustom[linestyle=none,fillstyle=solid,fillcolor=curcolor]
{
\newpath
\moveto(491.06905093,106.44848712)
\curveto(491.53291812,106.44848712)(491.98254377,106.33862384)(492.41792789,106.11889727)
\curveto(492.853312,105.90323972)(493.18493635,105.622478)(493.41280093,105.27661212)
\curveto(493.6325275,104.94702227)(493.77901187,104.56250079)(493.85225406,104.12304766)
\curveto(493.91735823,103.82194089)(493.94991031,103.34179766)(493.94991031,102.68261798)
\lineto(489.15865054,102.68261798)
\curveto(489.1789956,102.01936928)(489.3356525,101.48632891)(489.62862125,101.08349688)
\curveto(489.92159,100.68473386)(490.37528466,100.48535235)(490.98970523,100.48535235)
\curveto(491.5634357,100.48535235)(492.02119937,100.67456134)(492.36299625,101.0529793)
\curveto(492.55830875,101.27270587)(492.6966551,101.52701902)(492.77803531,101.81591876)
\lineto(493.85835757,101.81591876)
\curveto(493.8298745,101.57584714)(493.73425276,101.30729246)(493.57149234,101.0102547)
\curveto(493.41280093,100.71728595)(493.23376448,100.47721433)(493.03438296,100.29003985)
\curveto(492.70072411,99.96451902)(492.28771955,99.74479246)(491.79536929,99.63086016)
\curveto(491.53088362,99.565756)(491.23181135,99.53320391)(490.8981525,99.53320391)
\curveto(490.08435041,99.53320391)(489.39465315,99.82820717)(488.8290607,100.41821368)
\curveto(488.26346825,101.0122892)(487.98067203,101.84236733)(487.98067203,102.90844805)
\curveto(487.98067203,103.95825274)(488.26550276,104.81071042)(488.83516421,105.4658211)
\curveto(489.40482567,106.12093178)(490.14945458,106.44848712)(491.06905093,106.44848712)
\closepath
\moveto(492.82075992,103.55542071)
\curveto(492.7760008,104.03149493)(492.67224104,104.4119474)(492.50948062,104.69677813)
\curveto(492.20837385,105.22574949)(491.70585106,105.49023516)(491.00191226,105.49023516)
\curveto(490.49735497,105.49023516)(490.07417789,105.3071297)(489.73238101,104.94091876)
\curveto(489.39058414,104.57877683)(489.20951317,104.11694415)(489.18916812,103.55542071)
\lineto(492.82075992,103.55542071)
\closepath
\moveto(490.96529117,106.46069415)
\lineto(490.96529117,106.46069415)
\closepath
}
}
{
\newrgbcolor{curcolor}{0 0 0}
\pscustom[linestyle=none,fillstyle=solid,fillcolor=curcolor]
{
\newpath
\moveto(495.30489078,106.30200274)
\lineto(496.34859195,106.30200274)
\lineto(496.34859195,105.37426837)
\curveto(496.65783674,105.75675535)(496.98539208,106.03141355)(497.33125796,106.19824298)
\curveto(497.67712385,106.3650724)(498.06164533,106.44848712)(498.48482242,106.44848712)
\curveto(499.41255679,106.44848712)(500.0391844,106.12500079)(500.36470523,105.47802813)
\curveto(500.54374169,105.12402423)(500.63325992,104.61743243)(500.63325992,103.95825274)
\lineto(500.63325992,99.76513751)
\lineto(499.51631656,99.76513751)
\lineto(499.51631656,103.88501055)
\curveto(499.51631656,104.28377358)(499.45731591,104.6052254)(499.33931461,104.84936602)
\curveto(499.14400211,105.25626707)(498.7899982,105.45971759)(498.27730289,105.45971759)
\curveto(498.01688622,105.45971759)(497.80326317,105.43326902)(497.63643375,105.38037188)
\curveto(497.33532698,105.29085365)(497.0708413,105.1118172)(496.84297671,104.84326251)
\curveto(496.65987125,104.62760496)(496.53983544,104.40380938)(496.48286929,104.17187579)
\curveto(496.42997216,103.94401121)(496.40352359,103.61645587)(496.40352359,103.18920977)
\lineto(496.40352359,99.76513751)
\lineto(495.30489078,99.76513751)
\lineto(495.30489078,106.30200274)
\closepath
\moveto(497.88667789,106.46069415)
\lineto(497.88667789,106.46069415)
\closepath
}
}
{
\newrgbcolor{curcolor}{0 0 0}
\pscustom[linestyle=none,fillstyle=solid,fillcolor=curcolor]
{
\newpath
\moveto(502.91597476,101.81591876)
\curveto(502.94852685,101.44970782)(503.04007958,101.1689461)(503.19063296,100.9736336)
\curveto(503.46732567,100.6196297)(503.9474689,100.44262774)(504.63106265,100.44262774)
\curveto(505.03796369,100.44262774)(505.39603661,100.53011147)(505.7052814,100.70507891)
\curveto(506.01452619,100.88411537)(506.16914859,101.15877358)(506.16914859,101.52905352)
\curveto(506.16914859,101.80981524)(506.04504377,102.02343829)(505.79683414,102.16992266)
\curveto(505.63814273,102.25944089)(505.32482893,102.36320066)(504.85689273,102.48120196)
\lineto(503.98409,102.70092852)
\curveto(503.42663557,102.83927488)(503.01566552,102.99389727)(502.75117984,103.16479571)
\curveto(502.27917463,103.46183347)(502.04317203,103.87280352)(502.04317203,104.39770587)
\curveto(502.04317203,105.01619545)(502.2649331,105.51668373)(502.70845523,105.89917071)
\curveto(503.15604638,106.28165769)(503.75622541,106.47290118)(504.50899234,106.47290118)
\curveto(505.49369286,106.47290118)(506.20373518,106.18400144)(506.63911929,105.60620196)
\curveto(506.91174299,105.23999102)(507.04398583,104.84529701)(507.03584781,104.42211993)
\lineto(505.99825015,104.42211993)
\curveto(505.9779051,104.67032957)(505.89042138,104.89615964)(505.73579898,105.09961016)
\curveto(505.48352033,105.3885099)(505.04610171,105.53295977)(504.42354312,105.53295977)
\curveto(504.00850406,105.53295977)(503.69315575,105.45361407)(503.4774982,105.29492266)
\curveto(503.26590966,105.13623126)(503.16011539,104.92667722)(503.16011539,104.66626055)
\curveto(503.16011539,104.38142983)(503.30049625,104.15356524)(503.58125796,103.9826668)
\curveto(503.74401838,103.88094154)(503.98409,103.79142332)(504.30147281,103.71411212)
\lineto(505.02779117,103.53711016)
\curveto(505.81717919,103.34586667)(506.34615054,103.1607267)(506.61470523,102.98169024)
\curveto(507.04195132,102.70092852)(507.25557437,102.25944089)(507.25557437,101.65722735)
\curveto(507.25557437,101.07535886)(507.0338133,100.57283608)(506.59029117,100.14965899)
\curveto(506.15083804,99.72648191)(505.47945132,99.51489337)(504.57613101,99.51489337)
\curveto(503.60363752,99.51489337)(502.91394026,99.73461993)(502.50703921,100.17407305)
\curveto(502.10420718,100.61759519)(501.88854963,101.16487709)(501.86006656,101.81591876)
\lineto(502.91597476,101.81591876)
\closepath
\moveto(504.53950992,106.46069415)
\lineto(504.53950992,106.46069415)
\closepath
}
}
{
\newrgbcolor{curcolor}{0 0 0}
\pscustom[linestyle=none,fillstyle=solid,fillcolor=curcolor]
{
\newpath
\moveto(511.13130679,100.47314532)
\curveto(511.85965966,100.47314532)(512.35811343,100.74780352)(512.62666812,101.29711993)
\curveto(512.89929182,101.85050535)(513.03560367,102.46492592)(513.03560367,103.14038165)
\curveto(513.03560367,103.75073321)(512.93794742,104.24715248)(512.74263492,104.62963946)
\curveto(512.43339013,105.231853)(511.90034976,105.53295977)(511.14351382,105.53295977)
\curveto(510.47212711,105.53295977)(509.98384586,105.27661212)(509.67867007,104.7639168)
\curveto(509.37349429,104.25122149)(509.2209064,103.63273191)(509.2209064,102.90844805)
\curveto(509.2209064,102.21264727)(509.37349429,101.63281329)(509.67867007,101.1689461)
\curveto(509.98384586,100.70507891)(510.4680581,100.47314532)(511.13130679,100.47314532)
\closepath
\moveto(511.1740314,106.49121173)
\curveto(512.01631656,106.49121173)(512.72839338,106.21045001)(513.31026187,105.64892657)
\curveto(513.89213036,105.08740313)(514.18306461,104.26139402)(514.18306461,103.17089923)
\curveto(514.18306461,102.11702553)(513.92671695,101.2462573)(513.41402164,100.55859454)
\curveto(512.90132632,99.87093178)(512.10583479,99.5271004)(511.02754703,99.5271004)
\curveto(510.12829573,99.5271004)(509.4141844,99.83024167)(508.88521304,100.43652423)
\curveto(508.35624169,101.04687579)(508.09175601,101.86474688)(508.09175601,102.89013751)
\curveto(508.09175601,103.98877032)(508.37048323,104.86360756)(508.92793765,105.51464923)
\curveto(509.48539208,106.16569089)(510.23409,106.49121173)(511.1740314,106.49121173)
\closepath
\moveto(511.13741031,106.46069415)
\lineto(511.13741031,106.46069415)
\closepath
}
}
{
\newrgbcolor{curcolor}{0 0 0}
\pscustom[linestyle=none,fillstyle=solid,fillcolor=curcolor]
{
\newpath
\moveto(515.52583804,106.30200274)
\lineto(516.56953921,106.30200274)
\lineto(516.56953921,105.17285235)
\curveto(516.65498843,105.39257891)(516.86454247,105.6590991)(517.19820132,105.9724129)
\curveto(517.53186018,106.28979571)(517.91638166,106.44848712)(518.35176578,106.44848712)
\curveto(518.37211083,106.44848712)(518.40669742,106.44645261)(518.45552554,106.4423836)
\curveto(518.50435367,106.43831459)(518.58776838,106.43017657)(518.70576968,106.41796954)
\lineto(518.70576968,105.25830157)
\curveto(518.64066552,105.2705086)(518.57963036,105.27864662)(518.52266421,105.28271563)
\curveto(518.46976708,105.28678464)(518.41076643,105.28881915)(518.34566226,105.28881915)
\curveto(517.79227685,105.28881915)(517.36706526,105.10978269)(517.0700275,104.75170977)
\curveto(516.77298974,104.39770587)(516.62447086,103.98877032)(516.62447086,103.52490313)
\lineto(516.62447086,99.76513751)
\lineto(515.52583804,99.76513751)
\lineto(515.52583804,106.30200274)
\closepath
}
}
{
\newrgbcolor{curcolor}{0 0 0}
\pscustom[linestyle=none,fillstyle=solid,fillcolor=curcolor]
{
\newpath
\moveto(520.32320132,101.81591876)
\curveto(520.35575341,101.44970782)(520.44730614,101.1689461)(520.59785953,100.9736336)
\curveto(520.87455224,100.6196297)(521.35469546,100.44262774)(522.03828921,100.44262774)
\curveto(522.44519026,100.44262774)(522.80326317,100.53011147)(523.11250796,100.70507891)
\curveto(523.42175276,100.88411537)(523.57637515,101.15877358)(523.57637515,101.52905352)
\curveto(523.57637515,101.80981524)(523.45227033,102.02343829)(523.2040607,102.16992266)
\curveto(523.04536929,102.25944089)(522.73205549,102.36320066)(522.26411929,102.48120196)
\lineto(521.39131656,102.70092852)
\curveto(520.83386213,102.83927488)(520.42289208,102.99389727)(520.1584064,103.16479571)
\curveto(519.68640119,103.46183347)(519.45039859,103.87280352)(519.45039859,104.39770587)
\curveto(519.45039859,105.01619545)(519.67215966,105.51668373)(520.11568179,105.89917071)
\curveto(520.56327294,106.28165769)(521.16345198,106.47290118)(521.9162189,106.47290118)
\curveto(522.90091942,106.47290118)(523.61096174,106.18400144)(524.04634586,105.60620196)
\curveto(524.31896955,105.23999102)(524.45121239,104.84529701)(524.44307437,104.42211993)
\lineto(523.40547671,104.42211993)
\curveto(523.38513166,104.67032957)(523.29764794,104.89615964)(523.14302554,105.09961016)
\curveto(522.8907469,105.3885099)(522.45332828,105.53295977)(521.83076968,105.53295977)
\curveto(521.41573062,105.53295977)(521.10038231,105.45361407)(520.88472476,105.29492266)
\curveto(520.67313622,105.13623126)(520.56734195,104.92667722)(520.56734195,104.66626055)
\curveto(520.56734195,104.38142983)(520.70772281,104.15356524)(520.98848453,103.9826668)
\curveto(521.15124494,103.88094154)(521.39131656,103.79142332)(521.70869937,103.71411212)
\lineto(522.43501773,103.53711016)
\curveto(523.22440575,103.34586667)(523.75337711,103.1607267)(524.02193179,102.98169024)
\curveto(524.44917789,102.70092852)(524.66280093,102.25944089)(524.66280093,101.65722735)
\curveto(524.66280093,101.07535886)(524.44103987,100.57283608)(523.99751773,100.14965899)
\curveto(523.55806461,99.72648191)(522.88667789,99.51489337)(521.98335757,99.51489337)
\curveto(521.01086408,99.51489337)(520.32116682,99.73461993)(519.91426578,100.17407305)
\curveto(519.51143375,100.61759519)(519.29577619,101.16487709)(519.26729312,101.81591876)
\lineto(520.32320132,101.81591876)
\closepath
\moveto(521.94673648,106.46069415)
\lineto(521.94673648,106.46069415)
\closepath
}
}
{
\newrgbcolor{curcolor}{0 0 0}
\pscustom[linewidth=1,linecolor=curcolor]
{
\newpath
\moveto(509.216019,149.76513778)
\lineto(509.216019,119.76513778)
}
}
{
\newrgbcolor{curcolor}{0 0 0}
\pscustom[linestyle=none,fillstyle=solid,fillcolor=curcolor]
{
\newpath
\moveto(509.216019,145.76513778)
\lineto(511.216019,143.76513778)
\lineto(509.216019,150.76513778)
\lineto(507.216019,143.76513778)
\lineto(509.216019,145.76513778)
\closepath
}
}
{
\newrgbcolor{curcolor}{0 0 0}
\pscustom[linewidth=0.5,linecolor=curcolor]
{
\newpath
\moveto(509.216019,145.76513778)
\lineto(511.216019,143.76513778)
\lineto(509.216019,150.76513778)
\lineto(507.216019,143.76513778)
\lineto(509.216019,145.76513778)
\closepath
}
}
{
\newrgbcolor{curcolor}{0 0 0}
\pscustom[linestyle=none,fillstyle=solid,fillcolor=curcolor]
{
\newpath
\moveto(509.216019,123.76513778)
\lineto(507.216019,125.76513778)
\lineto(509.216019,118.76513778)
\lineto(511.216019,125.76513778)
\lineto(509.216019,123.76513778)
\closepath
}
}
{
\newrgbcolor{curcolor}{0 0 0}
\pscustom[linewidth=0.5,linecolor=curcolor]
{
\newpath
\moveto(509.216019,123.76513778)
\lineto(507.216019,125.76513778)
\lineto(509.216019,118.76513778)
\lineto(511.216019,125.76513778)
\lineto(509.216019,123.76513778)
\closepath
}
}
{
\newrgbcolor{curcolor}{0 0 0}
\pscustom[linewidth=1,linecolor=curcolor]
{
\newpath
\moveto(359.550069,180.33396778)
\lineto(329.550069,180.33396778)
}
}
{
\newrgbcolor{curcolor}{0 0 0}
\pscustom[linestyle=none,fillstyle=solid,fillcolor=curcolor]
{
\newpath
\moveto(333.550069,180.33396778)
\lineto(335.550069,182.33396778)
\lineto(328.550069,180.33396778)
\lineto(335.550069,178.33396778)
\lineto(333.550069,180.33396778)
\closepath
}
}
{
\newrgbcolor{curcolor}{0 0 0}
\pscustom[linewidth=0.5,linecolor=curcolor]
{
\newpath
\moveto(333.550069,180.33396778)
\lineto(335.550069,182.33396778)
\lineto(328.550069,180.33396778)
\lineto(335.550069,178.33396778)
\lineto(333.550069,180.33396778)
\closepath
}
}
{
\newrgbcolor{curcolor}{0 0 0}
\pscustom[linewidth=5,linecolor=curcolor]
{
\newpath
\moveto(399.550069,141.77121778)
\lineto(399.550069,31.77121778)
}
}
{
\newrgbcolor{curcolor}{0 0 0}
\pscustom[linestyle=none,fillstyle=solid,fillcolor=curcolor]
{
\newpath
\moveto(399.550069,121.77121778)
\lineto(409.550069,111.77121778)
\lineto(399.550069,146.77121778)
\lineto(389.550069,111.77121778)
\lineto(399.550069,121.77121778)
\closepath
}
}
{
\newrgbcolor{curcolor}{0 0 0}
\pscustom[linewidth=2.5,linecolor=curcolor]
{
\newpath
\moveto(399.550069,121.77121778)
\lineto(409.550069,111.77121778)
\lineto(399.550069,146.77121778)
\lineto(389.550069,111.77121778)
\lineto(399.550069,121.77121778)
\closepath
}
}
{
\newrgbcolor{curcolor}{0 0 0}
\pscustom[linestyle=none,fillstyle=solid,fillcolor=curcolor]
{
\newpath
\moveto(399.550069,51.77121778)
\lineto(389.550069,61.77121778)
\lineto(399.550069,26.77121778)
\lineto(409.550069,61.77121778)
\lineto(399.550069,51.77121778)
\closepath
}
}
{
\newrgbcolor{curcolor}{0 0 0}
\pscustom[linewidth=2.5,linecolor=curcolor]
{
\newpath
\moveto(399.550069,51.77121778)
\lineto(389.550069,61.77121778)
\lineto(399.550069,26.77121778)
\lineto(409.550069,61.77121778)
\lineto(399.550069,51.77121778)
\closepath
}
}
{
\newrgbcolor{curcolor}{0 0 0}
\pscustom[linestyle=none,fillstyle=solid,fillcolor=curcolor]
{
\newpath
\moveto(371.08815494,13.24511798)
\lineto(380.50124088,13.24511798)
\lineto(380.50124088,11.66308673)
\lineto(372.79323306,11.66308673)
\lineto(372.79323306,7.74316485)
\lineto(379.92116275,7.74316485)
\lineto(379.92116275,6.24902423)
\lineto(372.79323306,6.24902423)
\lineto(372.79323306,1.8720711)
\lineto(380.63307681,1.8720711)
\lineto(380.63307681,0.33398516)
\lineto(371.08815494,0.33398516)
\lineto(371.08815494,13.24511798)
\closepath
\moveto(375.86061588,13.24511798)
\lineto(375.86061588,13.24511798)
\closepath
}
}
{
\newrgbcolor{curcolor}{0 0 0}
\pscustom[linestyle=none,fillstyle=solid,fillcolor=curcolor]
{
\newpath
\moveto(383.050069,12.37500079)
\lineto(384.64967838,12.37500079)
\lineto(384.64967838,9.7470711)
\lineto(386.15260806,9.7470711)
\lineto(386.15260806,8.45507891)
\lineto(384.64967838,8.45507891)
\lineto(384.64967838,2.31152423)
\curveto(384.64967838,1.98339923)(384.7610065,1.76367266)(384.98366275,1.65234454)
\curveto(385.10670963,1.58789141)(385.31178775,1.55566485)(385.59889713,1.55566485)
\lineto(385.84499088,1.55566485)
\curveto(385.9328815,1.56152423)(386.03542056,1.57031329)(386.15260806,1.58203204)
\lineto(386.15260806,0.33398516)
\curveto(385.97096744,0.28125079)(385.78053775,0.24316485)(385.581319,0.21972735)
\curveto(385.38795963,0.19628985)(385.17702213,0.1845711)(384.9485065,0.1845711)
\curveto(384.21022525,0.1845711)(383.70924869,0.3720711)(383.44557681,0.7470711)
\curveto(383.18190494,1.12793048)(383.050069,1.62011798)(383.050069,2.2236336)
\lineto(383.050069,8.45507891)
\lineto(381.77565494,8.45507891)
\lineto(381.77565494,9.7470711)
\lineto(383.050069,9.7470711)
\lineto(383.050069,12.37500079)
\closepath
}
}
{
\newrgbcolor{curcolor}{0 0 0}
\pscustom[linestyle=none,fillstyle=solid,fillcolor=curcolor]
{
\newpath
\moveto(387.7610065,13.28906329)
\lineto(389.34303775,13.28906329)
\lineto(389.34303775,8.47265704)
\curveto(389.71803775,8.94726641)(390.05495181,9.28125079)(390.35377994,9.47461016)
\curveto(390.86354556,9.80859454)(391.49928775,9.97558673)(392.2610065,9.97558673)
\curveto(393.62624088,9.97558673)(394.55202213,9.49804766)(395.03835025,8.54296954)
\curveto(395.30202213,8.02148516)(395.43385806,7.29785235)(395.43385806,6.3720711)
\lineto(395.43385806,0.33398516)
\lineto(393.8078815,0.33398516)
\lineto(393.8078815,6.26660235)
\curveto(393.8078815,6.9580086)(393.71999088,7.46484454)(393.54420963,7.78711016)
\curveto(393.25710025,8.30273516)(392.71803775,8.56054766)(391.92702213,8.56054766)
\curveto(391.27077213,8.56054766)(390.67604556,8.33496173)(390.14284244,7.88378985)
\curveto(389.60963931,7.43261798)(389.34303775,6.58007891)(389.34303775,5.32617266)
\lineto(389.34303775,0.33398516)
\lineto(387.7610065,0.33398516)
\lineto(387.7610065,13.28906329)
\closepath
}
}
{
\newrgbcolor{curcolor}{0 0 0}
\pscustom[linestyle=none,fillstyle=solid,fillcolor=curcolor]
{
\newpath
\moveto(401.70045963,9.9580086)
\curveto(402.36842838,9.9580086)(403.01588931,9.79980548)(403.64284244,9.48339923)
\curveto(404.26979556,9.17285235)(404.74733463,8.76855548)(405.07545963,8.2705086)
\curveto(405.39186588,7.79589923)(405.60280338,7.24218829)(405.70827213,6.60937579)
\curveto(405.80202213,6.17578204)(405.84889713,5.48437579)(405.84889713,4.53515704)
\lineto(398.94948306,4.53515704)
\curveto(398.97877994,3.58007891)(399.20436588,2.81250079)(399.62624088,2.23242266)
\curveto(400.04811588,1.65820391)(400.70143619,1.37109454)(401.58620181,1.37109454)
\curveto(402.41237369,1.37109454)(403.07155338,1.64355548)(403.56374088,2.18847735)
\curveto(403.84499088,2.5048836)(404.04420963,2.87109454)(404.16139713,3.28711016)
\lineto(405.71706119,3.28711016)
\curveto(405.67604556,2.94140704)(405.53835025,2.55468829)(405.30397525,2.12695391)
\curveto(405.07545963,1.70507891)(404.81764713,1.35937579)(404.53053775,1.08984454)
\curveto(404.050069,0.62109454)(403.45534244,0.30468829)(402.74635806,0.14062579)
\curveto(402.36549869,0.04687579)(401.93483463,0.00000079)(401.45436588,0.00000079)
\curveto(400.28249088,0.00000079)(399.28932681,0.42480548)(398.47487369,1.27441485)
\curveto(397.66042056,2.1298836)(397.253194,3.3251961)(397.253194,4.86035235)
\curveto(397.253194,6.3720711)(397.66335025,7.59961016)(398.48366275,8.54296954)
\curveto(399.30397525,9.48632891)(400.37624088,9.9580086)(401.70045963,9.9580086)
\closepath
\moveto(404.22292056,5.79199298)
\curveto(404.15846744,6.47753985)(404.00905338,7.02539141)(403.77467838,7.43554766)
\curveto(403.34108463,8.19726641)(402.61745181,8.57812579)(401.60377994,8.57812579)
\curveto(400.87721744,8.57812579)(400.26784244,8.31445391)(399.77565494,7.78711016)
\curveto(399.28346744,7.26562579)(399.02272525,6.60058673)(398.99342838,5.79199298)
\lineto(404.22292056,5.79199298)
\closepath
\moveto(401.55104556,9.97558673)
\lineto(401.55104556,9.97558673)
\closepath
}
}
{
\newrgbcolor{curcolor}{0 0 0}
\pscustom[linestyle=none,fillstyle=solid,fillcolor=curcolor]
{
\newpath
\moveto(407.84401431,9.7470711)
\lineto(409.346944,9.7470711)
\lineto(409.346944,8.12109454)
\curveto(409.46999088,8.43750079)(409.77174869,8.82128985)(410.25221744,9.27246173)
\curveto(410.73268619,9.72949298)(411.28639713,9.9580086)(411.91335025,9.9580086)
\curveto(411.94264713,9.9580086)(411.99245181,9.95507891)(412.06276431,9.94921954)
\curveto(412.13307681,9.94336016)(412.253194,9.93164141)(412.42311588,9.91406329)
\lineto(412.42311588,8.24414141)
\curveto(412.32936588,8.26171954)(412.24147525,8.27343829)(412.159444,8.27929766)
\curveto(412.08327213,8.28515704)(411.99831119,8.28808673)(411.90456119,8.28808673)
\curveto(411.10768619,8.28808673)(410.4953815,8.03027423)(410.06764713,7.51464923)
\curveto(409.63991275,7.0048836)(409.42604556,6.41601641)(409.42604556,5.74804766)
\lineto(409.42604556,0.33398516)
\lineto(407.84401431,0.33398516)
\lineto(407.84401431,9.7470711)
\closepath
}
}
{
\newrgbcolor{curcolor}{0 0 0}
\pscustom[linestyle=none,fillstyle=solid,fillcolor=curcolor]
{
\newpath
\moveto(413.81178775,9.7470711)
\lineto(415.31471744,9.7470711)
\lineto(415.31471744,8.4111336)
\curveto(415.76002994,8.96191485)(416.23170963,9.35742266)(416.7297565,9.59765704)
\curveto(417.22780338,9.83789141)(417.78151431,9.9580086)(418.39088931,9.9580086)
\curveto(419.72682681,9.9580086)(420.62917056,9.49218829)(421.09792056,8.56054766)
\curveto(421.35573306,8.05078204)(421.48463931,7.32128985)(421.48463931,6.3720711)
\lineto(421.48463931,0.33398516)
\lineto(419.87624088,0.33398516)
\lineto(419.87624088,6.26660235)
\curveto(419.87624088,6.8408211)(419.79127994,7.30371173)(419.62135806,7.65527423)
\curveto(419.34010806,8.24121173)(418.83034244,8.53418048)(418.09206119,8.53418048)
\curveto(417.71706119,8.53418048)(417.409444,8.49609454)(417.16920963,8.41992266)
\curveto(416.73561588,8.29101641)(416.3547565,8.03320391)(416.0266315,7.64648516)
\curveto(415.76295963,7.33593829)(415.59010806,7.01367266)(415.50807681,6.67968829)
\curveto(415.43190494,6.35156329)(415.393819,5.8798836)(415.393819,5.26464923)
\lineto(415.393819,0.33398516)
\lineto(413.81178775,0.33398516)
\lineto(413.81178775,9.7470711)
\closepath
\moveto(417.52956119,9.97558673)
\lineto(417.52956119,9.97558673)
\closepath
}
}
{
\newrgbcolor{curcolor}{0 0 0}
\pscustom[linestyle=none,fillstyle=solid,fillcolor=curcolor]
{
\newpath
\moveto(427.75124088,9.9580086)
\curveto(428.41920963,9.9580086)(429.06667056,9.79980548)(429.69362369,9.48339923)
\curveto(430.32057681,9.17285235)(430.79811588,8.76855548)(431.12624088,8.2705086)
\curveto(431.44264713,7.79589923)(431.65358463,7.24218829)(431.75905338,6.60937579)
\curveto(431.85280338,6.17578204)(431.89967838,5.48437579)(431.89967838,4.53515704)
\lineto(425.00026431,4.53515704)
\curveto(425.02956119,3.58007891)(425.25514713,2.81250079)(425.67702213,2.23242266)
\curveto(426.09889713,1.65820391)(426.75221744,1.37109454)(427.63698306,1.37109454)
\curveto(428.46315494,1.37109454)(429.12233463,1.64355548)(429.61452213,2.18847735)
\curveto(429.89577213,2.5048836)(430.09499088,2.87109454)(430.21217838,3.28711016)
\lineto(431.76784244,3.28711016)
\curveto(431.72682681,2.94140704)(431.5891315,2.55468829)(431.3547565,2.12695391)
\curveto(431.12624088,1.70507891)(430.86842838,1.35937579)(430.581319,1.08984454)
\curveto(430.10085025,0.62109454)(429.50612369,0.30468829)(428.79713931,0.14062579)
\curveto(428.41627994,0.04687579)(427.98561588,0.00000079)(427.50514713,0.00000079)
\curveto(426.33327213,0.00000079)(425.34010806,0.42480548)(424.52565494,1.27441485)
\curveto(423.71120181,2.1298836)(423.30397525,3.3251961)(423.30397525,4.86035235)
\curveto(423.30397525,6.3720711)(423.7141315,7.59961016)(424.534444,8.54296954)
\curveto(425.3547565,9.48632891)(426.42702213,9.9580086)(427.75124088,9.9580086)
\closepath
\moveto(430.27370181,5.79199298)
\curveto(430.20924869,6.47753985)(430.05983463,7.02539141)(429.82545963,7.43554766)
\curveto(429.39186588,8.19726641)(428.66823306,8.57812579)(427.65456119,8.57812579)
\curveto(426.92799869,8.57812579)(426.31862369,8.31445391)(425.82643619,7.78711016)
\curveto(425.33424869,7.26562579)(425.0735065,6.60058673)(425.04420963,5.79199298)
\lineto(430.27370181,5.79199298)
\closepath
\moveto(427.60182681,9.97558673)
\lineto(427.60182681,9.97558673)
\closepath
}
}
{
\newrgbcolor{curcolor}{0 0 0}
\pscustom[linestyle=none,fillstyle=solid,fillcolor=curcolor]
{
\newpath
\moveto(434.1672565,12.37500079)
\lineto(435.76686588,12.37500079)
\lineto(435.76686588,9.7470711)
\lineto(437.26979556,9.7470711)
\lineto(437.26979556,8.45507891)
\lineto(435.76686588,8.45507891)
\lineto(435.76686588,2.31152423)
\curveto(435.76686588,1.98339923)(435.878194,1.76367266)(436.10085025,1.65234454)
\curveto(436.22389713,1.58789141)(436.42897525,1.55566485)(436.71608463,1.55566485)
\lineto(436.96217838,1.55566485)
\curveto(437.050069,1.56152423)(437.15260806,1.57031329)(437.26979556,1.58203204)
\lineto(437.26979556,0.33398516)
\curveto(437.08815494,0.28125079)(436.89772525,0.24316485)(436.6985065,0.21972735)
\curveto(436.50514713,0.19628985)(436.29420963,0.1845711)(436.065694,0.1845711)
\curveto(435.32741275,0.1845711)(434.82643619,0.3720711)(434.56276431,0.7470711)
\curveto(434.29909244,1.12793048)(434.1672565,1.62011798)(434.1672565,2.2236336)
\lineto(434.1672565,8.45507891)
\lineto(432.89284244,8.45507891)
\lineto(432.89284244,9.7470711)
\lineto(434.1672565,9.7470711)
\lineto(434.1672565,12.37500079)
\closepath
}
}
{
\newrgbcolor{curcolor}{0 0 0}
\pscustom[linestyle=none,fillstyle=solid,fillcolor=curcolor]
{
\newpath
\moveto(323.87135806,125.50976641)
\curveto(324.38405338,125.50976641)(324.7828164,125.5809741)(325.06764713,125.72338946)
\curveto(325.51523827,125.94718503)(325.73903384,126.35001707)(325.73903384,126.93188555)
\curveto(325.73903384,127.51782305)(325.50099673,127.91251707)(325.02492252,128.11596759)
\curveto(324.75636783,128.22989988)(324.35760481,128.28686602)(323.82863345,128.28686602)
\lineto(321.66188541,128.28686602)
\lineto(321.66188541,125.50976641)
\lineto(323.87135806,125.50976641)
\closepath
\moveto(324.28029361,121.37158282)
\curveto(325.02492252,121.37158282)(325.55592838,121.58724037)(325.87331119,122.01855548)
\curveto(326.0726927,122.29117917)(326.17238345,122.62076902)(326.17238345,123.00732501)
\curveto(326.17238345,123.65836667)(325.88144921,124.10188881)(325.29958072,124.33789141)
\curveto(324.99033593,124.46403074)(324.58140038,124.5271004)(324.07277408,124.5271004)
\lineto(321.66188541,124.5271004)
\lineto(321.66188541,121.37158282)
\lineto(324.28029361,121.37158282)
\closepath
\moveto(320.47169986,129.30004962)
\lineto(324.32301822,129.30004962)
\curveto(325.37282291,129.30004962)(326.11948632,128.98673582)(326.56300845,128.36010821)
\curveto(326.82342512,127.98982826)(326.95363345,127.56258217)(326.95363345,127.07836993)
\curveto(326.95363345,126.51277748)(326.79290754,126.04891029)(326.47145572,125.68676837)
\curveto(326.30462629,125.49552488)(326.06455468,125.32055743)(325.75124088,125.16186602)
\curveto(326.21103905,124.98689858)(326.55487043,124.78955157)(326.78273502,124.56982501)
\curveto(327.18556705,124.17920001)(327.38698306,123.64005613)(327.38698306,122.95239337)
\curveto(327.38698306,122.37459389)(327.2059121,121.85172605)(326.84377017,121.38378985)
\curveto(326.30259179,120.68392006)(325.44199608,120.33398516)(324.26198306,120.33398516)
\lineto(320.47169986,120.33398516)
\lineto(320.47169986,129.30004962)
\closepath
}
}
{
\newrgbcolor{curcolor}{0 0 0}
\pscustom[linestyle=none,fillstyle=solid,fillcolor=curcolor]
{
\newpath
\moveto(329.54762759,122.07348712)
\curveto(329.54762759,121.7561043)(329.66359439,121.50586016)(329.89552798,121.3227547)
\curveto(330.12746158,121.13964923)(330.40211978,121.04809649)(330.71950259,121.04809649)
\curveto(331.10605858,121.04809649)(331.48040754,121.13761472)(331.84254947,121.31665118)
\curveto(332.45290103,121.61368894)(332.75807681,122.09993569)(332.75807681,122.77539141)
\lineto(332.75807681,123.66040118)
\curveto(332.62379947,123.57495196)(332.45086653,123.50374428)(332.23927798,123.44677813)
\curveto(332.02768944,123.38981199)(331.82016991,123.34912188)(331.61671939,123.32470782)
\lineto(330.95143619,123.2392586)
\curveto(330.55267317,123.18636147)(330.2536009,123.10294675)(330.05421939,122.98901446)
\curveto(329.71649153,122.79777097)(329.54762759,122.49259519)(329.54762759,122.07348712)
\closepath
\moveto(332.20876041,124.2951668)
\curveto(332.46103905,124.32771889)(332.62990298,124.43351316)(332.7153522,124.61254962)
\curveto(332.76418033,124.71020587)(332.78859439,124.85058673)(332.78859439,125.0336922)
\curveto(332.78859439,125.40804115)(332.65431705,125.67863035)(332.38576236,125.84545977)
\curveto(332.12127668,126.01635821)(331.74082421,126.10180743)(331.24440494,126.10180743)
\curveto(330.67067447,126.10180743)(330.26377343,125.94718503)(330.02370181,125.63794024)
\curveto(329.88942447,125.4670418)(329.80194074,125.21272865)(329.76125064,124.87500079)
\lineto(328.73586002,124.87500079)
\curveto(328.75620507,125.68066485)(329.01662173,126.24015378)(329.51711002,126.55346759)
\curveto(330.02166731,126.8708504)(330.6055703,127.0295418)(331.268819,127.0295418)
\curveto(332.03786197,127.0295418)(332.66245507,126.88305743)(333.1425983,126.59008868)
\curveto(333.61867252,126.29711993)(333.85670963,125.84139076)(333.85670963,125.22290118)
\lineto(333.85670963,121.45703204)
\curveto(333.85670963,121.34309975)(333.87908918,121.25154701)(333.9238483,121.18237384)
\curveto(333.97267642,121.11320066)(334.07236718,121.07861407)(334.22292056,121.07861407)
\curveto(334.27174869,121.07861407)(334.32668033,121.08064858)(334.38771548,121.08471759)
\curveto(334.44875064,121.09285561)(334.51385481,121.10302813)(334.58302798,121.11523516)
\lineto(334.58302798,120.30346759)
\curveto(334.41212955,120.25463946)(334.28192121,120.22412188)(334.19240298,120.21191485)
\curveto(334.10288476,120.19970782)(333.98081444,120.1936043)(333.82619205,120.1936043)
\curveto(333.44777408,120.1936043)(333.17311588,120.32788165)(333.00221744,120.59643634)
\curveto(332.91269921,120.7388517)(332.84962955,120.94026772)(332.81300845,121.20068438)
\curveto(332.58921288,120.90771563)(332.26776106,120.65340248)(331.84865298,120.43774493)
\curveto(331.42954491,120.22208738)(330.96771223,120.1142586)(330.46315494,120.1142586)
\curveto(329.85687239,120.1142586)(329.36045311,120.29736407)(328.97389713,120.66357501)
\curveto(328.59141015,121.03385496)(328.40016666,121.49568764)(328.40016666,122.04907305)
\curveto(328.40016666,122.65535561)(328.58937564,123.12532631)(328.96779361,123.45898516)
\curveto(329.34621158,123.79264402)(329.84263085,123.99812904)(330.45705142,124.07544024)
\lineto(332.20876041,124.2951668)
\closepath
\moveto(331.29933658,127.0295418)
\lineto(331.29933658,127.0295418)
\closepath
}
}
{
\newrgbcolor{curcolor}{0 0 0}
\pscustom[linestyle=none,fillstyle=solid,fillcolor=curcolor]
{
\newpath
\moveto(338.1841022,127.06005938)
\curveto(338.92059309,127.06005938)(339.51873762,126.88102292)(339.9785358,126.52295001)
\curveto(340.44240298,126.16487709)(340.7211302,125.54842201)(340.81471744,124.67358477)
\lineto(339.7466022,124.67358477)
\curveto(339.68149804,125.0764168)(339.53297916,125.41007566)(339.30104556,125.67456134)
\curveto(339.06911197,125.94311602)(338.69679752,126.07739337)(338.1841022,126.07739337)
\curveto(337.48423241,126.07739337)(336.98374413,125.73559649)(336.68263736,125.05200274)
\curveto(336.48732486,124.60848061)(336.38966861,124.06119871)(336.38966861,123.41015704)
\curveto(336.38966861,122.75504636)(336.52801496,122.20369545)(336.80470767,121.7561043)
\curveto(337.08140038,121.30851316)(337.51678449,121.08471759)(338.11086002,121.08471759)
\curveto(338.56658918,121.08471759)(338.9266966,121.22306394)(339.19118228,121.49975665)
\curveto(339.45973697,121.78051837)(339.64487694,122.16300535)(339.7466022,122.64721759)
\lineto(340.81471744,122.64721759)
\curveto(340.69264713,121.78051837)(340.38747134,121.14575274)(339.89919009,120.74292071)
\curveto(339.41090884,120.34415769)(338.78631574,120.14477618)(338.0254108,120.14477618)
\curveto(337.17091861,120.14477618)(336.48935936,120.45605548)(335.98073306,121.07861407)
\curveto(335.47210676,121.70524167)(335.21779361,122.48649167)(335.21779361,123.42236407)
\curveto(335.21779361,124.56982501)(335.49652082,125.46297279)(336.05397525,126.10180743)
\curveto(336.61142968,126.74064207)(337.32147199,127.06005938)(338.1841022,127.06005938)
\closepath
\moveto(338.01320377,127.0295418)
\lineto(338.01320377,127.0295418)
\closepath
}
}
{
\newrgbcolor{curcolor}{0 0 0}
\pscustom[linestyle=none,fillstyle=solid,fillcolor=curcolor]
{
\newpath
\moveto(341.88893619,129.30004962)
\lineto(342.94484439,129.30004962)
\lineto(342.94484439,124.09375079)
\lineto(345.76466861,126.8708504)
\lineto(347.1684772,126.8708504)
\lineto(344.6660358,124.42334063)
\lineto(347.30885806,120.33398516)
\lineto(345.90504947,120.33398516)
\lineto(343.86647525,123.6298836)
\lineto(342.94484439,122.78759845)
\lineto(342.94484439,120.33398516)
\lineto(341.88893619,120.33398516)
\lineto(341.88893619,129.30004962)
\closepath
}
}
{
\newrgbcolor{curcolor}{0 0 0}
\pscustom[linestyle=none,fillstyle=solid,fillcolor=curcolor]
{
\newpath
\moveto(324.28029361,117.04419024)
\curveto(325.41554752,117.04419024)(326.29648827,116.74511798)(326.92311588,116.14697345)
\curveto(327.54974348,115.54882891)(327.89764387,114.86930417)(327.96681705,114.10839923)
\lineto(326.78273502,114.10839923)
\curveto(326.64845767,114.68619871)(326.37990298,115.14396238)(325.97707095,115.48169024)
\curveto(325.57830793,115.81941811)(325.01678449,115.98828204)(324.29250064,115.98828204)
\curveto(323.40952538,115.98828204)(322.69541405,115.67700274)(322.15016666,115.05444415)
\curveto(321.60898827,114.43595457)(321.33839908,113.48584063)(321.33839908,112.20410235)
\curveto(321.33839908,111.15429766)(321.5825397,110.30183998)(322.07082095,109.6467293)
\curveto(322.56317121,108.99568764)(323.29559309,108.6701668)(324.26808658,108.6701668)
\curveto(325.16326887,108.6701668)(325.84482811,109.01399819)(326.31276431,109.70166095)
\curveto(326.56097395,110.06380287)(326.74611392,110.53987709)(326.86818423,111.1298836)
\lineto(328.05226627,111.1298836)
\curveto(327.94647199,110.18587319)(327.5965371,109.39445066)(327.00246158,108.75561602)
\curveto(326.29038476,107.98657305)(325.3300983,107.60205157)(324.1216022,107.60205157)
\curveto(323.07993554,107.60205157)(322.2050983,107.91739988)(321.49709048,108.54809649)
\curveto(320.5652871,109.38224363)(320.09938541,110.67008542)(320.09938541,112.41162188)
\curveto(320.09938541,113.73405027)(320.4493203,114.81844154)(321.14919009,115.66479571)
\curveto(321.90602603,116.58439207)(322.9497272,117.04419024)(324.28029361,117.04419024)
\closepath
\moveto(324.03615298,117.04419024)
\lineto(324.03615298,117.04419024)
\closepath
}
}
{
\newrgbcolor{curcolor}{0 0 0}
\pscustom[linestyle=none,fillstyle=solid,fillcolor=curcolor]
{
\newpath
\moveto(329.38893619,116.8305672)
\lineto(330.487569,116.8305672)
\lineto(330.487569,113.48584063)
\curveto(330.74798567,113.81543048)(330.98195377,114.04736407)(331.1894733,114.18164141)
\curveto(331.5434772,114.41357501)(331.98496483,114.5295418)(332.51393619,114.5295418)
\curveto(333.46201561,114.5295418)(334.10491926,114.19791746)(334.44264713,113.53466876)
\curveto(334.62575259,113.17252683)(334.71730533,112.67000404)(334.71730533,112.0271004)
\lineto(334.71730533,107.83398516)
\lineto(333.58815494,107.83398516)
\lineto(333.58815494,111.95385821)
\curveto(333.58815494,112.43400144)(333.52711978,112.78597084)(333.40504947,113.00976641)
\curveto(333.20566796,113.36783933)(332.831319,113.54687579)(332.28200259,113.54687579)
\curveto(331.82627343,113.54687579)(331.41326887,113.39021889)(331.04298892,113.07690509)
\curveto(330.67270897,112.76359128)(330.487569,112.17155027)(330.487569,111.30078204)
\lineto(330.487569,107.83398516)
\lineto(329.38893619,107.83398516)
\lineto(329.38893619,116.8305672)
\closepath
}
}
{
\newrgbcolor{curcolor}{0 0 0}
\pscustom[linestyle=none,fillstyle=solid,fillcolor=curcolor]
{
\newpath
\moveto(337.18922916,109.57348712)
\curveto(337.18922916,109.2561043)(337.30519595,109.00586016)(337.53712955,108.8227547)
\curveto(337.76906314,108.63964923)(338.04372134,108.54809649)(338.36110416,108.54809649)
\curveto(338.74766015,108.54809649)(339.1220091,108.63761472)(339.48415103,108.81665118)
\curveto(340.09450259,109.11368894)(340.39967838,109.59993569)(340.39967838,110.27539141)
\lineto(340.39967838,111.16040118)
\curveto(340.26540103,111.07495196)(340.09246809,111.00374428)(339.88087955,110.94677813)
\curveto(339.66929101,110.88981199)(339.46177147,110.84912188)(339.25832095,110.82470782)
\lineto(338.59303775,110.7392586)
\curveto(338.19427473,110.68636147)(337.89520246,110.60294675)(337.69582095,110.48901446)
\curveto(337.35809309,110.29777097)(337.18922916,109.99259519)(337.18922916,109.57348712)
\closepath
\moveto(339.85036197,111.7951668)
\curveto(340.10264061,111.82771889)(340.27150455,111.93351316)(340.35695377,112.11254962)
\curveto(340.40578189,112.21020587)(340.43019595,112.35058673)(340.43019595,112.5336922)
\curveto(340.43019595,112.90804115)(340.29591861,113.17863035)(340.02736392,113.34545977)
\curveto(339.76287824,113.51635821)(339.38242577,113.60180743)(338.8860065,113.60180743)
\curveto(338.31227603,113.60180743)(337.90537499,113.44718503)(337.66530338,113.13794024)
\curveto(337.53102603,112.9670418)(337.44354231,112.71272865)(337.4028522,112.37500079)
\lineto(336.37746158,112.37500079)
\curveto(336.39780663,113.18066485)(336.6582233,113.74015378)(337.15871158,114.05346759)
\curveto(337.66326887,114.3708504)(338.24717186,114.5295418)(338.91042056,114.5295418)
\curveto(339.67946353,114.5295418)(340.30405663,114.38305743)(340.78419986,114.09008868)
\curveto(341.26027408,113.79711993)(341.49831119,113.34139076)(341.49831119,112.72290118)
\lineto(341.49831119,108.95703204)
\curveto(341.49831119,108.84309975)(341.52069074,108.75154701)(341.56544986,108.68237384)
\curveto(341.61427798,108.61320066)(341.71396874,108.57861407)(341.86452213,108.57861407)
\curveto(341.91335025,108.57861407)(341.96828189,108.58064858)(342.02931705,108.58471759)
\curveto(342.0903522,108.59285561)(342.15545637,108.60302813)(342.22462955,108.61523516)
\lineto(342.22462955,107.80346759)
\curveto(342.05373111,107.75463946)(341.92352278,107.72412188)(341.83400455,107.71191485)
\curveto(341.74448632,107.69970782)(341.62241601,107.6936043)(341.46779361,107.6936043)
\curveto(341.08937564,107.6936043)(340.81471744,107.82788165)(340.643819,108.09643634)
\curveto(340.55430077,108.2388517)(340.49123111,108.44026772)(340.45461002,108.70068438)
\curveto(340.23081444,108.40771563)(339.90936262,108.15340248)(339.49025455,107.93774493)
\curveto(339.07114647,107.72208738)(338.60931379,107.6142586)(338.1047565,107.6142586)
\curveto(337.49847395,107.6142586)(337.00205468,107.79736407)(336.61549869,108.16357501)
\curveto(336.23301171,108.53385496)(336.04176822,108.99568764)(336.04176822,109.54907305)
\curveto(336.04176822,110.15535561)(336.2309772,110.62532631)(336.60939517,110.95898516)
\curveto(336.98781314,111.29264402)(337.48423241,111.49812904)(338.09865298,111.57544024)
\lineto(339.85036197,111.7951668)
\closepath
\moveto(338.94093814,114.5295418)
\lineto(338.94093814,114.5295418)
\closepath
}
}
{
\newrgbcolor{curcolor}{0 0 0}
\pscustom[linestyle=none,fillstyle=solid,fillcolor=curcolor]
{
\newpath
\moveto(343.30495181,114.3708504)
\lineto(344.34865298,114.3708504)
\lineto(344.34865298,113.44311602)
\curveto(344.65789778,113.825603)(344.98545311,114.10026121)(345.331319,114.26709063)
\curveto(345.67718489,114.43392006)(346.06170637,114.51733477)(346.48488345,114.51733477)
\curveto(347.41261783,114.51733477)(348.03924543,114.19384845)(348.36476627,113.54687579)
\curveto(348.54380272,113.19287188)(348.63332095,112.68628009)(348.63332095,112.0271004)
\lineto(348.63332095,107.83398516)
\lineto(347.51637759,107.83398516)
\lineto(347.51637759,111.95385821)
\curveto(347.51637759,112.35262123)(347.45737694,112.67407305)(347.33937564,112.91821368)
\curveto(347.14406314,113.32511472)(346.79005923,113.52856524)(346.27736392,113.52856524)
\curveto(346.01694726,113.52856524)(345.80332421,113.50211667)(345.63649478,113.44921954)
\curveto(345.33538801,113.35970131)(345.07090233,113.18066485)(344.84303775,112.91211016)
\curveto(344.65993228,112.69645261)(344.53989647,112.47265704)(344.48293033,112.24072345)
\curveto(344.43003319,112.01285886)(344.40358463,111.68530352)(344.40358463,111.25805743)
\lineto(344.40358463,107.83398516)
\lineto(343.30495181,107.83398516)
\lineto(343.30495181,114.3708504)
\closepath
\moveto(345.88673892,114.5295418)
\lineto(345.88673892,114.5295418)
\closepath
}
}
{
\newrgbcolor{curcolor}{0 0 0}
\pscustom[linestyle=none,fillstyle=solid,fillcolor=curcolor]
{
\newpath
\moveto(350.26295963,114.3708504)
\lineto(351.3066608,114.3708504)
\lineto(351.3066608,113.44311602)
\curveto(351.61590559,113.825603)(351.94346093,114.10026121)(352.28932681,114.26709063)
\curveto(352.6351927,114.43392006)(353.01971418,114.51733477)(353.44289127,114.51733477)
\curveto(354.37062564,114.51733477)(354.99725324,114.19384845)(355.32277408,113.54687579)
\curveto(355.50181054,113.19287188)(355.59132877,112.68628009)(355.59132877,112.0271004)
\lineto(355.59132877,107.83398516)
\lineto(354.47438541,107.83398516)
\lineto(354.47438541,111.95385821)
\curveto(354.47438541,112.35262123)(354.41538476,112.67407305)(354.29738345,112.91821368)
\curveto(354.10207095,113.32511472)(353.74806705,113.52856524)(353.23537173,113.52856524)
\curveto(352.97495507,113.52856524)(352.76133202,113.50211667)(352.59450259,113.44921954)
\curveto(352.29339582,113.35970131)(352.02891015,113.18066485)(351.80104556,112.91211016)
\curveto(351.61794009,112.69645261)(351.49790429,112.47265704)(351.44093814,112.24072345)
\curveto(351.38804101,112.01285886)(351.36159244,111.68530352)(351.36159244,111.25805743)
\lineto(351.36159244,107.83398516)
\lineto(350.26295963,107.83398516)
\lineto(350.26295963,114.3708504)
\closepath
\moveto(352.84474673,114.5295418)
\lineto(352.84474673,114.5295418)
\closepath
}
}
{
\newrgbcolor{curcolor}{0 0 0}
\pscustom[linestyle=none,fillstyle=solid,fillcolor=curcolor]
{
\newpath
\moveto(359.94313541,114.51733477)
\curveto(360.40700259,114.51733477)(360.85662824,114.40747149)(361.29201236,114.18774493)
\curveto(361.72739647,113.97208738)(362.05902082,113.69132566)(362.28688541,113.34545977)
\curveto(362.50661197,113.01586993)(362.65309634,112.63134845)(362.72633853,112.19189532)
\curveto(362.7914427,111.89078855)(362.82399478,111.41064532)(362.82399478,110.75146563)
\lineto(358.03273502,110.75146563)
\curveto(358.05308007,110.08821694)(358.20973697,109.55517657)(358.50270572,109.15234454)
\curveto(358.79567447,108.75358152)(359.24936913,108.55420001)(359.8637897,108.55420001)
\curveto(360.43752017,108.55420001)(360.89528384,108.74340899)(361.23708072,109.12182696)
\curveto(361.43239322,109.34155352)(361.57073957,109.59586667)(361.65211978,109.88476641)
\lineto(362.73244205,109.88476641)
\curveto(362.70395897,109.6446948)(362.60833723,109.37614011)(362.44557681,109.07910235)
\curveto(362.28688541,108.7861336)(362.10784895,108.54606199)(361.90846744,108.35888751)
\curveto(361.57480858,108.03336667)(361.16180403,107.81364011)(360.66945377,107.69970782)
\curveto(360.40496809,107.63460365)(360.10589582,107.60205157)(359.77223697,107.60205157)
\curveto(358.95843489,107.60205157)(358.26873762,107.89705483)(357.70314517,108.48706134)
\curveto(357.13755272,109.08113686)(356.8547565,109.91121498)(356.8547565,110.97729571)
\curveto(356.8547565,112.0271004)(357.13958723,112.87955808)(357.70924869,113.53466876)
\curveto(358.27891015,114.18977944)(359.02353905,114.51733477)(359.94313541,114.51733477)
\closepath
\moveto(361.69484439,111.62426837)
\curveto(361.65008528,112.10034259)(361.54632551,112.48079506)(361.38356509,112.76562579)
\curveto(361.08245832,113.29459714)(360.57993554,113.55908282)(359.87599673,113.55908282)
\curveto(359.37143944,113.55908282)(358.94826236,113.37597735)(358.60646548,113.00976641)
\curveto(358.26466861,112.64762449)(358.08359765,112.1857918)(358.06325259,111.62426837)
\lineto(361.69484439,111.62426837)
\closepath
\moveto(359.83937564,114.5295418)
\lineto(359.83937564,114.5295418)
\closepath
}
}
{
\newrgbcolor{curcolor}{0 0 0}
\pscustom[linestyle=none,fillstyle=solid,fillcolor=curcolor]
{
\newpath
\moveto(364.20949283,116.80004962)
\lineto(365.30812564,116.80004962)
\lineto(365.30812564,107.83398516)
\lineto(364.20949283,107.83398516)
\lineto(364.20949283,116.80004962)
\closepath
}
}
{
\newrgbcolor{curcolor}{0 0 0}
\pscustom[linewidth=1,linecolor=curcolor]
{
\newpath
\moveto(339.550069,130.33396778)
\lineto(339.550069,170.33396778)
}
}
{
\newrgbcolor{curcolor}{0 0 0}
\pscustom[linestyle=none,fillstyle=solid,fillcolor=curcolor]
{
\newpath
\moveto(339.550069,166.33396778)
\lineto(341.550069,164.33396778)
\lineto(339.550069,171.33396778)
\lineto(337.550069,164.33396778)
\lineto(339.550069,166.33396778)
\closepath
}
}
{
\newrgbcolor{curcolor}{0 0 0}
\pscustom[linewidth=0.5,linecolor=curcolor]
{
\newpath
\moveto(339.550069,166.33396778)
\lineto(341.550069,164.33396778)
\lineto(339.550069,171.33396778)
\lineto(337.550069,164.33396778)
\lineto(339.550069,166.33396778)
\closepath
}
}
\end{pspicture}

  \caption{Example Sensor Network}
  \label{fig:Example}
\end{figure}

\section{Components}
The network consists of two required components plus optional components.  The required components are a single controller and one or more responders.  Listeners may optionally be added as they simply monitor the bus traffic and do not transmit.  A BeagleBone Black listener is used to present the data from the network on web pages.  Note that the component boundaries are a little fuzzy in that some components can perform functions from multiple categories.

\subsection{Controller}
The controller is responsible for the operation of the bus.  It sends a request command to each node on the bus and waits for a response.  It may also send data messages.  Typically requests are done in a round-robin fashion -- request node 1, request node 2, etc.  If a node does not respond within a timeout period, it is marked as missing and the next node is polled.

It would probably be possible to combine the controller functions and the web gateway listener function into a BeagleBone Black or other similar embedded computer.  However, keeping them separate allows the controller to be simpler and keep the network functioning even if the web gateway has problems.  So, if presenting data collected on a web page is the primary function of the network, go ahead and combine them.  If, however the primary function of the network is monitoring and controlling something, it would be better to keep the web gateway separate.

\subsection{Responder}
One or more responders may be attached to the bus.  Each responder can produce one or more addresses of data.  The first address (address 0) is used for node identification.  It contains the name and the number of addresses produced by the node.  Additional addresses are produced depending on the sensors attached to the node and the node configuration.

\subsection{Listener}
Any number of listeners may be added to the network.  These simply monitor the network traffic and usually present it in a different format.  A listener can also be used to perform actions based on certain data on the network, though it would probably be better to use a responder that can report back what it is doing.

The BeagleBone Black is used as one example of a listener that presents the data on a web page.  Another example listener is an Arduino that simply listens to the bus and copies the characters to the host computer for display.

\subsection{Web Gateway}
The web gateway is a specialized listener.  It collects data from the bus and presents it as web pages and XML formatted data.  It is uses the \texttt{Ada-Web-Server} repository as the web server.  The RS-485 listener interface is implemented as an Ada Task that receives and processes data from the bus.  The back channel is another Ada Task that is used to transmit commands to the controller.  The use of a task is intended to prevent multiple commands being transmitted simultaneously and getting garbled.

\section{Protocol}
\subsection{Physical Layer}
The physical layer is RS-485 twisted pair.  CAT-5 cable is used because I happen to have a bunch of CAT-5 available and it works.  One pair is used leaving the other pairs available for other uses such as power distribution.  The baud rate is set to 115.2kbps.  This can be changed depending on the application.  The only constraint is that all nodes agree on the baud rate.

\subsection{Node Organization}
Each node is given a unique node ID.  The controller node is given ID 0.  The message format allows 32 bits for the node ID though typically small numbers would be used.  Each node can produce a number of addresses of data.  Address 0 is used to provide node information.  The information includes the number of addresses that the node can produce and the node's name.

\subsection{Messages}
Two types of messages are defined: Requests which only come from the controller and request a specific address of data from a specific node, and Responses which are the data produced by a responder in response to a request.

To aid in debugging, the format of the messages is in ASCII text.  The type of message is indicated by the first character, `\texttt{\@}' for requests and `\texttt{\#}' for responses.  The first two elements in both types of messages is the device ID and the address separated by a forward slash, `\texttt{/}'.  The end of the message is a percent sign, `\texttt{\%}' followed by a (currently unimplemented) checksum and a CR-LF.

All of the numbers are 32 bit unsigned integers expressed in hexadecimal, with leading zeros optional.  

\subsubsection{Requests}
Requests should only be sent by the controller node.  The first character in a request message is an ampersand, `\texttt{\@}'.  This is followed by the device ID in hex, followed by a forward slash, `\texttt{/}'.  Then the address in hex.  The address is optionally followed by a command code and value, each preceded by an apmersand `\texttt{\&}'.  This is followed by a percent sign, `\texttt{\%}' and a currently unused checksum in hex.  The message is terminated by a CR-LF.

If a command code is not specified, the default command code of 0 for \texttt{CMD\_READ} is used.  The currently defined command codes are in Table \ref{tab:commands}.

\begin{table}
  \centering
  \begin{tabular}{c l l}
    \hline
    Value & Symbol & Meaning \\
    \hline
    0 & \texttt{CMD\_READ} & Read data from the specified address.\\
    1 & \texttt{CMD\_RESET} & Perform a software reset.\\
    2 & \texttt{CMD\_WRITE} & Perform an application specific write.\\
  \end{tabular}
  \caption{Defined Command Codes}
  \label{tab:commands}
\end{table}

A typical request looks something like, `\texttt{@00000001/00000000\%FF}'.  This requests address 0 from device 1.  Since leading zeros are optional, the same message can also be expressed as, `\texttt{@1/0\%FF}'.

\subsubsection{Responses}
Responses are sent in response to a request from the controller node.  The first character in a response message is an octathorpe, `\texttt{\#}'.  This is followed by the device ID in hex, the address in hex, and the message type in hex, separated by a forward slash, `\texttt{/}', and ended by an ampersand, `\texttt{\&}'.  This is followed by the data fields.  These are up to 32 32-bit hex numbers separated by apmersands, `\texttt{\&}' and terminated by a percent sign, `\texttt{\%}'.  This is followed by a currently unused checksum in hex.  The message is terminated by a CR-LF.

The format of the responses flexible and the number of data fields sent depends on the type of message.  It may even vary between messages of the same type.  A typical response looks something like \texttt{\#2/1/4\&0\&0\&1E690\&175C0F1\&C775\%FF}.  This is a response from device ID 2, address 1 with message type 4 (\texttt{MSG\_TYPE\_BME280})

If the controller has data to send, it can send a response message without sending a request, as the request can be considered to be implied.

\subsubsection{Message Types}
To make the response messages easier to decode, each message has a message type field.  This indicates what to do with the rest of the data in the message.  Not all messages are currently implemented and more will likely be added.  The various message types are in Table \ref{tab:messagetype}.  Some of the message types have a subtype.  In particular, discretes have the defined types as shown in Table \ref{tab:disctype}, with more likely to be added.  The analog values have the defined types as shown in Table \ref{tab:analogtype}, with more likely to be added.

\begin{table}
  \centering
  \begin{tabular}{r l l}
    \hline
    Value & Symbol & Meaning \\
    \hline
    0 & \texttt{MSG\_TYPE\_UNKNOWN} & Undefined or not present.\\
    1 & \texttt{MSG\_TYPE\_EMPTY} & Everything is OK, but no data to send \\
    2 & \texttt{MSG\_TYPE\_NAK} & Address not supported \\
    3 & \texttt{MSG\_TYPE\_INFO} & Address 0 information message \\
    4 & \texttt{MSG\_TYPE\_BME280} & BME280 sensor values \\
    5 & \texttt{MSG\_TYPE\_DISCRETE} & Discretes \\
    6 & \texttt{MSG\_TYPE\_ANALOG} & Analog values (not yet implemented) \\
    7 & \texttt{MSG\_TYPE\_VERSION} & Version/Software ID (not yet implemented) \\
    8 & \texttt{MSG\_TYPE\_CCS811} & CCS811 sensor values \\
    9 & \texttt{MSG\_TYPE\_TSL2561} & TSL2651 sensor values \\
    10 & \texttt{MSG\_TYPE\_PCA9685} & PCA9685 PWM Controller\\
  \end{tabular}
  \caption{Defined Message Types}
  \label{tab:messagetype}
\end{table}

\begin{table}
  \centering
  \begin{tabular}{c l l}
    \hline
    Value & Symbol & Meaning \\
    \hline
    0 & \texttt{DISCRETE\_UNKNOWN} & Unknown discretes.\\
    1 & \texttt{DISCRETE\_CMD} & Command discretes.\\
    2 & \texttt{DISCRETE\_MIXED} & Mixed discrete types.\\
    3 & \texttt{DISCRETE\_SWITCH} & Discretes from switches.\\
  \end{tabular}
  \caption{Defined Discrete Types}
  \label{tab:disctype}
\end{table}

\begin{table}
  \centering
  \begin{tabular}{c l l}
    \hline
    Value & Symbol & Meaning \\
    \hline
    0 & \texttt{ANALOG\_UNKNOWN} & Unknown analog types.\\
    32 & \texttt{ANALOG\_MIXED} & Mixed analog types.\\
    64 & \texttt{ANALOG\_POT} & Analog values from potentiometers.\\
  \end{tabular}
  \caption{Defined Analog Types}
  \label{tab:analogtype}
\end{table}

\subsection{Back-Channel Commands}
While the commands are human usable, they are designed for ease of parsing, not ease of use.  It is expected that higher level software will provide a wrapper for them.  At this point, they should be considered to be experimental and subject to change.  There are two groups of commands:
\begin{itemize}
  \item General commands tell the controller to do something that applies to all nodes on the network.  The specific general commands currently defined are:
  \begin{itemize}
    \item \texttt{!A} Tell all nodes to turn their indicator LED off.
    \item \texttt{!B} Tell all nodes to turn their indicator LED on.
    \item \texttt{!C} Tell all nodes to toggle their indicator LED.
    \item \texttt{!D} Tell all nodes to identify themselves.
  \end{itemize}
  \item Directed commands are sent to a specific node telling it to do something.  The currently defined directed commands are:
  \begin{itemize}
    \item \texttt{!R<node>,} Request node it node id of \texttt{<node>} to perform a software reset.
    \item \texttt{!S<cmd>,<node>,<arg>,} Send an arbitrary command, \texttt{<cmd>}, to node id \texttt{<node>}, with command arguments, \texttt{<arg>}.
  \end{itemize}
  Where all numbers are in hexidecimal.  Note that the trailing comma is required.
\end{itemize}

\section{Logging}
Data can be logged to files for later analysis.  Provisions exist for logging to be started and stopped and started on a global basis or by record type.  Starting global logging creates the files for the data and stopping global logging closes the files.  To log data, logging has to be started globally and logging of the desired data has to be enabled.  The data records generally supported for logging are:
\begin{itemize}
  \item Node information records
  \item BME280 Temperature, pressure, and humidity sensor
  \item CCS811 eCO$_2$ and TVOC (total volatile organic compounds) air quality sensor
  \item TLS2561 light sensor
\end{itemize}
The discretes and analogs are customized for the application and thus any logging of them would also have to be customized.  The existing logging code can be used as an example.  Note that logging can generate an enormous amount of data, so be prepared.
%======================================
\section{API}
A number of the defined URLs return XML information.  This can be used to create an API that can be used by programs beyond just a web browser.

%---------------------------------------------------------------
\subsection{Graphics}
Two of the URLs return an image in SVG format.  Typically, it is used for display on a web page, but could be used for other purposes.

%---------------------------------------------------------------
\subsubsection{Dial}
The URL is \texttt{/Dial}.  The parameters are:
\begin{itemize}
  \item \texttt{min} - Specify the minimum value shown on the dial (required).
  \item \texttt{max} - Specify the maximum value shown on the dial (required).
  \item \texttt{value} - Specify the value indicated by the dial's pointer (required).
\end{itemize}

%---------------------------------------------------------------
\subsubsection{Thermometer}
The URL is \texttt{/Thermometer}.  The parameters are:
\begin{itemize}
  \item \texttt{min} - Specify the minimum value shown on the thermometer (required).
  \item \texttt{max} - Specify the maximum value shown on the thermometer (required).
  \item \texttt{value} - Specify the value indicated by the thermometer (required).
\end{itemize}

%---------------------------------------------------------------
\subsection{Data and Control}
The rest of the URLs either return data or are used to control the system.  All of these return an XML message containing either requested data or an error message.

%---------------------------------------------------------------
\subsubsection{Send a Command to a Node}
This function is used to send a command to the controller via the back channel.  The URL is \texttt{/xml/Command}.  There is one parameter:
\begin{itemize}
  \item \texttt{command} - The command to send as text (required).
\end{itemize}

The returned data is returned in a \texttt{<xml>} block with the following contents:
\begin{itemize}
  \item \texttt{command} - Echoes the sent command, if the \texttt{command} parameter was present, or
  \item \texttt{error} - Provides an error message if the \texttt{command} parameter was not present.
\end{itemize}

%---------------------------------------------------------------
\subsubsection{Get System Counters}
The URL is \texttt{/xml/Counter}.  There are no parameters.  The returned data is in a \texttt{<xml>} block with the following contents:
\begin{itemize}
  \item \texttt{<counter>} - XML container for number of web requests handled counter.
  \item \texttt{<tasks>} - XML contained for number of currently active web handler tasks.
  \item \texttt{<rs485>} - XML container for RS-485 state machine activity counter.
\end{itemize}

%---------------------------------------------------------------
\subsubsection{Get Maximum Device Number on the Network}
The URL is \texttt{/xml/Devices}.  There are no parameters.  The returned data is in a \texttt{<xml>} block with the following contents:
\begin{itemize}
  \item \texttt{<length>} - XML container for the maximum device number found on the network.  Used to set the length of data structures.
\end{itemize}

%---------------------------------------------------------------
\subsubsection{Get Data from a Device on the Network}
While this call is simple, this returns the most complex set of data.  The URL is \texttt{/xml/DevData}.  There is one parameter:
\begin{itemize}
  \item \texttt{device} - Contains the node ID of the device to get data from (required).
\end{itemize}

The returned data is in an \texttt{<xml>} block.  The following components may, or may not be present depending on the configuration of the node:
\begin{itemize}
  \item \texttt{<info>} - This should always be present.
  \begin{itemize}
    \item \texttt{<validity>} - Validity status for info data.
    \item \texttt{<aging>} - Number of seconds since info message received.
    \item \texttt{<addresses>} - Number of addresses supported by device.
    \item \texttt{<name>} - Name of the node.
  \end{itemize}
  \item \texttt{<bme280>} - Present if node is configured with a BME280 sensor.
  \begin{itemize}
    \item \texttt{<validity>} - Validity status for BME280 data.
    \item \texttt{<aging>} - Number of seconds since BME280 message received.
    \item \texttt{<bme280\_status>} - Status of the BME280 sensor reported by node.
    \item \texttt{<bme280\_age>} - Number of frames since BME280 sensor data read.
    \item \texttt{<bme280\_temp\_c>} - Temperature in degrees Celcius.
    \item \texttt{<bme280\_pressure\_pa>} - Atmospheric pressure in Pascals.
    \item \texttt{<bme280\_humidity>} - Relative humidity in percent.
  \end{itemize}
  \item \texttt{<discretes>} - Present if node is configured to transmit discretes.
  \begin{itemize}
    \item \texttt{<validity>} - Validity status for discretes data.
    \item \texttt{<aging>} - Number of seconds since discretes message received.
    \item \texttt{<disc\_type>} - Code indicating the type of these discretes.
    \item \texttt{<disc\_value>} - Value of the discretes as a 32 bit unsigned integer.  One bit for each discrete.
  \end{itemize}
  \item \texttt{<analogs>} - Present if node is configured to transmit analog values.
  \begin{itemize}
    \item \texttt{<validity>} - Validity status for analog data.
    \item \texttt{<aging>} - Number of seconds since analog message received.
    \item \texttt{<analog\_type>} - Code indicating the type of these analog values.
    \item \texttt{<analog\_count>} - The number of analog value provided in this message.
    \item \texttt{<value>} - Value of analog reading.  This entry is repeated \texttt{analog\_count} times.  Once for each analog value.
  \end{itemize}
  \item \texttt{<ccs811>} - Present if node is configured with a CCS811 sensor.
  \begin{itemize}
    \item \texttt{<validity>} - Validity status for CCS811 data.
    \item \texttt{<aging>} - Number of seconds since CCS811 message received.
    \item \texttt{<ccs811\_status>} - Status of the CCS811 sensor reported by node.
    \item \texttt{<ccs811\_age>} - Number of frames since CCS811 sensor data read.
    \item \texttt{<ccs811\_eco2>} - Measurement of CO$_2$ concentration.
    \item \texttt{<ccs811\_tvoc>} - Measurement of total volatile organic compounds (TVOC) concentration.
  \end{itemize}
  \item \texttt{<tsl2561>} - Present if node is configured with a TSL2561 sensor.
  \begin{itemize}
    \item \texttt{<validity>} - Validity status for TSL2561 data.
    \item \texttt{<aging>} - Number of seconds since TSL2561 message received.
    \item \texttt{<tsl2561\_status>} - Status of the TSL2561 sensor reported by node.
    \item \texttt{<tsl2561\_age>} - Number of frames since TSL2561 sensor data read.
    \item \texttt{<tsl2561\_data0>} - Measurement of one band of light.
    \item \texttt{<tsl2561\_data1>} - Measurement of one band of light.
    \item \texttt{<tsl2561\_lux>} - Calculated illumination in Lux.
  \end{itemize}
  \item \texttt{<pca9685>} - Present if node is configured with a PCA9685 PWM controller.
  \begin{itemize}
    \item \texttt{<validity>} - Validity status for PCA9685 data.
    \item \texttt{<aging>} - Number of seconds since PCA9685 message received.
    \item \texttt{<channel>} - One of these entries for each of the 16 PWM channels.  Each entry contains three comma separated values.  The values are:
    \begin{itemize}
      \item A boolean indicating if this channel has been set.
      \item An unsigned integer indicating the PWM ``On'' count.
      \item An unsigned integer indicating the PWM ``Off'' count.
    \end{itemize}
  \end{itemize}
  \item \texttt{<error>} - Present if an error has occurred.
\end{itemize}

%---------------------------------------------------------------
\subsubsection{Control Debugging Messages}
Various debugging messages can be displayed on the console.  This entry can be used to turn messages on or off and to report the status of debugging messages.  The URL is \texttt{/xml/Debug}.  There are several parameters:
\begin{itemize}
  \item \texttt{rs485.char} - Set to `T' to print characters that are read from the RS-485 bus (optional).
  \item \texttt{rs485.msg} - Set to `T' to print the type of message that is read from the RS-485 bus (optional).
  \item \texttt{http.head} - Set to `T' to print the headers from each HTTP request (optional).
  \item \texttt{http.msg} - Set to `T' to print the HTTP requests (optional).
  \item \texttt{web.dbg} - Set to `T' to print debugging messages from the web server (optional).
\end{itemize}

The returned data is in a \texttt{<xml>} block with the following contents:
\begin{itemize}
  \item \texttt{<rs485.char>} - \texttt{TRUE} if RS-485 character debugging is enabled.
  \item \texttt{<rs485.msg>} - \texttt{TRUE} if RS-485 message debugging is enabled.
  \item \texttt{<http.head>} - \texttt{TRUE} if HTTP header debugging is enabled.
  \item \texttt{<http.msg>} - \texttt{TRUE} if HTTP request debugging is enabled.
  \item \texttt{<web.dbg>} - \texttt{TRUE} if web debugging is enabled.
\end{itemize}

%---------------------------------------------------------------
\subsubsection{Control Data Logging}
Starting logging is a two step process.  First the specific messages to log need to be enabled.  Then logging needs to be turned on.  The URL is \texttt{/xml/Log}.  There are several parameters:
\begin{itemize}
  \item \texttt{log.info} - Set to `T' to enable logging of info messages (optional).
  \item \texttt{log.BME280} - Set to `T' to enable logging of BME280 messages (optional).
  \item \texttt{log.CCS811} - Set to `T' to enable logging of CCS811 messages (optional).
  \item \texttt{log.TSL2561} - Set to `T' to enable logging of TSL2561 messages (optional).
  \item \texttt{logging} - Set to `T' to turn logging on (optional).
\end{itemize}

The returned data is in a \texttt{<xml>} block with the following contents:
\begin{itemize}
  \item \texttt{<log.info>} - \texttt{TRUE} if info message logging is enabled.
  \item \texttt{<log.BME280>} - \texttt{TRUE} if BME280 message logging is enabled.
  \item \texttt{<log.CCS811>} - \texttt{TRUE} if CCS811 message logging is enabled.
  \item \texttt{<log.TSL2561>} - \texttt{TRUE} if TSL2561 message logging is enabled.
  \item \texttt{<log.type>} - Set to the type of log files.  Currently either \texttt{NONE} or \texttt{CSV}.
\end{itemize}

%---------------------------------------------------------------
\subsubsection{Get Device Name}
The URL is \texttt{/xml/Name} name internal.  There is one parameter:
\begin{itemize}
  \item \texttt{device} - Specify the device number to get information from (required).
\end{itemize}

The returned data is in a \texttt{<xml>} block with the following contents:
\begin{itemize}
  \item \texttt{<info>} - Block containing device information, if present.  The information consists of the following:
  \begin{itemize}
    \item \texttt{<validity>} - Validity code for the data.
    \item \texttt{<aging>} - The time in seconds since the info record was received from the node.
    \item \texttt{<presence>} - The time in seconds since any records were received from the node.
    \item \texttt{<addresses>} - The number of addresses supported by the node.
    \item \texttt{<name>} - The node's name as a string.
  \end{itemize}
  \item \texttt{<error>} - Block containing an error message if device information cannot be provided.
\end{itemize}


\end{document}
